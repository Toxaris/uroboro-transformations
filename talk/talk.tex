\documentclass[xcolor=svgnames]{beamer}

\usepackage[utf8]    {inputenc}
\usepackage[T1]      {fontenc}
\usepackage[english] {babel}

\usepackage{amsmath,amsfonts,graphicx,listings}
\usepackage{beamerleanprogress}
\usepackage[numbers]{natbib}

\frenchspacing

\title
  [Automatic Program Transformations ...\hspace{2em}]
  {Automatic \only<-4>{{\color<3>{red}Program Transformations}}\\ \only<-4>{{\color<4>{red}for a Language}} with \only<-4>{{\color<2>{red}Copattern Matching}}}

\author
  [Julian Jabs]
  {\only<1>{Julian Jabs}}

\date
  {\only<1>{October 19, 2015}}

\begin{document}

\begin{frame}<1>[label=title]
  \titlepage
\end{frame}

\begin{frame}{Overview}
\tableofcontents
\end{frame}

\section
  {Motivation}

\begin{frame}
  {Motivation}

  \begin{itemize}
    \item Important feature of many functional programming languages: \textbf{first-class functions}

    \item \textbf{First-order} programs (without first-class functions) VS. \textbf{higher-order} programs (with first-class functions)
  \end{itemize}
\end{frame}

\begin{frame}
  {Motivation}

  \begin{itemize}
    \item First-order programs: certain properties beneficial for machine processing, recognizable as abstract machines

    \item Higher-order programs: beneficial for human understandability

    \item Want to find and automatize transformations between first-order and higher-order programs which preserve the program's semantics

    \item Concepts connected to this: \textbf{patterns} and \textbf{copatterns}
  \end{itemize}
\end{frame}

\section
  {Background}

\subsection
  {Patterns and copatterns}

\againframe<2>{title}

\begin{frame}[fragile]
  {Patterns and copatterns}

  \begin{block}{Data and patterns}
    \begin{lstlisting}
    data Nat where
      zero(): Nat
      succ(Nat): Nat

    function add(Nat, Nat): Nat where
      add(zero(), n)  = n
      add(succ(m), n) = succ(add(m, n))
    \end{lstlisting}
  \end{block}
\end{frame}

\begin{frame}[fragile]
  {Patterns and copatterns}

  \begin{block}{Codata and copatterns\citep{abel13copatterns}}
    \begin{lstlisting}
    codata Stream where
      Stream.head(): Nat
      Stream.tail(): Stream

    function repeat(Nat): Stream where
      repeat(n).head() = n
      repeat(n).tail() = repeat(n)
    \end{lstlisting}
  \end{block}
\end{frame}

\begin{frame}
  {Patterns and copatterns}

  \begin{block}{Coverage}
    A set of equations \textbf{covers} the function when the coverage for their lhss can be derived with \textbf{variable splits} and \textbf{result splits}.

  (Adapted from \citet{abel13copatterns})

   \underline{Example:} Suppose (co)data types \texttt{Nat} and \texttt{Stream} as shown before, and a function with signature \texttt{f(Nat): Stream}
    \begin{itemize}
      \item Variable split: \texttt{f(x)} $\to$ \texttt{f(zero())}, \texttt{f(succ(x))}

      \texttt{x} in \texttt{f(succ(x))} can be further variable split.
      \item Result split: \texttt{f(x)} $\to$ \texttt{f(x).head()}, \texttt{f(x).tail()}

      \texttt{f(x).tail()} itself can be further result split.
    \end{itemize}

  Afterwards, further result and/or variable splitting could be done.
  \end{block}
\end{frame}

\subsection
  {(De|Re)functionalization}

\againframe<3>{title}

\begin{frame}
  {(De|Re)functionalization}

  \begin{block}{Defunctionalization \citep{reynolds72definitional}}
    From \textbf{higher-order} to \textbf{first-order} programs.

    Introduces a \textbf{data type} for functions:

    \textbf{One constructor per first-class function}.
  \end{block}

  \begin{block}{Refunctionalization (original form) \citep{danvy09refunctionalization}}
    \textbf{Left inverse} of defunctionalization.
  \end{block}
\end{frame}

\begin{frame}[fragile]
  {(De|Re)functionalization}

  \begin{block}{Defunc. example}
    \begin{lstlisting}
    function all(Fun, List): Bool where
     all(f, x:xs) = (f x) && all(f, xs)

    function even: Nat -> Bool where
     even(zero())  = true()
     even(succ(n)) = not(even(n))

    function odd: Nat -> Bool where
      odd(n) = not(even(n))
    \end{lstlisting}
  \end{block}
\end{frame}

\begin{frame}[fragile]
  {(De|Re)functionalization}

  \begin{block}{Defunc. example, defunc'ed}
    \begin{lstlisting}[escapechar=!]
    data Fun where
      !\alert{even()}!: Fun
      !\alert{odd()}!: Fun

    function all(Fun, List): Bool where
     all(f, x:xs) = apply(f, x) && all(f, xs)

    function apply(Fun, Nat): Bool where
     apply(even(), zero())  = true()
     apply(even(), succ(n)) =
       not(apply(even(), n))
     apply(odd(), n) = not(apply(even, n))
    \end{lstlisting}
  \end{block}
\end{frame}

\begin{frame}[fragile]
  {(De|Re)functionalization}

  \textbf{Refunc'ing} the last program gives back the original example.

  \textbf{But:} With the original form of refunctionalization, one can only refunc. programs in the image of defunctionalization.

  \begin{block}{Counterexample}
    \begin{lstlisting}[escapechar=!]
    data Fun where
      fun1(): Fun
      fun2(): Fun

    function apply1(Fun, X): Y where
      ...
    
    function apply2(Fun, X): Y where
      ...
    \end{lstlisting}
  \end{block}
\end{frame}

\begin{frame}[fragile]
  {(De|Re)functionalization}
  \begin{block}{Counterexample, refunc'ed}
    \begin{lstlisting}[escapechar=!]
    codata Fun where
      Fun.apply1(X): Y
      Fun.apply2(X): Y

    function fun1(): Fun where
      ...
    
    function fun2(): Fun where
      ...
    \end{lstlisting}
  \end{block}
\end{frame}

\begin{frame}[fragile]
  {(De|Re)functionalization}

Solution of \citet{rendel15automatic}: 

\textbf{``Generalize first-class functions to arbitrary codata''}

\begin{block}{First-class functions as codata}
    \begin{lstlisting}[escapechar=!]
    codata Fun where
      Fun.apply(Nat): Bool

    function all(Fun, List): Bool where
     all(f, x:xs) = f.apply(x) && all(f, xs)

    function even(): Fun where
     even.apply(zero())  = true()
     even.apply(succ(n)) = not(even.apply(n))
    \end{lstlisting}
  \end{block}

\end{frame}

\begin{frame}
  {(De|Re)functionalization}

\begin{block}{(Co)Data Fragments (\citeauthor{rendel15automatic})}
  \textbf{Data Fragment}: only data types, \underline{limited} pattern matching

  <->

  \textbf{Codata Fragment}: only codata types, \underline{limited} copattern matching
\end{block}

De- and refunctionalization are \textbf{symmetric} on these fragments.

\end{frame}

\begin{frame}
  {(De|Re)functionalization}

  \begin{block}{Defunctionalization (data/codata)}
    From \textbf{Codata Fragment} to \textbf{Data Fragment} programs.

    Replaces each \textbf{codata type def.} with a \textbf{data type def.}:

    \textbf{One constructor per function definition}.

    \textbf{One function def. per destructor}.
  \end{block}

  \begin{block}{Refunctionalization (data/codata)}
    From \textbf{Data Fragment} to \textbf{Codata Fragment} programs.

    Replaces each \textbf{data type def.} with a \textbf{codata type def.}:

    \textbf{One destructor per function definition}.

    \textbf{One function def. per constructor}.
  \end{block}
\end{frame}

\section
  {Transformations for Uroboro}

\againframe<4>{title}

\begin{frame}<1>[label=contribs]
  {Transformations for Uroboro}

  \underline{\textbf{Contributions of this work:}}

  \begin{itemize}
  \item \alert<2>{Formally defines the language \textbf{Uroboro}: \textbf{a common superset} of both the Data and Codata Fragment}

  \item \alert<3>{Extends de- and refunctionalization to all of Uroboro.}

  \item \alert<4>{Identifies a generalization for certain steps in these transformations, called \textbf{Extraction}.}

  \item \alert<5>{By-product: Shines some light on an asymmetry between de- and refunctionalization.}
   \end{itemize}
\end{frame}

\againframe<2>{contribs}

\begin{frame}[fragile]
  {Transformations for Uroboro: Uroboro}

  \begin{block}{Uroboro example w/ patterns \underline{and} copatterns in one lhs}
    \begin{lstlisting}[escapechar=!]
    data Expr where
      var(Nat): Expr
      fun(Var, Expr): Expr

    codata Val where
      Val.apply(Val): Val

    function eval(Expr, Env): Val where
      ...
      !\alert{eval(fun(var,expr), env).apply(val)}! = 
        eval(expr, env.extendWith(var, val))
    \end{lstlisting}
  \end{block}

  Note: equations are \textbf{not ordered}
\end{frame}

\againframe<3>{contribs}

\begin{frame}[fragile]
  {Transformations for Uroboro: De- and refunctionalization}

  \begin{block}{Two phases for de- and refunctionalization}
    \begin{itemize}
      \item \textbf{Unnesting}: brings (co)patterns to the limited form allowed in the (Co)Data Fragment
      \item \textbf{Core (de|re)func.:} essentially the (de|re)functionalization for the (Co)Data Fragment
    \end{itemize}
  \end{block}

  \textbf{Unnesting}: adapted for Uroboro from \citet{setzer14unnesting}
\end{frame}

\begin{frame}[fragile]
  {Transformations for Uroboro: De- and refunctionalization}

  \begin{block}{Uroboro example, unnested}
    \begin{lstlisting}[escapechar=!]
    ...

    function eval(Expr, Env): Val where
      ...
      eval(fun(var,expr), env) =
        aux(var, expr, env)
        

    !\alert{function aux(Expr, Expr, Env): Val}! where
      aux(var, expr, env).apply(val) =
        eval(expr, env.extendWith(var, val))
    \end{lstlisting}
  \end{block}

  Call this: \textbf{Extraction} of a destructor
\end{frame}

\begin{frame}
  {Transformations for Uroboro: De- and refunctionalization}

\underline{In general:}

Unnesting works on the coverage trees and ``undoes'' the coverage steps:
\begin{itemize}
\item Variable split: undone by constructor extraction

Ex.: \texttt{f(c1()) = t1}, \texttt{f(c2()) = t2} \\
\qquad $\to$ \texttt{f(x) = aux(x)}, \texttt{aux(c1()) = t1}, \texttt{aux(c2()) = t2}

\item Result split: undone by destructor extraction

(ex. on previous slide)
\end{itemize}

\textbf{Important:} The extractions must be carried out in exactly the reverse order of the coverage derivation---otherwise semantics preservation is not guaranteed.
\end{frame}

\begin{frame}[fragile]
  {Transformations for Uroboro: De- and refunctionalization}

  \begin{block}{Unnesting order example}
    \begin{block}{}
      \begin{lstlisting}[escapechar=!]
         function fun(S): T where
           fun(c1()).d1() = t1
           fun(c1()).d2() = t2
           fun(c2())      = t3
      \end{lstlisting}
    \end{block}
    Clearly, last coverage derivation step was the result split; thus extract destr. first!
    \begin{block}{}
      \begin{lstlisting}[escapechar=!]
         function fun(S): T where
           fun(c1()) = aux()
           fun(c2()) = t3
      \end{lstlisting}
    \end{block}
  \end{block}
\end{frame}

\begin{frame}[fragile]
  {Transformations for Uroboro: De- and refunctionalization}

  \begin{block}{Unnesting order example (cont'd)}
    First extracting the constructor changes the semantics!
    \begin{block}{}
      \begin{lstlisting}[escapechar=!]
         function fun(S): T where
           fun(x).d1() = ...
           fun(x).d2() = ...
           fun(x)      = ...
      \end{lstlisting}
    \end{block}
    The equations overlap, unlike the original equations.
  \end{block}
\end{frame}

\begin{frame}[fragile]
  {Transformations for Uroboro: De- and refunctionalization}
The unnested program can now be passed to core defunc. ...

  \begin{block}{Uroboro example, defunc'ed}
    \begin{lstlisting}[escapechar=!]
    !\alert{data Val where}!
      !\alert{aux(Expr, Expr, Env): Val}!

    function eval(Expr, Env): Val where
      ...
      eval(fun(var,expr), env) =
        aux(var, expr, env)        

    !\alert{function apply(Val, Val): Val}! where
      apply(aux(var, expr, env), val) =
        eval(expr, env.extendWith(var, val))
    \end{lstlisting}
  \end{block}
\end{frame}

\begin{frame}[fragile]
  {Transformations for Uroboro: De- and refunctionalization}
... or to core refunc.

  \begin{block}{Uroboro example, refunc'ed}
    \begin{lstlisting}[escapechar=!]
    !\alert{codata Expr where}!
      !\alert{Expr.eval(Env): Val}!

    !\alert{function var(Nat): Expr}! where
      ...

    !\alert{function fun(Var, Expr): Expr}! where     
      fun(var, expr).eval(env) =
        aux(var, expr, env)

    ...
    \end{lstlisting}
  \end{block}
\end{frame}

\againframe<4>{contribs}

\begin{frame}
  {Transformations for Uroboro: Extraction}

  In general: \textbf{Extractions} make a program simpler \textbf{syntactically}, while preserving its \textbf{semantics} (cf. refactoring).

  In Uroboro: Can either extract destructors or constructors.

  Semantics preservation proof: Adapted and significantly extended from \citet{setzer14unnesting}

\end{frame}

\againframe<5>{contribs}

\begin{frame}
  {Transformations for Uroboro: An asymmetry}

  In the Data Fragment: constructors only allowed in the first argument.

  not allowed: \texttt{f(c(), c'())}

  allowed: \texttt{f(c(), x)}

  No analogous limitation for the Codata Fragment possible!

  \texttt{f().d()}

  Destructors observe output usage in \textbf{only one dimension}.

  Compare: the difference between natural deduction and arbitrary sequent calculus, which allows multiple formulas on right-hand sides.
\end{frame}

\section{Conclusion}

\begin{frame}
  {Conclusion}
  \begin{itemize}
    \item \citet{rendel15automatic} made de- and refunctionalization symmetric on two fragment languages
    \item This work: makes them more practical by extending them to a full language (= common generalization of the fragments)
    \item Adapted the unnesting algorithm of \citet{setzer14unnesting} to achieve this
    \item Identified and proved correct a generalization of the unnesting steps
    \item Recognized an asymmetry between patterns and copatterns
  \end{itemize}

  \centering\textbf{Thanks!}
\end{frame}

\begin{frame}%[allowframebreaks]
  {Bibliography}
  \bibliographystyle{plainnat}
  \tiny\bibliography{bibliography}
\end{frame}

\end{document}

