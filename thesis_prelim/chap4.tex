\chapter{Automatic de- and refunctionalization}

%%-- under construction

...

We present an algorithm that defunctionalizes arbitrary Uroboro programs with copattern coverage, and an algorithm that refunctionalizes such programs. Both algorithms are made up of two phases, in order:
\begin{enumerate}
\item Unnesting, and
\item core de-/refunctionalization.
\end{enumerate}

In short, the entire process can be described as follows. By unnesting them as far as necessary, the preprocessing phases brings the lhss of the program to the form accepted by the core de-/refunctionalization, which is essentially the two-way transformation of Rendel et al.

The unnesting is done by a parameterized algorithm $\textsf{Unnest}_i$, where the parameter $i$ can be either $d$ or $r$, standing for defunctionalization and refunctionalization, respectively. The parameter determines when the algorithm may stop, such that its results are legal inputs for de- or refunctionalization. The unnesting algorithm is essentially the copattern unnesting algorithm of Setzer et al., adapted to Uroboro.

We give the intended domain and range for each of the algorithms, all of which are specific fragments of Uroboro. Call the set of all Uroboro programs with copattern coverage $\mathbf{U}_{cc}$; the other fragments are defined below.
\begin{align*}
& \textsf{Unnest}_i: \mathbf{U}_{cc} \to \mathbf{U}_{cc}[fdef_{un,i}] \text{ for } i \in \{d,r\} \\
& \textsf{CoreDefunc}: \mathbf{U}_{cc}[fdef_{un,d}] \to \mathbf{U}^{defunc}_{cc} \\
& \textsf{CoreRefunc}: \mathbf{U}_{cc}[fdef_{un,r}] \to \mathbf{U}^{refunc}_{cc}
\end{align*}

\begin{definition}[Unnested Fragments for \textsf{CoreDefunc} and \textsf{CoreRefunc}]
The fragments $\mathbf{U}_{cc}[fdef_{un,d}]$, $\mathbf{U}_{cc}[fdef_{un,r}]$, for lhss unnested as far as necessary for \textsf{CoreDefunc} and \textsf{CoreRefunc}, respectively, are induced by the function definition fragments $fdef_{un,d}$ and $fdef_{un,r}$, respectively, both defined below using EBNF rules. For these, the basic EBNF rules from the definition of the syntax of Uroboro are reused.
\begin{align*}
fdef_{un,d} &::= eqn_{nd}^* ~ | ~ eqn_d^* \\
eqn_{nd} &::= fun(x^*).des(y^*) = t \\
eqn_d &::= fun(p^*) = t \\
\end{align*}
\begin{align*}
fdef_{un,r} &::= eqn_{nr}^* ~ | ~ eqn_r^* \\
eqn_r &::= fun(x^*).des(y^*)^* = t \\
eqn_{nr} &::= fun(x^*, con(y^*), z^*) = t \\
\end{align*}
\end{definition}

\begin{definition}[Defunctionalized fragment]
A Uroboro program is in the defunctionalized fragment $\mathbf{U}^{defunc}_{cc}$ if and only if it has copattern coverage and contains no codata type definitions.
\end{definition}

\begin{definition}[Refunctionalized fragment]
A well-typed Uroboro program is in the refunctionalized fragment $\mathbf{U}^{refunc}_{cc}$ if and only if it has copattern coverage and contains no data type definitions.
\end{definition}

For each algorithm it will be shown, in its respective section, that it actually has the range specified above. In total, four important properties will be shown for each algorithm: Termination, correct range, preservation of well-typedness, and preservation of semantics in a (weak) bisimulation.

The rest of the section is organized as follows. The first section describes the unnesting algorithm, shows why the three properties hold, and gives applies it to an example. Section 2 and 3 do the same for the core de- and refunctionalization, respectively.

\section{Unnesting}
\label{sec:unn}

We adapt the translation algorithm from section 3.3 of the paper of Setzer et al. to Uroboro. Their algorithm translates equations step-by-step depending upon the derivation of the copattern coverage of their lhss. Our adaptation only has steps which concern patterns that have a counterpart in Uroboro\footnote{I.e., not those for applications, units, and pairs. In the adapted algorithm for Uroboro, the last one is merged into the steps for destructors and constructors, while the others are irrelevant because Uroboro doesn't have the corresponding higher-order constructs.}. Each such translation step can be considered an extraction as defined in chapter 3.

\subsection{Simple patterns}

Before we start with the translation algorithm, we adapt and introduce, respectively, two notions of simplicity for patterns, the second of which corresponds to the intended range of the translation algorithm. We first adapt the definition of a simple pattern, as given by Setzer et al., to Uroboro. Call a copattern \textit{simple} if it is of one of the three forms $fun(\overline{x})$, $fun(\overline{x}).des(\overline{y})$, or $fun(\overline{x}, con(\overline{y}), \overline{z})$.

Next, we generalize this definition to define \textit{sufficiently simple patterns} for a purpose, which is either de- or refunctionalization. Call a copattern \textit{sufficiently simple for defunctionalization} if it is of one of the two forms $fun(\overline{x}).des(\overline{y})$, $fun(\overline{p})$, and \textit{sufficiently simple for refunctionalization} if it is of one of the two forms $fun(\overline{x}).\overline{des(\overline{y})}$, $fun(\overline{x}, con(\overline{y}), \overline{z})$. We will also say that a copattern is sufficiently simple for $i$, for $i \in \{d,r\}$, and mean that it is sufficiently simple for defunctionalization or refunctionalization, respectively.

Those notions correspond directly to the forms of lhss allowed in the unnested fragments for \textsf{CoreDefunc} and \textsf{CoreRefunc}, respectively. It is clear that each simple copattern is also sufficiently simple for both de- and refunctionalization.

\subsection{Translation algorithm}
\label{ssec:unntransl}

We describe the algorithm for $\textsf{Unnest}_i$, adapted from section 3.3 of the paper of Setzer et al. to Uroboro, as lined out above.

\begin{algorithm}[$\textsf{Unnest}_i$]

Like Setzer et al., we consider the last step in the derivation of the copattern coverage for the lhss of a function definition. Just like them, we are not interested in the case where the last step is C\textsubscript{Head}, since that means the lhs is already simple, and we terminate. When the last step isn't C\textsubscript{Head}, but the lhss are all sufficiently simple for $i$ nonetheless, we also terminate.

\begin{prooftree}
\AxiomC{$fun \lhd | ~ Q ~ (q)$}
\RightLabel{\textbf{C}}
\UnaryInfC{$fun \lhd | ~ Q ~ (q_i)_{i \in I}$}
\end{prooftree}

Also like Setzer et al., we then introduce a fresh auxiliary function and translate the program depending on \textbf{C}. As stated before, these translations are slightly adapted to fit the framework of Uroboro, such that two remain, one for destructors and one for constructors. More importantly, both of those are in fact extractions as defined in chapter 3, namely destructor and constructor extraction. 

\begin{enumerate}
\item \textbf{C} is C\textsubscript{Des}, and the last derivation step looks as follows.

\begin{prooftree}
\AxiomC{$fun \lhd | ~ Q ~ (q^\tau)$}
\RightLabel{C\textsubscript{Des}}
\UnaryInfC{$fun \lhd | ~ Q ~ (q.des(\overline{x^{des}}))_{des \in Dess_\tau}$}
\end{prooftree}

Apply extraction \textsf{ExtractDes} targeting the equations with lhss $q.des(\overline{x^{des}})$, for all $des \in Dess_\tau$, to the function definition for $fun$. From this we obtain the changed function definition $def'_{fun}$ for $fun$ and the auxiliary function definition. The lhss of $def'_{fun}$ are
\[
Q \cup \{ \texttt{get}(q.des(\overline{x^{des}})) ~ | ~ des \in Dess_\tau \} = Q \cup \{q\},
\]
where the pair of \texttt{get} and \texttt{putback} is the lens underlying \textsf{ExtractDes}. This set of lhss is the same as that in $fun \lhd | ~ Q ~ (q^\tau)$ from the derivation step above. The lhss of the auxiliary function definition are
\begin{multline*}
\{ q_{\zeta_r} ~ | ~ r \in T \} = \{ \texttt{putback}(\langle \texttt{get}(q_r) \rangle^{aux}, q_r) ~ | ~ r \in T \} \\
= \{ \sigma^{q_r}_\pi(aux(\langle q \rangle^{vars})) ~ | ~ r \in T \} = \{ aux(\langle q \rangle^{vars}).des(\overline{x^{des}}) ~ | ~ des \in Dess_\tau \} \\
= \{ aux(\langle q \rangle^{vars}).des(\overline{x^{des}}) ~ | ~ des \in Dess_\tau \},
\end{multline*}
where $\zeta$ is the function in the triple that is \textsf{ExtractCon}(p). It is clear that each $q_{\zeta_r}$ is simple and that $\{ q_{\zeta_r} ~ | ~ r \in T \}$ covers $aux$:
\begin{prooftree}
\AxiomC{}
\RightLabel{C\textsubscript{Head}}
\UnaryInfC{$aux \lhd | ~ aux(x, \overline{y})$}
\RightLabel{C\textsubscript{ResSplit}}
\UnaryInfC{$aux \lhd | ~ (q_{\zeta_r})_{r \in T}$}
\end{prooftree}

\item \textbf{C} is C\textsubscript{Con}, and the last derivation step looks as follows. 

\begin{prooftree}
\AxiomC{$fun \lhd | ~ Q ~ (q(x^\tau))$}
\RightLabel{C\textsubscript{Con}}
\UnaryInfC{$fun \lhd | ~ Q ~ (q[x := con(\overline{y^{con}})])_{con \in Cons_\tau}$}
\end{prooftree}

Let $p$ be the position of $x$ in $q$. Apply extraction \textsf{ExtractCon}(p) targeting the equations with lhss $q[x := con(\overline{y^{con}})]$, for all $con \in Cons_\tau$ to the original function definition for $fun$. From this we obtain the changed function definition $def'_{fun}$ for $fun$ and the auxiliary function definition. The lhss of $def'_{fun}$ are
\[
Q \cup \{ \texttt{get}(q[x := con(\overline{y^{con}})]) ~ | ~ con \in Cons_\tau \} = Q \cup \{q\},
\]
where the pair of \texttt{get} and \texttt{putback} is the lens underlying \textsf{ExtractCon}(p). This set of lhss is the same as that in $fun \lhd | ~ Q ~ (q)$ from the derivation step above. The lhss of the auxiliary function definition are
\begin{multline*}
\{ q_{\zeta_r} ~ | ~ r \in T \} = \{ \texttt{putback}(\langle \texttt{get}(q_r) \rangle^{aux}, q_r) ~ | ~ r \in T \} \\
= \{ \sigma^{q_r}_\pi(aux(\langle q \rangle^{vars})) ~ | ~ r \in T \} = \{ aux(\langle q \rangle^{vars})[x := con(\overline{y^{con}})] ~ | ~ con \in Cons_\tau \} \\
= \{ aux(x, \langle q_r \rangle^{vars})[x := con(\overline{y^{con}})] \text{ for some } r \in T ~ | ~ con \in Cons_\tau \},
\end{multline*}
where $\zeta$ is the function in the triple that is \textsf{ExtractCon}(p). It is clear that each $q_{\zeta_r}$ is simple and that $\{ q_{\zeta_r} ~ | ~ r \in T \}$ covers $aux$:
\begin{prooftree}
\AxiomC{}
\RightLabel{C\textsubscript{Head}}
\UnaryInfC{$aux \lhd | ~ aux(x, \overline{y})$}
\RightLabel{C\textsubscript{VarSplit}}
\UnaryInfC{$aux \lhd | ~ (q_{\zeta_r})_{r \in T}$}
\end{prooftree}
\end{enumerate}

As shown for each of the two cases, after the extractions copattern coverage still holds for the translated program, and we even know its derivation tree. Thus we can go back to the start and do further translations, if necessary.
\end{algorithm}

Next, we consider important properties of the translation algorithm.

\textbf{Termination.} To show that the algorithm terminates, we can basically repeat the argument of Setzer et al. Each translation results in a changed function definition $def'_{fun}$ for $fun$ and a new function definition $def_{aux}$ for the auxiliary function. The lhss of $def'_{fun}$ are $Q \cup \{q\}$, which means that they cover $fun$ in the same way $fun$ is covered via $fun \lhd | ~ Q ~ (q)$ in the derivation step above. This means that the translation reduces the depth of the derivation tree for the coverage of $fun$, thus eventually arriving at sufficiently simple patterns. The added auxiliary function definitions already have simple lhss, thus they are not translated, and eventually the algorithm terminates.

\textbf{Range.} The algorithm only stops when all lhss are sufficiently simple for $i$ and works on the coverage derivation trees, thus we have the desired range $\mathbf{U}_{cc}[fdef_{un,i}]$ if it terminates in all cases, which is shown next.

\textbf{Preservation of well-typedness.} The only changes made to the program are a series of extractions (each is either constructor or destructor extraction). All extractions preserve the well-typedness of a program, therefore well-typedness is also preserved by $\texttt{Unnest}_i$.

\textbf{Bisimulation.} Because copattern coverage holds for the translated program, the translated program has no overlapping lhss. Therefore, by the propositions 3.3.1 and 3.3.2, we know that the translation preserves the semantics of the program in the kind of weak bisimulation described in section 3.3.

\subsection{Example}

As an example application of the algorithm, consider the following program fragment.

\begin{lstlisting}

fun(x).des().des1(con1()) = t1
fun(x).des().des1(con2()) = t2
fun(x).des().des2() = t3

\end{lstlisting}

The final step in the coverage derivation for $fun$ is splitting \texttt{fun(x).des().des1(y)} into \texttt{fun(x).des().des1(con1())} and \texttt{fun(x).des().des1(con2())} by splitting variable $y$. Therefore we first target these two lhss for constructor extraction at the common position of \texttt{con1()} and \texttt{con2()} in the two lhss; the result of the extraction can be seen below.

\begin{lstlisting}

fun(x).des().des1(y) = aux(y, x)
fun(x).des().des2() = t3

aux(con1(), x) = t1
aux(con2(), x) = t2

\end{lstlisting}

For this program, the final step in the coverage derivation for $fun$ is splitting \texttt{fun(x).des()} into \texttt{fun(x).des().des1(y)} and \texttt{fun(x).des().des2()} by result splitting. Therefore we target these two lhss for destructor extraction; the result of the extraction can be seen below. Note that \texttt{fun(x).des() = aux2(x)} is generated twice but only present once in the function definition, as these are sets.

\begin{lstlisting}

fun(x).des() = aux2(x)

aux2(x).des1(y) = aux(y, x)
aux2(x).des2() = t3

\end{lstlisting}

The \texttt{aux} function definition remains unchanged and is omitted. This program is in the desired fragment; the algorithm terminates here because all lhss of the program are simple.

\section{Core de-/refunctionalization}
\label{sec:coredr}

...

\textsf{CoreDefunc} and \textsf{CoreRefunc} use -- and in fact essentially are -- the respective sides of the two-way transformation of Rendel et al. One side of this is the defunctionalization for the Codata Fragment, which we call $d^{codata}$ in here, and the other side is the refunctionalization for the Data Fragment, which we call $r^{data}$. Their definitions are repeated below for the convenience of the reader.

...

\subsection{Core defunctionalization}

The core defunctionalization \textsf{CoreDefunc} applies the defunctionalization of Rendel et al. for the Codata Fragment to the parts of the program that are in this fragment, and leaves the rest, which is already in a defunctionalized form after \textsf{Unnest}, virtually unchanged. That is, the only change to these parts affects the right-hand sides of equations, which are transformed to account for the syntactic differences between function and destructor calls.\footnote{I.e., the difference between $fun(...)$ and $t.des(...)$. This difference vanishes when changing the syntax of destructor calls to $des(t, ...)$.} The respective transformation for terms is $d$, defined below with the algorithm.

First, we define the class of function definitions of a program $prg$ which are already defunctionalized. A function definition is already defunctionalized when all of its equations have lhss without destructors. We define the class of already defunctionalized functions $F^{prg}_d$ as those which have already defunctionalized function definitions.

Using this definition, we define defunctionalization of terms.

\begin{align*}
\langle x \rangle^d = x \\
\langle s.des(t_1, ..., t_n) \rangle^d = \langle des \rangle^d (\langle s \rangle^d, \langle t_1 \rangle^d, ..., \langle t_n \rangle^d) \\
\langle fun(t_1, ..., t_n) \rangle^d = fun(\langle t_1 \rangle^d, ..., \langle t_n \rangle^d), \text{ if } fun \in F_d \\
\langle fun(t_1, ..., t_n) \rangle^d = \langle fun \rangle^d (\langle t_1 \rangle^d, ..., \langle t_n \rangle^d), \text{ if } fun \not\in F_d \\
\langle con(t_1, ..., t_n) \rangle^d = con(\langle t_1 \rangle^d, ..., \langle t_n \rangle^d)
\end{align*}

This transformation only changes symbols, but leaves the structure of the terms unchanged (disregarding the syntactic sugar for destructor calls). Especially, this means that $\langle q[\sigma] \rangle^d = \langle q \rangle^d [\langle \cdot \rangle^d \circ \sigma]$ for any copattern $q$ and substitution $\sigma$. A property that is important for the definition of \textsf{CoreDefunc} below is that, for any copattern $q$ that is allowed as a lhs in the domain of \textsf{CoreDefunc}, $\langle q \rangle^d$ is also a copattern. This can be easily shown by induction on the structure of $q$.

Next, we give the definition for \textsf{CoreDefunc}. Like Rendel et al. do for their de- and refunctionalization, we use set-builder notation.

\begin{algorithm}[\textsf{CoreDefunc}]

\begin{align*}
\langle prg \rangle^{\textsf{CoreDefunc}} & = ~&& \{ && \textrm{\textbf{data }} \sigma \textrm{\textbf{ where }} \span\span\span\span \\
& && && && \{ \langle fun \rangle^d (\tau_1, ..., \tau_n): \sigma ~ | ~ `` fun(\tau_1, ..., \tau_n): \tau " \in prg \text{ s.t. } fun \not\in F^{prg}_d \} \\
& && | && `` \textrm{\textbf{codata }} \sigma ... " \in prg \} \span\span\span\span \\
& \cup && \{ && \textrm{\textbf{function }} \langle des \rangle^d (\sigma, \tau_1, ..., \tau_n) \textrm{\textbf{ where}} \span\span\span\span \\
& && && && \{ \langle des \rangle^d (\langle fun \rangle^d (\overline{x}), \overline{y}) = \langle t \rangle^d ~ | ~ `` fun(\overline{x}).des(\overline{y}) = t " \in prg \} \\
& && | && `` \sigma.des(\tau_1, ..., \tau_n) " \in prg \} \span\span\span\span \\
& \cup && \{ && \textrm{\textbf{data }} ... ~ | ~ `` \textrm{\textbf{data }} ... " \in prg \} \span\span\span\span \\
& \cup && \{ && \textrm{\textbf{function }} fun(\sigma, \tau_1, ..., \tau_k): \tau \textrm{\textbf{ where }} \{ q = \langle t \rangle^d ~ | ~ "q = t" \in eqns \} \span\span\span\span \\
& && | && `` \textrm{\textbf{function }} fun(\sigma, \tau_1, ..., \tau_k): \tau \textrm{\textbf{ where }} eqns " \text{ s.t. } fun \in F^{prg}_d \} \span\span\span\span
\end{align*}

\end{algorithm}

We consider important properties of the transformation.

\textbf{Termination.} In the set-builder notation we used to define the algorithm, it is clear how the resulting program is constructed by going through the syntax of the input program. That is, the set-builder definitions for each of the sets in the union can be implemented as loops over specific syntactic parts of the input program, like the codata signatures, the destructor signatures, or the equations. Some of the loops are nested, but each time the collection that is looped over is finite, thus the transformation algorithm terminates.

\textbf{Range.} It can be directly seen in the definition that the resulting program contains no codata type definitions. The copattern coverage of the resulting program is also easy to see. In the set-builder definition, the last set in the union contains function definitions taken over from the original program with only the right-hand sides of their equations changed (by $\langle \cdot \rangle^d$); since the left-hand sides stay the same, these function definitions again have copattern coverage. The function definitions from the second set in the union remain to be considered. For each of their equations, each of its left-hand sides is constructed (disregarding the receiver notation for destructors) from the original left-hand side by turning the destructor name into the corresponding function name and the function name into the corresponding constructor name. For the original function definitions, copattern coverage was derived by result splitting; consequently, copattern coverage for the new function definitions can be derived by variable splitting on the first variable.

\textbf{Preservation of well-typedness.} Let $prg$ be the program to be transformed; we assume $prg$ to be well-typed. This means that for each equation $eqn = `` q = t ''$ of $prg$ we have a derivation of $\Sigma_{prg} \vdash eqn \textrm{ ok}$, with $\Sigma_{prg}$ being the signatures of $prg$. We will show that each equation of $\langle prg \rangle^{\textsf{CoreDefunc}}$ is well-typed. In the definition of \textsf{CoreDefunc}, we can see that the signatures $\Sigma_{\langle prg \rangle^{\textsf{CoreDefunc}}}$ of $\langle prg \rangle^{\textsf{CoreDefunc}}$ are exactly the following: for each destructor signature in $\Sigma_{prg}$ an identical function signature, for some function signatures in $\Sigma_{prg}$ identical constructor signatures, and for the other function signatures in $\Sigma_{prg}$ the identical function signatures. From this, we have the following lemma.

\begin{lemma}
\label{lem:pwtdefunc}
When $\Sigma_{prg} \vdash `` q = t " \textrm{ ok}$, then $\Sigma_{\langle prg \rangle^{\textsf{CoreDefunc}}} \vdash `` \langle q \rangle^d = \langle t \rangle^d " \textrm{ ok}$.
\begin{proof}
By induction on the derivation of $\Sigma \vdash `` q = t " \textrm{ ok}$, using the similarity of $\Sigma_{prg}$ and $\Sigma_{\langle prg \rangle^{\textsf{CoreDefunc}}}$.
\end{proof}
\end{lemma}

In the definition of \textsf{CoreDefunc} we can see that each equation $eqn = `` q = t "$ in $prg$ corresponds to exactly one equation $`` \langle q \rangle^d = \langle t \rangle^d "$ of $\langle prg \rangle^{\textsf{CoreDefunc}}$. Combine this with \autoref{lem:pwtdefunc} to get $\Sigma_{\langle prg \rangle^{\textsf{CoreDefunc}}} \vdash `` q = t " \textrm{ ok}$ for each equation $eqn$ of $\langle prg \rangle^{\textsf{CoreDefunc}}$. Thus $\langle prg \rangle^{\textsf{CoreDefunc}}$ is well-typed whenever $prg$ is.

\textbf{Bisimulation.} For \textsf{CoreDefunc}, strong bisimulation holds. To show this, we first show a general result that an isomorphism between rule sets of two programs is also an isomorphism between the respective reduction relations of the programs. In the following, $\langle \cdot \rangle$ is some program transformation when applied to programs. When applied to terms it stands for some transformation of terms that only changes symbols, but leaves the structure of the terms unchanged.

\begin{lemma}[Rules isomorphism is reduction isomorphism]
\label{lem:iso}
For any program $prg$, when it is
\[
\forall s,t \in \textrm{Term}_{prg}. (s, t) \in \textrm{Rules}(prg) \iff (\langle s \rangle, \langle t \rangle) \in \textrm{Rules}(prg).
\]
, then it also holds that
\[
\forall s,t \in \textrm{Term}_{prg}. s \longrightarrow_{prg} t \iff \langle s \rangle \longrightarrow_{\langle prg \rangle} \langle t \rangle.
\]
\end{lemma}

In order to prove this, we first show some auxiliary lemmas concerning the value judgement and evaluation contexts.

\begin{lemma}[Isomorphic value judgements]
For any program $prg$, when it is
\[
\forall s,t \in \textrm{Term}_{prg}. (s, t) \in \textrm{Rules}(prg) \iff (\langle s \rangle, \langle t \rangle) \in \textrm{Rules}(prg).
\]
, then it also holds that
\label{lem:ivj}
\[
\forall t \in \textrm{Term}_{prg}. \vdash_{v_{prg}} t \iff \vdash_{v_{\langle prg \rangle}} \langle t \rangle
\]
\begin{proof}
$``\Rightarrow"$: By induction on the structure of $t$. By the derivation of $\vdash_{v_{prg}} t$ it is (a) $\vdash_{v_{prg}} t_i$ for all immediate subterms $t_i$ of $t$ and (b) $t \neq^? q$ for all lhss $q$ of rules of $prg$. By the premiss we know that for all lhss $q'$ of rules of $\langle prg \rangle$, there is a lhs $q$ of $prg$ such that $q' = \langle q \rangle$; thus it is $\langle t \rangle \neq^? q'$ for all lhss $q'$ of rules of $\langle prg \rangle$. By the induction hypothesis, we know that $\vdash_{v_{\langle prg \rangle}} \langle t_i \rangle$. Since the immediate subterms of $\langle t \rangle^d$ are the $\langle t_i \rangle$ for each $t_i$, we have the derivation for $\vdash_{v_{\langle prg \rangle}} \langle t \rangle$.

$``\Leftarrow"$: By induction on the structure of $t$. By the derivation of $\vdash_{v_{\langle prg \rangle}} t$ it is (a) $\vdash_{v_{\langle prg \rangle}} t'_i$ for all immediate subterms $t'_i$ of $\langle t \rangle$ and (b) $\langle t \rangle \neq^? q'$ for all lhss $q'$ of rules of $\langle prg \rangle$. By the premiss we know that for all lhss $q$ of rules of $prg$, there is a lhs $\langle q \rangle$ of $\langle prg \rangle$; thus it is $t \neq^? q$ for all lhss $q$ of rules of $prg$. By the induction hypothesis, since it is $\langle t_i \rangle = t'_i$ for each immediate subterm $t_i$ of $t$, we know that $\vdash_{v_{prg}} t_i$. All in all, we have the derivation for $\vdash_{v_{prg}} t$.
\end{proof}
\end{lemma}

\begin{lemma}[Isomorphic evaluation contexts]
\label{lem:iec}
For any program $prg$, when it is
\[
\forall s,t \in \textrm{Term}_{prg}. (s, t) \in \textrm{Rules}(prg) \iff (\langle s \rangle, \langle t \rangle) \in \textrm{Rules}(prg).
\]
, then it also holds that
\[
\forall \mathcal{E}. \mathcal{E} \in \textrm{EC}_{prg} \iff \langle \mathcal{E} \rangle \in \textrm{EC}_{\langle prg \rangle}
\]
\begin{proof}
$``\Rightarrow"$: By induction on the structure of $\mathcal{E}$.

When $\mathcal{E} = []$, then it is $\langle \mathcal{E} \rangle = []$, and this is clearly an evaluation context for $\langle prg \rangle$.

When $\mathcal{E} \neq []$, all immediate subterms $t_{left}$ of $\mathcal{E}$ to the left of the hole are values under judgement $prg \vdash_v$. The ``middle'' term that follows after all these is again an evaluation context $\mathcal{E}_0$ for $prg$. $\langle \mathcal{E} \rangle$ is $\mathcal{E}$ with each $t_{left}$ replaced by $\langle t_{left} \rangle$, the immediate subterms right of the hole replaced in the same way, and the ``middle'' term replaced by $\langle \mathcal{E}_0 \rangle$. By the induction hypothesis, $\langle \mathcal{E}_0 \rangle$ is an evaluation context for $\langle prg \rangle$ because $\mathcal{E}_0$ is one for $prg$. By \autoref{lem:ivj}, it is $\langle prg \rangle \vdash_v \langle t_{left} \rangle$ for all $t_{left}$ because $prg \vdash_v t_{left}$.

$``\Leftarrow"$: Analogously to $``\Rightarrow"$.
\end{proof}
\end{lemma}

Next, we show that \autoref{lem:iso} holds restricted to contraction, from which the full \autoref{lem:iso} directly follows.

\begin{lemma}[Rules isomorphism is contraction isomorphism]
\label{lem:isoc}
For any program $prg$, when it is
\[
\forall s,t \in \textrm{Term}_{prg}. (s, t) \in \textrm{Rules}(prg) \iff (\langle s \rangle, \langle t \rangle) \in \textrm{Rules}(prg).
\]
, then it also holds that
\[
\forall s,t \in \textrm{Term}_{prg}. s \mapsto_{prg} t \iff \langle s \rangle \mapsto_{\langle prg \rangle} \langle t \rangle.
\]
\begin{proof}
$``\Rightarrow"$: By the derivation of $s \mapsto_{prg} t$ there is a rule $r := (q_r, t_r) \in \textrm{Rules}(prg)$ and a substitution $\sigma$ such that $s = q_r[\sigma]$ and $t = t_r[\sigma]$ and the right-hand sides of $\sigma$ are values in $prg$. By the premiss we know that there is a rule $\langle r \rangle := (\langle q_r \rangle, \langle t_r \rangle) \in \textrm{Rules}(prg)$, and by \autoref{lem:ivj} we know that, for each term $v$ that is a value in $prg$, $\langle v \rangle$ is a value in $\langle prg \rangle$ and thus the right-hand sides of $\sigma \circ \langle \cdot \rangle$ are values in $\langle prg \rangle$. Consequently, it is $\langle s \rangle = \langle q_r \rangle[\langle \cdot \rangle \circ \sigma] \mapsto \langle t_r \rangle[\langle \cdot \rangle \circ \sigma] = \langle t \rangle$.

$``\Leftarrow"$: By the derivation of $\langle s \rangle \mapsto_{prg} \langle t \rangle$ there is a rule $r' := (q_{r'}, t_{r'}) \in \textrm{Rules}(\langle prg \rangle)$ and a substitution $\sigma'$ such that $\langle s \rangle = q_{r'}[\sigma']$ and $\langle t \rangle = t_{r'}[\sigma']$ and the right-hand sides of $\sigma'$ are values in $prg$. By the premiss we know that there is a rule $r  := (q_r, t_r) \in \textrm{Rules}(prg)$ such that $\langle q_r \rangle = q_{r'}, \langle t_r \rangle = t_{r'}$, and by \autoref{lem:ivj} we know that, for each term $v'$ that is a value in $\langle prg \rangle$, each term $v$ with $v' = \langle v \rangle$ is a value in $prg$ and thus the right-hand sides of $\sigma$ with $\sigma' = \sigma \circ \langle \cdot \rangle$ are values in $prg$. Consequently, it is $s = q_r[\sigma] \mapsto t_r[\sigma] = t$.
\end{proof}
\end{lemma}

\begin{proof}[Proof of \autoref{lem:iso}]
$``\Rightarrow"$: By the derivation of $s \longrightarrow_{prg} t$ there are terms $s', t'$ and an evaluation context $\mathcal{E}$ for $prg$ such that $s = \mathcal{E}[s'], t = \mathcal{E}[t']$ and $s' \mapsto_{prg} t'$. By \autoref{lem:iec} and \autoref{lem:isoc} we know that $\langle \mathcal{E} \rangle$ is an evaluation context for $\langle prg \rangle$ and that $\langle s' \rangle \mapsto_{\langle prg \rangle} \langle t' \rangle^d$, respectively. Consequently, it is $\langle s \rangle = \langle \mathcal{E} \rangle [\langle s' \rangle] \longrightarrow_{\langle prg \rangle} \langle \mathcal{E} \rangle [\langle t' \rangle] = \langle t \rangle$.

$``\Leftarrow"$: By the derivation of $\langle s \rangle \longrightarrow_{\langle prg \rangle} \langle t \rangle$ there are terms $\langle s' \rangle, \langle t' \rangle$ and an evaluation context $\langle \mathcal{E} \rangle$ for $prg$ such that $\langle s \rangle = \langle \mathcal{E} \rangle[\langle s' \rangle], \langle t \rangle = \langle \mathcal{E} \rangle [\langle t' \rangle]$ and $\langle s' \rangle \mapsto_{\langle prg \rangle} \langle t' \rangle$. By \autoref{lem:iec} and \autoref{lem:isoc} we know that $\mathcal{E}$ is an evaluation context for $prg$ and that $s' \mapsto_{prg} t'$, respectively. Consequently, it is $s = \mathcal{E}[s'] \longrightarrow_{prg} \mathcal{E}[t'] = t$.
\end{proof}

Using \autoref{lem:iso}, we finally prove strong bisimulation for \textsf{CoreDefunc}, by showing that the premiss of the lemma holds.

\begin{lemma}[Strong bisimulation of the rule relations]
\label{lem:sbruld}
\[
\forall s,t \in \textrm{Term}_{prg}. (s, t) \in \textrm{Rules}(prg) \iff (\langle s \rangle^d, \langle t \rangle^d) \in \textrm{Rules}(\langle prg \rangle^{\textsf{CoreDefunc}}).
\]
\begin{proof}
By inspecting the definition of \textsf{CoreDefunc}, it can be easily seen that for each rule $(s, t) \in \textrm{Rules}(prg)$, there is a rule $(\langle s \rangle^d, \langle t \rangle^d) \in \textrm{Rules}(\langle prg \rangle^{\textsf{CoreDefunc}})$, and that there are no other rules in $\textrm{Rules}(\langle prg \rangle^{\textsf{CoreDefunc}})$.
\end{proof}
\end{lemma}

\begin{corollary}[Strong bisimulation of \textsf{CoreDefunc}]
\[
\forall s,t \in \textrm{Term}_{prg}. s \longrightarrow_{prg} t \iff \langle s \rangle^d \longrightarrow_{\langle prg \rangle^{\textsf{CoreDefunc}}} \langle t \rangle^d.
\]
\begin{proof}
By combining \autoref{lem:sbruld} and \autoref{lem:iso}.
\end{proof}
\end{corollary}

\subsection{Core refunctionalization}

Core refunctionalization is defined analogously to core defunctionalization, with the exception of needing one more preprocessing, described below. The actual algorithm applies the refunctionalization of Rendel et al. for the Data Fragment to the parts of the program that are in this fragment, and leaves the rest, which is already in a refunctionalized form after \textsf{Unnest}, virtually unchanged. Unfortunately, and unlike defunctionalization, this algorithm requires some kind of preprocessing, since the actual refunctionalization requires all constructors to be the first pattern of an lhs. We first describe this asymmetry in more detail and illustrate the underlying problem, then consider a possible workaround for our situation, and finally give the actual refunctionalization and, as with defunctionalization, consider important properties of it.

\subsubsection{An asymmetry: The problem at hand}
\label{sssec:asym}

In the introduction to this chapter, we gave two fragments of Uroboro induced by unnested function definition fragments, as the domain of defunctionalization and refunctionalization, respectively. The first fragment, for defunctionalization, has the following syntax for function definitions.

\begin{align*}
fdef_{un,d} &::= eqn_{nr}^* ~ | ~ eqn_r^* \\
eqn_r &::= fun(x^*).des(y^*) = t \\
eqn_{nr} &::= fun(p^*) = t \\
\end{align*}

It is unnested regarding destructors, and allows arbitrary patterns when no destructors are present. The second fragment, for refunctionalization, is somewhat analogous to the first, in that it is unnested regarding constructors, and allows arbitrarily many destructors when no constructors are present.

\begin{align*}
fdef_{un,r} &::= eqn_{nr}^* ~ | ~ eqn_r^* \\
eqn_r &::= fun(x^*).des(y^*)^* = t \\
eqn_{nr} &::= fun(x^*, con(y^*), z^*) = t \\
\end{align*}

However, there is an asymmetry between the two fragments. When only one destructor is allowed, this can also only be placed at one position. On the other hand, allowing only one constructor leaves open the possibility of placing this at different positions. More on this later.

The refunctionalization algorithm given below requires the constructor to be at the first position, i.e., it has as its domain the fragment $\mathbf{U}_{cc}[fdef_{un,r'}]$ of Uroboro defined as follows.

\begin{definition}[Unnested Fragment for the actual \textsf{CoreRefunc}]
The fragment $\mathbf{U}_{cc}[fdef_{un,r'}]$, for the actual \textsf{CoreRefunc}, is induced by the function definition fragment $fdef_{un,r'}$, defined below using EBNF rules.
\begin{align*}
fdef_{un,r'} &::= eqn_{nr'}^* ~ | ~ eqn_r^* \\
eqn_r &::= fun(x^*).des(y^*)^* = t \\
eqn_{nr'} &::= fun(con(x^*), y^*) = t \\
\end{align*}
\end{definition}

The two fragments $\mathbf{U}_{cc}[fdef_{un,d}]$ and $\mathbf{U}_{cc}[fdef_{un,r'}]$ are the Data and Codata fragment from Rendel et al., extended to allow arbitrary lhss, as long as they are already defunctionalized or refunctionalized, i.e., they contain no destructors or no constructors, respectively. This allows us to extend their de- and refunctionalization transformations for the Codata and Data Fragments, to our core de- and refunctionalization for$\mathbf{U}_{cc}[fdef_{un,r'}]$ and $\mathbf{U}_{cc}[fdef_{un,d}]$, respectively. The Data and Codata Fragment are symmetric to each other, as de- and refunctionalization can switch between them; as Rendel et al. describe it, for these fragments, de- and refunctionalization is the same transformation, namely a matrix transposition for a two-dimensional matrix representation of programs.

All this would be good and well, were it not for the fact that unnesting to prepare for refunctionalization doesn't arrive at $\mathbf{U}_{cc}[fdef_{un,r'}]$. A somewhat ad-hoc workaround is presented in the section following the next. More importantly, however, this situation shines a light on the asymmetry between destructors and constructors -- in Uroboro and elsewhere, as this difference is not one exclusive to Uroboro. This underlying problem is illustrated in the next section.

\subsubsection{An asymmetry: The general problem}

The general problem is that destructors allow one to ``observe'' the result in only one way at a time, i.e., the next destructor in the stack already observes the result of the previous, while constructors allow one to observe input in different ways. As an example, consider the function $fun$ with two arguments $x, y$.
\[
fun(x, y) = ?
\]
In order to refine the definition of $fun$, one might choose to observe its output by destructing it. Suppose there are two destructors that fit the type of $fun$, a coverage complete definition thus would be:
\begin{align*}
& fun(x, y).des1 = x \\
& fun(x, y).des2 = y
\end{align*}
Instead of stopping here, this might be refined further, but, when starting refinement with output observation, the refinement will always invariantly start with this step as long as coverage completeness is desired. On the other hand, one might choose to observe the input of $fun$ by replacing a variable with the constructor calls that fit its type.
\begin{align*}
& fun(con1(), y) = con1() \\
& fun(con2(), y) = y
\end{align*}
Here $x$ was chosen to be observed. But, unlike with outputs, this isn't the only way to start the refinement, since we could have instead chosen to observe $y$.
\begin{align*}
& fun(x, con()) = con() \\
\end{align*}

The problem is directly related to the difference between natural deduction and arbitrary sequent calculus. In natural deduction, the left-hand side sequence consists of many elements, while the right-hand side sequence consists of only one element. Similarly, the many sets of constructors for each type of the variables allow many ``dimensions'' of the input to be observed, while the set of destructors for the codata type of a function only allows one ``dimension'' of the output to be observed.

\subsubsection{A workaround}

...

\subsubsection{The actual algorithm}

Now, we turn to the actual algorithm, with domain $\mathbf{U}_{cc}[fdef_{un,r'}]$.

First, we define the class of function definitions of a program $prg$ which are already refunctionalized. A function definition is already refunctionalized when (a) all of its equations have destructor patterns, or (b) the function has no arguments, or (c) the first argument of every lhs is a variable. We define the class of already refunctionalized functions $F^{prg}_r$ as those which have already defunctionalized function definitions.

Using this definition, we define refunctionalization of terms.

\begin{align*}
\langle x \rangle^r = x \\
\langle s.des(t_1, ..., t_n) \rangle^r = \langle s \rangle^r .des(\langle t_1 \rangle^r, ..., \langle t_n \rangle^r) \\
\langle fun(t_1, ..., t_n) \rangle^r = fun(\langle t_1 \rangle^r, ..., \langle t_n \rangle^r), \text{ if } fun \in F_r \\
\langle fun(t_1, ..., t_n) \rangle^r = \langle t_1 \rangle^r .\langle fun \rangle^r (\langle t_2 \rangle^r, ..., \langle t_n \rangle^r), \text{ if } fun \not\in F_r \\
\langle con(t_1, ..., t_n) \rangle^r = \langle con \rangle^r (\langle t_1 \rangle^r, ..., \langle t_n \rangle^r)
\end{align*}

Next, we give the definition for \textsf{CoreRefunc}.

\begin{algorithm}[\textsf{CoreRefunc}]

\begin{align*}
\langle prg \rangle^{\textsf{CoreDefunc}} & = ~&& \{ && \textrm{\textbf{codata }} \sigma \textrm{\textbf{ where }} \span\span\span\span \\
& && && && \{ \sigma.\langle fun \rangle^d (\tau_1, ..., \tau_n): \tau ~ | ~ `` fun(\sigma, \tau_1, ..., \tau_n): \tau " \in prg \text{ s.t. } fun \not\in F^{prg}_d \} \\
& && | && `` \textrm{\textbf{data }} \sigma ... " \in prg \} \span\span\span\span \\
& \cup && \{ && \textrm{\textbf{function }} \langle con \rangle^r (\tau_1, ..., \tau_n): \tau \textrm{\textbf{ where}} \span\span\span\span \\
& && && && \{ \langle con \rangle^r (\langle fun \rangle^d (\overline{x}), \overline{y}) = \langle t \rangle^d ~ | ~ `` fun(con(\overline{x}), \overline{y}) = t " \in prg \} \\
& && | && `` con(\tau_1, ..., \tau_n): \tau " \in prg \} \span\span\span\span \\
& \cup && \{ && \textrm{\textbf{codata }} ... ~ | ~ `` \textrm{\textbf{codata }} ... " \in prg \} \span\span\span\span \\
& \cup && \{ && \textrm{\textbf{function }} fun(\tau_1, ..., \tau_k): \tau \textrm{\textbf{ where }} \{ q = \langle t \rangle^r ~ | ~ "q = t" \in eqns \} \span\span\span\span \\
& && | && `` \textrm{\textbf{function }} fun(\tau_1, ..., \tau_k): \tau \textrm{\textbf{ where }} eqns " \text{ s.t. } fun \in F^{prg}_r \} \span\span\span\span
\end{align*}

\end{algorithm}

We consider important properties of the transformation. Termination and preservation of well-typedness can be shown completely analogously to \textsf{CoreDefunc}; simply use \textsf{CoreRefunc} and $\langle \cdot \rangle^r$ instead of \textsf{CoreDefunc} and $\langle \cdot \rangle^d$, respectively.

\textbf{Range.} This is also shown completely analogously to \textsf{CoreDefunc}, again using the respective dual notions. It can be directly seen in the definition that the resulting program contains no data type definitions. The copattern coverage of the resulting program is also easy to see. In the set-builder definition, the last set in the union contains function definitions taken over from the original program with only the right-hand sides of their equations changed (by $\langle \cdot \rangle^r$); since the left-hand sides stay the same, these function definitions again have copattern coverage. The function definitions from the second set in the union remain to be considered. For each of their equations, each of its left-hand sides is constructed (disregarding the receiver notation for destructors) from the original left-hand side by turning the function name into the corresponding destructor name and the constructor name into the corresponding function name. For the original function definitions, copattern coverage was derived by variable splitting on the first variable; consequently, copattern coverage for the new function definitions can be derived by result splitting.

\textbf{Bisimulation.} For \textsf{CoreRefunc}, strong bisimulation holds. Again, this is shown completely analogously to \textsf{CoreDefunc}; we reuse \autoref{lem:iso}.

\begin{lemma}[Strong bisimulation of the rule relations]
\label{lem:sbrulr}
\[
\forall s,t \in \textrm{Term}_{prg}. (s, t) \in \textrm{Rules}(prg) \iff (\langle s \rangle^r, \langle t \rangle^r) \in \textrm{Rules}(\langle prg \rangle^{\textsf{CoreRefunc}}).
\]
\begin{proof}
By inspecting the definition of \textsf{CoreRefunc}, it can be easily seen that for each rule $(s, t) \in \textrm{Rules}(prg)$, there is a rule $(\langle s \rangle^r, \langle t \rangle^r) \in \textrm{Rules}(\langle prg \rangle^{\textsf{CoreRefunc}})$, and that there are no other rules in $\textrm{Rules}(\langle prg \rangle^{\textsf{CoreRefunc}})$.
\end{proof}
\end{lemma}

\begin{corollary}[Strong bisimulation of \textsf{CoreRefunc}]
\[
\forall s,t \in \textrm{Term}_{prg}. s \longrightarrow_{prg} t \iff \langle s \rangle^r \longrightarrow_{\langle prg \rangle^{\textsf{CoreRefunc}}} \langle t \rangle^r.
\]
\begin{proof}
By combining \autoref{lem:sbrulr} and \autoref{lem:iso}.
\end{proof}
\end{corollary}

\subsection{Example}

We continue with the example fragment of section 4.1.3, augmented with (co)data definitions and signatures.

\begin{lstlisting}

data P where
  con1(): P
  con2(): P

codata N1 where
  N1.des(): N2

codata N2 where
  N2.des1(P): P
  N2.des2(): P

function fun(N1): N1 where
  fun(x).des() = aux2(x)

function aux(P, N1): P where
  aux(con1(), x) = t1
  aux(con2(), x) = t2

function aux2(N1): N2 where
  aux2(x).des1(y) = aux(y, x)
  aux2(x).des2() = t3

\end{lstlisting}

Using the \textsf{CoreDefunc} algorithm on this yields the following program, by leaving the function definition for \texttt{aux} and the data definition unchanged, and applying the defunctionalization of Rendel et al. to the other definitions.

\begin{lstlisting}

data P where
  con1(): P
  con2(): P

data N1 where
  fun(N1): N1

data N2 where
  aux2(N1): N2

function aux(P, N1): P where
  aux(con1(), x) = t1
  aux(con2(), x) = t2

function des(N1): N2 where
  des(fun(x)) = aux2(x)

function des1(N2, P): P where
  des1(aux2(x), y) = aux(y, x)

function des2(N2): P where
  des2(aux2(x)) = t3

\end{lstlisting}

Using the \textsf{CoreRefunc} algorithm on the original program yields the following program, by leaving the function definitions for \texttt{fun} and \texttt{aux2} and the codata definitions unchanged, and applying the refunctionalization of Rendel et al. to the other definitions.

\begin{lstlisting}

codata P where
  P.aux(N1): P

codata N1 where
  N1.des(): N2

codata N2 where
  N2.des1(P): P
  N2.des2(): P

function fun(N1): N1 where
  fun(x).des() = aux2(x)

function aux2(N1): N2 where
  aux2(x).des1(y) = aux(y, x)
  aux2(x).des2() = t3

function con1() where
  con1().aux(x) = t1

function con2() where
  con2().aux(x) = t2

\end{lstlisting}

%%-- end under construction