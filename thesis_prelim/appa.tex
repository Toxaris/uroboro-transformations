\chapter{Proofs}

\cdpaux*
\begin{proof}
There is an evaluation context $\mathcal{E}'$, with underlying value judgement $\vdash'_v$, and terms $a^0, b^0$, such that $a = \mathcal{E}'[a^0]$, $b = \mathcal{E}'[b^0]$, and $a^0 \longrightarrow' b^0$. Now the proof proceeds by induction on the structure of $\mathcal{E}'$.

For the hole context $\mathcal{E}' = []$, it is $a^0 = a$ and $b^0 = b$. Choose $d = b$. since we know that $a$ reduces by $\epsilon$ because $a \longrightarrow^{aux} c$, it follows that $b = c$ and thus the desired property holds.

Next, consider contexts of the form $\mathcal{E}' = fun(\overline{v}, \mathcal{E}'^0, \overline{t})$, where $\mathcal{E}'^0$ is an evaluation context with respect to $\vdash'_v$, the $v_i$ are terms with $\vdash'_v v_i$, and the $t_i$ are arbitrary terms. It can only be the case that $\langle prg \rangle \not\vdash_v v_i$ for one of the $v_i$ when this $v_i$ reduces by $\epsilon$. Let $v_k$ be the first $v_i$ (with the lowest index) for which this is the case, that is, it is $v_k = \mathcal{E}''[q_\epsilon[\sigma]] \longrightarrow^{aux} \mathcal{E}''[t_\epsilon[\sigma]] =: v'_k$. We know that $c$ is $a$ with $v_k$ replaced by $v'_k$, and that $v_k$ appears in $b$ at its original position (of $a$). Thus choose $d$ as $b$ with $v_k$ replaced by $v'_k$. The argument is analogous for the two remaining forms of $\mathcal{E}'$ (constructor calls and destructor calls on values).

\end{proof}

\correctb*
\begin{proof}
By induction on the length $n$ of the sequence.

For $n = 0$, the translation doesn't change anything, thus correctness trivially holds.

Now, suppose $n = n' + 1$ such that correctness of the translation holds for all sequences of length $n'$ (the induction hypothesis). This means that
\[
s^{uc}_1 \longrightarrow_{prg} ... \longrightarrow_{prg} s^{uc}_{m'} = \langle s_{n-1} \rangle^{aux^{-1}}
\]
for the sequence $(s^{uc}_1, ..., s^{uc}_m) := uc(s_1, ..., s_{n-1})$. Two cases will be distinguished, according to the definition of $uc$:
\begin{itemize}
\item \underline{Case 1:} $s_{n-1} \neq s_n \land s_{n-1} \sim s_n$.
Then it is $uc((s_1, ..., s_n)) = uc((s_1, ..., s_{n-1})) = (s^{uc}_1, ..., s^{uc}_m)$. By the above, all steps in the sequence are indeed reduction steps of the relevant reduction. Because $s_{n-1} \sim s_n$, it is $\langle s_n \rangle^{aux^{-1}} = \langle s_{n-1} \rangle^{aux^{-1}} = \langle s^{uc}_n \rangle^{aux^{-1}}$.

\item \underline{Case 2:} otherwise.
Then it is $uc((s_1, ..., s_n)) = uc((s_1, ..., s_{n-1})) \cdot (\langle s_n \rangle^{aux^{-1}})$. By the above, all steps before the last are indeed reduction steps of the relevant reduction. The final term is by definition $\langle s_n \rangle^{aux^{-1}}$.

It remains to be shown why the final step is a reduction step of the relevant reduction. By the induction hypothesis, this step is $(\langle s_{n-1} \rangle^{aux^{-1}}, \langle s_n \rangle^{aux^{-1}})$. By assumption, it is $s_{n-1} \longrightarrow_{\langle prg \rangle} s_n$, and $\langle s_{n-1} \rangle^{aux^{-1}} \longrightarrow_{prg} \langle s_n \rangle^{aux^{-1}}$ follows from the more general $a \longrightarrow_{\langle prg \rangle} b \Rightarrow \langle a \rangle^{aux^{-1}} \longrightarrow_{\langle prg \rangle} \langle b \rangle^{aux^{-1}}$. This can be shown by induction on the derivation of the left-hand side reduction, by using
\[
\langle prg \rangle \vdash_v t \Rightarrow prg \vdash_v \langle t \rangle^{aux^{-1}},
\]
for all terms $t$, and the compatibility of evaluation contexts and $aux^{-1}$, i.e., $\langle \mathcal{E}[t] \rangle^{aux^{-1}} = \langle \mathcal{E} \rangle^{aux^{-1}}[\langle t \rangle^{aux^{-1}}]$.

\end{itemize}
\end{proof}

\clearpage
\newpage
