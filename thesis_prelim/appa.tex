\chapter{Proofs}

\begin{lemma}
Let $a$ be a term and $a^0$ be a subterm of $a$. When
\[
a \longrightarrow' b \land a^0 \longrightarrow^{aux} b^0
\]
and both reduction steps are derived by the ``Subst'' rule, then there is a $d$ such that
\[
a[a^0 \mapsto b^0] \longrightarrow' d \land b \longrightarrow^{aux} d.
\]

\begin{proof}
Since $a^0 = q_{r'}[\sigma']$ for some equation $r'$, we know that $a^0$ is a function or destructor call. By Lemma ???, because $a = q_r[\sigma], b = t_r[\sigma]$ for an equation $r$ and $a^0$ is a subterm of $a$, it follows that $a^0$ is a subterm of a right-hand side of $\sigma$. Let $\sigma_{mod}$ be $\sigma$ with this subterm of this right-hand side replaced by $b^0$, and choose $d = t_r[\sigma_{mod}]$.
\end{proof}
\end{lemma}

\begin{lemma}
Let $a$ be a term and $a^0$ be a subterm of $a$. When
\[
a \longrightarrow^{aux} b \land a^0 \longrightarrow' b^0
\]
and both reduction steps are derived by the ``Subst'' rule, then there is a $d$ such that
\[
a[a^0 \mapsto b^0] \longrightarrow^{aux} d \land b \longrightarrow' d.
\]

\begin{proof}
\end{proof}
\end{lemma}

\cdpaux*
\begin{proof}
By induction on the derivation of $a \longrightarrow' b$.

\begin{enumerate}
\item \textbf{``Subst'' case}: Two cases for the derivation of $a \longrightarrow^{aux} c$ will be distinguished. In both cases, it is $a = q_r[\sigma], b = t_r[\sigma]$ for some equation $r$.

\begin{enumerate}
\item \textbf{``Subst'' case}: It is $a = q_{r'}[\sigma'], c = t_{r'}[\sigma']$ for some equation $r'$; it is $r' = r$ because $a$ matches both $q_r$ and $q_{r'}$ and there are no overlapping lhss. It follows that $b = c$, thus we can simply choose $d = b = c$.

\item \textbf{``Cong'' case}: It is $a = \mathcal{E}[a^0], c = \mathcal{E}[c^0]$ for an $\mathcal{E} \in \textrm{EC}_{\vdash_v}$ and terms $a^0, c^0$ with $a^0 \longrightarrow^{aux} c^0$ and $a^0 = q_{r'}[\sigma'], c^0 = t_{r'}[\sigma']$ for an equation $r'$. Thus we have the desired property by Lemma A.0.1.
\end{enumerate}

\item \textbf{``Cong'' case}: Two cases for the derivation of $a \longrightarrow^{aux} c$ will be distinguished. In both cases, it is $a = \mathcal{E}[a^0], b = \mathcal{E}[b^0]$ for an $\mathcal{E} \in \textrm{EC}_{\vdash'_v}$ and terms $a^0, b^0$ with $a^0 \longrightarrow' b^0$ and $a^0 = q_r[\sigma], c^0 = t_r[\sigma]$ for an equation $r$.

\begin{enumerate}
\item \textbf{``Subst'' case}: It is $a = q_{r'}[\sigma'], c = t_{r'}[\sigma']$ for some equation $r'$. Thus we have the desired property by Lemma A.0.2.

\item \textbf{``Cong'' case}: It is $a = \mathcal{E}'[a'^0], c = \mathcal{E}'[c^0]$ for an $\mathcal{E}' \in \textrm{EC}_{\vdash_v}$ and terms $a'^0, c^0$ with $a'^0 \longrightarrow^{aux} c^0$ and $a'^0 = q_{r'}[\sigma'], c^0 = t_{r'}[\sigma']$ for some equation $r'$. We distinguish three cases:
\begin{itemize}
\item $a'^0$ is a subterm of $a^0$. By Lemma A.0.1, we have a $d^0$ with $a^0[a'^0 \mapsto c^0] \longrightarrow' d^0$ and $b^0 \longrightarrow^{aux} d^0$. Choose $d = \mathcal{E}[d^0]$. The property holds since (a) $b = \mathcal{E}[b^0] \longrightarrow^{aux} \mathcal{E}[d^0] = d$ and (b) $c = \mathcal{E}'[c^0] = \mathcal{E}[a^0[a'^0 \mapsto c^0]] \longrightarrow' \mathcal{E}[d^0] = d$. The reduction (a) is possible because $\mathcal{E} \in \textrm{EC}_{\vdash_v}$. This follows from Lemma ??? because the hole of $\mathcal{E}$ subsumes that of $\mathcal{E}'  \in \textrm{EC}_{\vdash_v}$ since $\mathcal{E}'[a'^0] = \mathcal{E}[a^0]$ and $a'^0$ is a subterm of $a^0$.

\item $a^0$ is a subterm of $a'^0$. By Lemma A.0.2, we have a $d^0$ with $a'^0[a^0 \mapsto b^0] \longrightarrow' d^0$ and $c^0 \longrightarrow^{aux} d^0$. Choose $d = \mathcal{E}'[d^0]$. That the property holds can now be shown analogously to the previous case.

\item Neither $a'^0$ is a subterm of $a^0$, nor the other way around. Then choose $d$ as $a$ with $a'^0$ replaced by $c^0$ and $a^0$ replaced by $b^0$. The property holds because $b = \mathcal{E}[b^0] = \mathcal{E}'[a^0 \mapsto b^0][a'^0] \longrightarrow^{aux} \mathcal{E}'[a^0 \mapsto b^0][c^0] = d$ and $c = \mathcal{E}'[c^0] = \mathcal{E}[a'^0 \mapsto c^0][a^0] \longrightarrow' \mathcal{E}[a'^0 \mapsto c^0][b^0] = d$. These reductions are possible because, by Lemma ???, $\mathcal{E}'[a^0 \mapsto b^0] \in \textrm{EC}_{\vdash_v}$ since $\mathcal{E}' \in \textrm{EC}_{\vdash_v}$ and $a^0 \longrightarrow' b^0$, and, by Lemma ???, $\mathcal{E}[a'^0 \mapsto c^0] \in \textrm{EC}_{\vdash'_v}$ since $\mathcal{E} \in \textrm{EC}_{\vdash'_v}$ and $a'^0 \longrightarrow^{aux} c^0$.
\end{itemize}

\end{enumerate}
\end{enumerate}

\end{proof}

\correctb*
\begin{proof}
By induction on the length $n$ of the sequence.

For $n = 0$, the translation doesn't change anything, thus correctness trivially holds.

Now, suppose $n = n' + 1$ such that correctness of the translation holds for all sequences of length $n'$ (the induction hypothesis). This means that
\[
s^{uc}_1 \longrightarrow_{prg} ... \longrightarrow_{prg} s^{uc}_{m'} = \langle s_{n-1} \rangle^{aux^{-1}}
\]
for the sequence $(s^{uc}_1, ..., s^{uc}_m) := uc(s_1, ..., s_{n-1})$. Two cases will be distinguished, according to the definition of $uc$:
\begin{itemize}
\item \underline{Case 1:} $s_{n-1} \neq s_n \land s_{n-1} \sim s_n$.
Then it is $uc((s_1, ..., s_n)) = uc((s_1, ..., s_{n-1})) = (s^{uc}_1, ..., s^{uc}_m)$. By the above, all steps in the sequence are indeed reduction steps of the relevant reduction. Because $s_{n-1} \sim s_n$, it is $\langle s_n \rangle^{aux^{-1}} = \langle s_{n-1} \rangle^{aux^{-1}} = \langle s^{uc}_n \rangle^{aux^{-1}}$.

\item \underline{Case 2:} otherwise.
Then it is $uc((s_1, ..., s_n)) = uc((s_1, ..., s_{n-1})) \cdot (\langle s_n \rangle^{aux^{-1}})$. By the above, all steps before the last are indeed reduction steps of the relevant reduction. The final term is by definition $\langle s_n \rangle^{aux^{-1}}$.

It remains to be shown why the final step is a reduction step of the relevant reduction. By the induction hypothesis, this step is $(\langle s_{n-1} \rangle^{aux^{-1}}, \langle s_n \rangle^{aux^{-1}})$. By assumption, it is $s_{n-1} \longrightarrow_{\langle prg \rangle} s_n$, and $\langle s_{n-1} \rangle^{aux^{-1}} \longrightarrow_{prg} \langle s_n \rangle^{aux^{-1}}$ follows from the more general $a \longrightarrow_{\langle prg \rangle} b \Rightarrow \langle a \rangle^{aux^{-1}} \longrightarrow_{\langle prg \rangle} \langle b \rangle^{aux^{-1}}$. This can be shown by induction on the derivation of the left-hand side reduction, by using
\[
\langle prg \rangle \vdash_v t \Rightarrow prg \vdash_v \langle t \rangle^{aux^{-1}},
\]
for all terms $t$, and the compatibility of evaluation contexts and $aux^{-1}$, i.e., $\langle \mathcal{E}[t] \rangle^{aux^{-1}} = \langle \mathcal{E} \rangle^{aux^{-1}}[\langle t \rangle^{aux^{-1}}]$.

\end{itemize}
\end{proof}

\clearpage
\newpage
