\chapter{Proofs}

\begin{lemma}
Let $a$ be a term and $a^0$ be a subterm of $a$. When
\[
a \longrightarrow' b \land a^0 \longrightarrow^{aux} b^0
\]
and both reduction steps are derived by the ``Subst'' rule, then there is a $d$ such that
\[
a[a^0 \mapsto b^0] \longrightarrow' d \land b {\longrightarrow^{aux}}^* d.
\]

\begin{proof}
Since $a^0 = q_{r'}[\sigma']$ for some equation $r'$, we know that $a^0$ is a function or destructor call. By Lemma ???, because $a = q_r[\sigma], b = t_r[\sigma]$ for an equation $r$ and $a^0$ is a subterm of $a$, it follows that $a^0$ is a subterm of a right-hand side of $\sigma$. Put in another way, there is a variable $x$ and a term $t(a^0)$ with subterm $a^0$ such that $\{x \mapsto t\} \subseteq \sigma$. Define a $\sigma_{mod}$ with $dom(\sigma_{mod}) = \sigma$ as $\sigma_{mod}(x) = t(b^0)$ and $\sigma_{mod}(y) = \sigma(y)$ for $y \neq x$, and choose $d = t_r[\sigma_{mod}]$. The two desired reductions can be derived as follows.
\begin{enumerate}
\item By the ``Subst'' rule for $r$:
\[
a[a^0 \mapsto b^0] = q_r[\sigma_{mod}] \longrightarrow' t_r[\sigma_{mod}] = d
\]
This is possible because all subterms of $a$ are values under judgement $\vdash'_v$, and its subterm $a^0$ is replaced by $b^0$ for which $\langle a^0 \rangle^{aux^{-1}} = \langle b^0 \rangle^{aux^{-1}}$ holds, thus it is $\vdash'_v b^0$, as well.

\item
\[
b = t_r[\sigma] \longrightarrow^{aux} t_r[\sigma][t(b^0)]_{p^x_1} \longrightarrow^{aux} ... \longrightarrow^{aux} t_r[\sigma][t(b^0)]_{p^x_n} = t_r[\sigma_{mod}] = d
\]
\end{enumerate}
\end{proof}
\end{lemma}

\begin{lemma}
Let $a$ be a term and $a^0$ be a subterm of $a$. When
\[
a \longrightarrow^{aux} b \land a^0 \longrightarrow' b^0
\]
and $a[a^0 \mapsto []] \in EC[\vdash'_v]$ and both reduction steps are derived by the ``Subst'' rule, then there is a $d$ such that
\[
a[a^0 \mapsto b^0] \longrightarrow^{aux} d \land b \longrightarrow' d.
\]

\begin{proof}
Since $a^0 = q_{r'}[\sigma']$ for some equation $r'$, we know that $a^0$ is a function or destructor call. By Lemma ???, because $a = q_r[\sigma], b = t_r[\sigma]$ for an equation $r$ and $a^0$ is a subterm of $a$, it follows that $a^0$ is a subterm of a right-hand side of $\sigma$. Put in another way, there is a variable $x$ and a term $t(a^0)$ with subterm $a^0$ such that $\{x \mapsto t\} \subseteq \sigma$. Define a $\sigma_{mod}$ with $dom(\sigma_{mod}) = \sigma$ as $\sigma_{mod}(x) = t(b^0)$ and $\sigma_{mod}(y) = \sigma(y)$ for $y \neq x$, and choose $d = t_r[\sigma_{mod}]$. The two desired reductions can be derived as follows.
\begin{enumerate}
\item Because $a \longrightarrow^{aux} b$, it must be $r = \epsilon$, as there is no other equation with unequal sides and its left-hand side interpreted the same as the right-hand side. Thus we have:
\[
a[a^0 \mapsto b^0] = q_\epsilon[\sigma_{mod}] \longrightarrow^{aux} t_\epsilon[\sigma_{mod}] = d
\]

\item
We know that $x$ only appears once in $t_\epsilon$, at some position $p^x$, thus it is
\[
b = t_\epsilon[\sigma] = \mathcal{E}'[a^0] \longrightarrow' \mathcal{E}'[b^0] = t_\epsilon[\sigma_{mod}] = d,
\]
with $\mathcal{E}' = b[a^0 \mapsto []]_{p^x}$. It is $\mathcal{E}' \in EC[\vdash'_v]$ because $a[a^0 \mapsto []] \in EC[\vdash'_v]$ and $\epsilon$ doesn't introduce any additional redexes.

\end{enumerate}
\end{proof}
\end{lemma}

\cdpaux*
\begin{proof}
We make use of the fact that a reduction that can be derived by multiple applications of the ``Cong" rule can also be derived by just one application of the ``Cong" rule. When a reduction is derived by the ``Subst'' rule, it can also be derived by applying the ``Cong'' rule for the empty evaluation context on top of that. Thus, we have the following:
\begin{itemize}
\item From $a \longrightarrow' b$: $a = \mathcal{E}[a^0], b = \mathcal{E}[b^0]$ for an $\mathcal{E} \in \textrm{EC}_<[\vdash'_v]$ and terms $a^0, b^0$ with $a^0 \longrightarrow' b^0$ and $a^0 = q_r[\sigma], b^0 = t_r[\sigma]$ for some equation $r$.

\item From $a \longrightarrow^{aux} c$: $a = \mathcal{E}^{aux}[a'^0], c = \mathcal{E}^{aux}[c^0]$ for some term with hole $\mathcal{E}^{aux}$ and terms $a'^0, c^0$ with $a'^0 \longrightarrow^{aux} c^0$ and $a'^0 = q_\epsilon[\sigma'], c^0 = t_\epsilon[\sigma']$.
\end{itemize}

We compare $a^0$ and $a'^0$ and distinguish three cases.
\begin{itemize}
\item $a'^0$ is a subterm of $a^0$. By Lemma A.0.1, we have a $d^0$ with $a^0[a'^0 \mapsto c^0] \longrightarrow' d^0$ and $b^0 \longrightarrow^{aux} d^0$. Choose $d = \mathcal{E}[d^0]$. The property holds since (a) $b = \mathcal{E}[b^0] \longrightarrow^{aux} \mathcal{E}[d^0] = d$ and (b) $c = \mathcal{E}^{aux}[c^0] = \mathcal{E}[a^0[a'^0 \mapsto c^0]] \longrightarrow' \mathcal{E}[d^0] = d$.

\item $a^0$ is a subterm of $a'^0$. Because $a = \mathcal{E}[a^0]$, $a^0$ is a subterm of $a'^0$, and $\mathcal{E} \in \textrm{EC}[\vdash'_v]$, it is $a'^0[a^0 \mapsto []] \in \textrm{EC}[\vdash'_v]$. By Lemma A.0.2, we have a $d^0$ with $a'^0[a^0 \mapsto b^0] \longrightarrow^{aux} d^0$ and $c^0 \longrightarrow' d^0$. Choose $d = \mathcal{E}^{aux}[d^0]$. The property holds since (a) $b = \mathcal{E}^{aux}[a'^0[a^0 \mapsto b^0]] \longrightarrow^{aux} \mathcal{E}^{aux}[d^0]$ and (b) $c = \mathcal{E}^{aux}[c^0] = \mathcal{E}[a'^0[a^0 \mapsto []] \mapsto []][c^0] \longrightarrow' \mathcal{E}[a'^0[a^0 \mapsto []] \mapsto []][d^0] = \mathcal{E}^{aux}[d^0] = d$.

\item Neither $a'^0$ is a subterm of $a^0$, nor the other way around. Then choose $d$ as $a$ with $a'^0$ replaced by $c^0$ and $a^0$ replaced by $b^0$. The property holds because $b = \mathcal{E}[b^0] = \mathcal{E}^{aux}[a^0 \mapsto b^0][a'^0] \longrightarrow^{aux} \mathcal{E}^{aux}[a^0 \mapsto b^0][c^0] = d$ and $c = \mathcal{E}^{aux}[c^0] = \mathcal{E}[a'^0 \mapsto c^0][a^0] \longrightarrow' \mathcal{E}[a'^0 \mapsto c^0][b^0] = d$. This second reduction is possible because, by Lemma 1.5.1, $\mathcal{E}[a'^0 \mapsto c^0] \in \textrm{EC}[\vdash'_v]$ since $\mathcal{E} \in \textrm{EC}[\vdash'_v]$ and $\langle a'^0 \rangle^{aux^{-1}} = \langle c^0 \rangle^{aux^{-1}}$ and thus $\vdash'_v a'^0$ iff $\vdash'_v c^0$.
\end{itemize}

\end{proof}

\correctb*
\begin{proof}
By induction on the length $n$ of the sequence.

For $n = 0$, the translation doesn't change anything, thus correctness trivially holds.

Now, suppose $n = n' + 1$ such that correctness of the translation holds for all sequences of length $n'$ (the induction hypothesis). This means that
\[
s^{uc}_1 \longrightarrow_{prg} ... \longrightarrow_{prg} s^{uc}_{m'} = \langle s_{n-1} \rangle^{aux^{-1}}
\]
for the sequence $(s^{uc}_1, ..., s^{uc}_m) := uc(s_1, ..., s_{n-1})$. Two cases will be distinguished, according to the definition of $uc$:
\begin{itemize}
\item \underline{Case 1:} $s_{n-1} \neq s_n \land s_{n-1} \sim s_n$.
Then it is $uc((s_1, ..., s_n)) = uc((s_1, ..., s_{n-1})) = (s^{uc}_1, ..., s^{uc}_m)$. By the above, all steps in the sequence are indeed reduction steps of the relevant reduction. Because $s_{n-1} \sim s_n$, it is $\langle s_n \rangle^{aux^{-1}} = \langle s_{n-1} \rangle^{aux^{-1}} = \langle s^{uc}_n \rangle^{aux^{-1}}$.

\item \underline{Case 2:} otherwise.
Then it is $uc((s_1, ..., s_n)) = uc((s_1, ..., s_{n-1})) \cdot (\langle s_n \rangle^{aux^{-1}})$. By the above, all steps before the last are indeed reduction steps of the relevant reduction. The final term is by definition $\langle s_n \rangle^{aux^{-1}}$.

It remains to be shown why the final step is a reduction step of the relevant reduction. By the induction hypothesis, this step is $(\langle s_{n-1} \rangle^{aux^{-1}}, \langle s_n \rangle^{aux^{-1}})$. By assumption, it is $s_{n-1} \longrightarrow_{\langle prg \rangle} s_n$, and $\langle s_{n-1} \rangle^{aux^{-1}} \longrightarrow_{prg} \langle s_n \rangle^{aux^{-1}}$ follows from the more general $a \longrightarrow_{\langle prg \rangle} b \Rightarrow \langle a \rangle^{aux^{-1}} \longrightarrow_{\langle prg \rangle} \langle b \rangle^{aux^{-1}}$. This can be shown by induction on the derivation of the left-hand side reduction, by using
\[
\langle prg \rangle \vdash_v t \Rightarrow prg \vdash_v \langle t \rangle^{aux^{-1}},
\]
for all terms $t$, and the compatibility of evaluation contexts and $aux^{-1}$, i.e., $\langle \mathcal{E}[t] \rangle^{aux^{-1}} = \langle \mathcal{E} \rangle^{aux^{-1}}[\langle t \rangle^{aux^{-1}}]$.

\end{itemize}
\end{proof}

\clearpage
\newpage
