\chapter{Extraction transformations}

Most of the steps that comprise the transformations of the following chapters are \textit{extractions}. Roughly speaking, this means that they decrease, in some way, the syntactic complexity of the program, while preserving its semantics to some degree.

As an example, consider the extraction of a destructor out of the following program fragment:
\begin{lstlisting}
fun().des().des() = t
\end{lstlisting}

\begin{lstlisting}
fun().des() = aux()
aux().des() = t
\end{lstlisting}
The resulting program fragment doesn't contain any copatterns with two destructors, unlike the original, thus the transformation has decreased the syntactic complexity in this way. The transformation preserves the semantics because, roughly, in the transformed program there is a way to go from the original equation's lhs to its rhs: $\mathtt{fun().des().des()} \longrightarrow \mathtt{aux().des()} \longrightarrow \mathtt{t}$.

\section{Extractions}

Extractions are induced by a function $\pi$ that specifies how to decrease the syntactic complexity. Before defining extractions a way is described to uniformly express extractions ``from the outside in'', such as the destructor extraction exemplified above.

Changing the ``outer'' form of the copattern, i.e., removing or adding destructors, is dual to changing its ``inner'' form, i.e., replacing patterns with other patterns. For the purpose of defining extractions of the latter type, substitutions will be used. For the purpose of defining extractions of the former, the notion of \textit{co-substitution} is introduced.

\begin{definition}[Co-substitution]
A function $\sigma$ from copatterns to copatterns is called a \textit{co-substitution}, if, for every copattern $q$, it is defined as follows:
\[
\sigma(q) = q.\overline{des(\overline{p})},
\]
for some $\overline{des(\overline{p})}$ possibly depending on the $q$.
\end{definition}

Finally, extraction functions can be defined. More precisely, it is defined what it means to be an extraction projection, and for such an extraction projection $\pi$, a $\pi$-extraction targeting a copattern $q$.

\begin{definition}[Extraction projection]
A function $\pi$ from copatterns to copatterns is called an extraction projection if for every copattern $q$ there exists a (co-)substitution $\sigma^q_\pi$ such that $\sigma^q_\pi(\langle q \rangle^\pi) = q$.
\end{definition}

\begin{definition}[$\pi$-lens]
The $\pi$-lens, for an extraction projection $\pi$, is the lens defined as follows:
\[
\mathtt{get} = \pi
\]
\[
\mathtt{putback}(q^a, q^c) = \sigma^{q^c}_\pi(q^a),
\]
where the $\sigma^{q^c}_\pi$ is the (co)-substitution for $q^c$ and $\pi$ as given in the definition of the extraction projection.
\end{definition}

\begin{definition}[$\pi$-extraction targeting $q$]
A function $e$ is called a $\pi$-extraction targeting $q$, when it is a function from equations to the union of equations and pairs of equations such that, for all equations $r$:
\begin{itemize}
\item If $\mathtt{get}(q_r) = \mathtt{get}(q)$, then $\langle r \rangle^e = \big\langle \epsilon, \zeta_r \big\rangle$ with
\begin{itemize}
\item $q_\epsilon = \mathtt{get}(q_r)$,
\item $t_\epsilon = \langle q_\epsilon \rangle^{aux}$,
\item $q_{\zeta_r} = \mathtt{putback}(t_\epsilon, q_r)$,
\item $t_{\zeta_r} = t_r$.
\end{itemize}

\item If $\mathtt{get}(q_r) \neq \mathtt{get}(q)$, then $\langle r \rangle^e = r$.
\end{itemize}
Here, the pair of \texttt{get} and \texttt{putback} is the $\pi$-lens. $\langle \cdot \rangle^{aux}$ is the call to the auxiliary function that corresponds to the given copattern, defined as follows for copatterns $q$:
\[
\langle q \rangle^{aux} = aux(\langle q \rangle^{vars}),
\]
where $aux$ is a fresh function name.\footnote{In practice this means that is in undeclared in the program that is transformed by the extraction lifted to programs (see next section).}
\end{definition}

\section{Lifting extractions to programs}

An extraction function was defined to have equations as its domain and range. As it will be used to transform programs into programs, however, it is necessary to define how it should be lifted to that domain. The definition of the lifting function is straightforward; all it does is apply the extraction to each equation $r$ of the program, replacing it with $\epsilon$ if it exists and otherwise leaving $r$ unchanged, and collecting the $\zeta_r$ for each changed equation $r$ in an auxiliary function definition.

...

\section{Bisimulation}

Every $\pi$-extraction function targeting a $q$ preserves the semantics of programs in a kind of weak bisimulation. There are two equivalent characterizations of this bisimulation. One is given and proved in the first subsection, the other follows in the second subsection and is shown to be equivalent to the first.

Before that, a function $\langle \cdot \rangle^{aux^{-1}}$ from terms to terms needs to be defined that serves as the opposite of $\langle \cdot \rangle^{aux}$. It is defined as replacing calls to the auxiliary function as generated by $\langle \cdot \rangle^{aux}$ with the original copatterns with their variables instantiated accordingly. This corresponds to what Setzer et al. call a \textit{back-interpretation} in the framework of their paper ``Unnesting of copatterns''; this term will be used here as well when referring to $\langle \cdot \rangle^{aux^{-1}}$.

\subsection{Using a modified value judgement}

For the definition of this weak bisimulation, a modification, using the back-interpretation, of the value judgement needs to be defined. For any term $t$ with names declared in $\langle prg \rangle$, let
\[
\langle prg \rangle \vdash'_v t :\iff prg \vdash_v \langle t \rangle^{aux^{-1}}.
\]

From this it immediately follows that $\mathcal{E}$ is an evaluation context with respect to this modified value judgement for $\langle prg \rangle$ if and only if $\langle \mathcal{E} \rangle^{aux^{-1}}$ is an evaluation context with respect to the original value judgement for $prg$. It also means that, for a term $t$ with all names declared in $\langle prg \rangle$, the immediate subterms of $t$ are values with this judgement for $\langle prg \rangle$ if and only if the immediate subterms of $\langle t \rangle^{aux^{-1}}$ are values with the original judgement for $prg$.

Write $\longrightarrow'$ for the reduction relation $\longrightarrow$ with its value judgement modified in this way. The weak bisimulation is defined as follows, for every $s,t$ with all of their names declared in $prg$:
\begin{equation}
s \longrightarrow_{prg}^* t \iff s {\longrightarrow'}_{\langle prg \rangle}^* t
\end{equation}
This statement is now proved using Theorem 4 of ``Unnesting of copatterns'' (Setzer et al.).

\begin{proposition}
The weak bisimulation statement (2.1) holds for any transformation defined as $liftp(e)$ for some $\pi$-extraction targeting a $q$.

\begin{proof}
By Theorem 4 of ``Unnesting of copatterns'' (Setzer et al.), it suffices to show the statements (SN1) and (SN2), defined there along with the theorem. For this, set $\textrm{Int} = \langle \cdot \rangle^{aux^{-1}}$. Note that this unconversion is compatible with evaluation contexts, i.e.,
\[
\langle \mathcal{E}[s'] \rangle^{aux^{-1}} = \langle \mathcal{E} \rangle^{aux^{-1}}[\langle s' \rangle^{aux^{-1}}],
\]
because $\langle \cdot \rangle^{aux}$ converts to function calls, not destructor calls.
Further, set $m$ as the number of calls to the function targeted in the transformation, i.e., that of $q_\epsilon$.

(SN1): We know that $s = \mathcal{E}[s'] = \mathcal{E}[q_r[\sigma]] \longrightarrow_{prg} \mathcal{E}[t_r[\sigma]] = t$, for some evaluation context $\mathcal{E}$ of $prg$ and some equation $r$ of $prg$.
\[
s = \mathcal{E}[s'] = \mathcal{E}[\sigma^{q_r}_\pi(q_\epsilon)[\sigma]]
\]
\[
\longrightarrow'_{\langle prg \rangle} \mathcal{E}[\sigma^{q_r}_\pi(t_\epsilon)[\sigma]] = \mathcal{E}[q_{\zeta_r}[\sigma]]
\]
\[
\longrightarrow'_{\langle prg \rangle} \mathcal{E}[t_{\zeta_r}[\sigma]] = \mathcal{E}[t_r[\sigma]] = t
\]

(SN2): Three cases will be distinguished: The reduction in $\langle prg \rangle$ can use either an equation taken over unchanged from $prg$ (1.), it can use a $\zeta_r$ (2.), or it can use $\epsilon$. Each case makes use of an evaluation context $\mathcal{E}$ of the reduction relation for $\langle prg \rangle$.
\begin{enumerate}
\item We know $s = \mathcal{E}[s'] = \mathcal{E}[q_r[\sigma]] \longrightarrow'_{\langle prg \rangle} \mathcal{E}[t_r[\sigma]] = t$. The desired reduction sequence can be given as follows:
\[
\langle \mathcal{E}[s'] \rangle^{aux^{-1}} = \langle \mathcal{E} \rangle^{aux^{-1}}[\langle s' \rangle^{aux^{-1}}] = \langle \mathcal{E} \rangle^{aux^{-1}}[\langle q_r \rangle^{aux^{-1}}[\langle \sigma \rangle^{aux^{-1}}]]
\]
\[
 \longrightarrow_{prg} \langle \mathcal{E} \rangle^{aux^{-1}}[t_r[\langle t_r \rangle^{aux^{-1}}]] = \langle \mathcal{E} \rangle^{aux^{-1}}[\langle t_r \rangle^{aux^{-1}}[\langle \sigma \rangle^{aux^{-1}}]] = \langle \mathcal{E} \rangle^{aux^{-1}}[\langle t_r[\sigma] \rangle^{aux^{-1}}]
\]
\[
= \langle \mathcal{E}[t_r[\sigma]] \rangle^{aux^{-1}} = \langle t \rangle^{aux^{-1}}.
\]

\item We know $s = \mathcal{E}[s'] = \mathcal{E}[q_{\zeta_r}[\sigma]] \longrightarrow'_{\langle prg \rangle} \mathcal{E}[t_{\zeta_r}[\sigma]] = t$. The desired reduction sequence can be given as follows:
\[
\langle \mathcal{E}[s'] \rangle^{aux^{-1}} = \langle \mathcal{E} \rangle^{aux^{-1}}[\langle s' \rangle^{aux^{-1}}] = \langle \mathcal{E} \rangle^{aux^{-1}}[\langle q_{\zeta_r} \rangle^{aux^{-1}}[\langle \sigma \rangle^{aux^{-1}}]] = \langle \mathcal{E} \rangle^{aux^{-1}}[q_r[\langle \sigma \rangle^{aux^{-1}}]]
\]
\[
\longrightarrow_{prg} \langle \mathcal{E} \rangle^{aux^{-1}}[t_r[\langle t_r \rangle^{aux^{-1}}]] = \langle \mathcal{E} \rangle^{aux^{-1}}[\langle t_r \rangle^{aux^{-1}}[\langle \sigma \rangle^{aux^{-1}}]] = \langle \mathcal{E} \rangle^{aux^{-1}}[\langle t_r[\sigma] \rangle^{aux^{-1}}]
\]
\[
= \langle \mathcal{E}[t_r[\sigma]] \rangle^{aux^{-1}} = \langle t \rangle^{aux^{-1}}.
\]

\item We know $s = \mathcal{E}[s'] = \mathcal{E}[q_\epsilon[\sigma]] \longrightarrow'_{\langle prg \rangle} \mathcal{E}[t_\epsilon[\sigma]] = t$. In this case, instead of giving a reduction sequence, the other side of the disjunction will be shown to hold.

Because $\langle q_\epsilon \rangle^{aux^{-1}} = \langle t_\epsilon \rangle^{aux^{-1}}$, it is $\langle s \rangle^{aux^{-1}} = \langle t \rangle^{aux^{-1}}$.

And because $q_\epsilon$ is of the function that is targeted in the transformation, and $t_\epsilon$ isn't, it is $m(q_\epsilon) > m(t_\epsilon)$, and consequently $m(s) > m(t)$.
\end{enumerate}
\end{proof}
\end{proposition}

\subsection{Using the back-interpretation directly}

This characterization of the bisimulation, equivalent to the first, directly uses the back-interpretation. In short, extractions preserve semantic properties by introducing an equation which leads from a, syntactically, more complex to a less complex term, where both terms are meant to represent, semantically, the same ``object''. This ``sameness'' is expressed by one being the back-interpretation of the other.

The bisimulation is characterized as follows:
\[
s {\longrightarrow}_{prg}^* t \iff s \longrightarrow^*_{\langle prg \rangle} \widetilde{t}, \text{ with } \langle \widetilde{t} \rangle^{aux^{-1}} = t
\]
In order to prove it, it suffices to show that this characterization is equivalent to the first characterization:
\begin{equation}
s {\longrightarrow'}_{\langle prg \rangle}^* t \iff s \longrightarrow^*_{\langle prg \rangle} \widetilde{t},
\end{equation}
for every $s, t$ with names declared in $prg$.

To show that the $`` \Rightarrow ''$ direction of (2.2) holds, a translation for sequences on the left-hand side of (2.2) to the corresponding sequences on the right-hand side is given (translation`A'). For the other direction, a translation for sequences $s \longrightarrow^*_{\langle prg \rangle} \widetilde{t}$ to sequences $s \longrightarrow^*_{prg} t$ is given (translation `B'). For both translations, their correctness is verified, i.e., it is proven that each step of the sequences returned by the function is indeed a reduction step of the relevant reduction, and that the desired equality or equivalence between the end points of the original and the translated sequence holds. These correctness proofs are left for the appendix.

\begin{definition}[Translation `A']
Translation `A' is given by a conversion function $c$ from sequences to sequences. It is defined recursively as follows.
\[
  c((s_1, ..., s_n))=\begin{cases}
               (), &\text{ if } n = 0 \\
               (s_1), &\text{ if } n = 1 \\
               \langle \mathcal{E} \rangle^{aux^*}[s'_1] \cdot \langle \mathcal{E} \rangle^{aux^*}[\langle s'_1 \rangle^{aux^*}] \cdot c(\langle (s_2, ..., s_n) \rangle^{aux^{max}_{s_1}}), &\text{ otherwise } \\
            \end{cases}
\]
The definition makes use of $\langle \cdot \rangle^{aux^*}$ and $\langle \cdot \rangle^{aux^{max}_{s_1}}$, defined below.

\begin{itemize}
\item $\langle \cdot \rangle^{aux^*}$ is defined for evaluation contexts and terms. First, define
\begin{alignat*}{2}
\langle \mathcal{E} \rangle^{aux} = &\mathcal{E} \text{ with the left-most subterm } t \text{ for which } \langle prg \rangle \vdash'_v t \text{ but not } \langle prg \rangle \vdash_v t \\
& \text{replaced by } \langle t \rangle^{aux}.
\end{alignat*}
For terms, $\langle \cdot \rangle^{aux}$ is defined analogously. Now, $\langle \cdot \rangle^{aux^*}$ can be defined as the sequence that obtains when applying $\langle \cdot \rangle^{aux}$ iteratively until it converges.

\item $\langle \cdot \rangle^{aux^{max}_{-r}}$ is defined for terms. The notation used above for term sequences is shorthand for applying $\langle \cdot \rangle^{aux^{max}_{-r}}$ to each term in the sequence. For a term $t$, $\langle \cdot \rangle^{aux^{max}_{s_1}}$ means $t$ with each subterm $t'$ for which
\begin{enumerate}
\item $\langle prg \rangle \vdash'_v t'$, but
\item $\langle prg \rangle \not\vdash_v t'$, and
\item $t'$ is already present in $s_1$,
\end{enumerate}
replaced by $\langle t' \rangle^{aux}$.

\end{itemize}
\end{definition}

\begin{restatable}[Correctness of translation `A']{lemma}{correcta}
Translation `A' is correct, that is, when
\[
s = s_1 {\longrightarrow'}_{\langle prg \rangle} ... {\longrightarrow'}_{\langle prg \rangle} s_n = t
\]
then
\[
s^c_1 \longrightarrow_{\langle prg \rangle} ... \longrightarrow_{\langle prg \rangle} s^c_m \text{ with } \langle s^c_m \rangle^{aux^{-1}} = t
\]
for the sequence $(s^c_1, ..., s^c_m) := c((s_1, ..., s_n))$.
\end{restatable}
\begin{proof}
Left for the appendix.
\end{proof}

\begin{definition}[Translation `B']
Translation `A' is given by a conversion function $uc$ from sequences to sequences. It is defined recursively as follows.
\[
  uc((s_0, ..., s_n))=\begin{cases}
               (), &\text{ if } n = 0 \\
               uc((s_0, ..., s_{n-1})), &\text{ if } s_{n-1} \neq s_n \land \langle s_{n-1} \rangle^{aux^{-1}} = \langle s_n \rangle^{aux^{-1}} \\
               uc((s_0, ..., s_{n-1})) \cdot (\langle s_n \rangle^{aux^{-1}}), &\text{ otherwise} \\
            \end{cases}
\]

\end{definition}

\begin{restatable}[Correctness of translation `B']{lemma}{correctb}
Translation `B' is correct, that is, when
\[
s = s_1 {\longrightarrow}_{\langle prg \rangle} ... {\longrightarrow}_{\langle prg \rangle} s_n \text{ with } \langle s_n \rangle^{aux^{-1}} = t
\]
then
\[
s^{uc}_1 \longrightarrow_{prg} ... \longrightarrow_{prg} s^{uc}_m t
\]
for the sequence $(s^{uc}_1, ..., s^{uc}_m) := uc((s_1, ..., s_n))$.
\end{restatable}
\begin{proof}
Left for the appendix.
\end{proof}

Finally, the bisimulation statement can be proven.
\begin{proposition}
Statement (2.2), that is
\[
s {\longrightarrow'}_{\langle prg \rangle}^* t \iff s \longrightarrow^*_{\langle prg \rangle} \widetilde{t},
\]
for every $s, t$ with names declared in $prg$, holds.

\begin{proof}
``$\Rightarrow$'': Assume that there is a sequence of the form on the left-hand side. Then by translation `A', there also is a corresponding right-hand side sequence.

``$\Leftarrow$'': Assume that there is a sequence of the form on the left-hand side. By translation `B', there also is a sequence $s \longrightarrow_{prg} t$, and by (2.1) the desired sequence $s {\longrightarrow'}_{\langle prg \rangle} t$ obtains.

\end{proof}

\end{proposition}

\section{Absence of overlaps}

Since, in the context of this work, it is always presupposed that a program to be transformed has no overlapping copatterns, a desirable property of the transformed program is that it doesn't have overlapping copatterns, as well. This is now shown to be the case whenever $q_\epsilon$ doesn't overlap with any lhs of an equation taken over unchanged from $prg$.

\begin{proposition}
For any well-behaved extraction function lifted to programs, $\langle \cdot \rangle$, it is the case that if $prg$ has no overlapping copatterns and $q_\epsilon$ doesn't overlap with any lhs of an equation taken over unchanged from $prg$, then $\langle prg \rangle$ has no overlapping copatterns, as well.

\begin{proof}
First, the equations of the transformed are classified. There are three kinds of them: Those taken over unchanged over from $prg$ (indicated as $r$ in the table below), those which are an $\epsilon$ in a transformation result ($\epsilon$), and those which are, also in such a transformation result, a $\zeta_r$ ($\zeta$). The table below shows all possible combinations; its fields are filled with the number of the proof that lhss of equations of the respective kinds don't overlap.

\begin{tabular}{ l | c | c | r }  & r & $\epsilon$ & $\zeta$ \\ \hline r & (1) &  &  \\ \hline $\epsilon$ & (2) & (3) &  \\ \hline $\zeta$ & (4) & (5) & (6) \\ \hline \end{tabular}

\textit{ad} (1): Both equations are present in $prg$, thus their lhss don't overlap.

\textit{ad} (2): By assumption.

\textit{ad} (3): By the definition of the $q$-extraction, there is only one $\epsilon$-equation in the transformed program.

\textit{ad} (4): The $\zeta$-equation has a function name not declared in $prg$, unlike the $r$-equation.

\textit{ad} (5): The $\zeta$-equation has a function name not declared in $prg$, unlike the $\epsilon$-equation.

\textit{ad} (6): The lhss of each of the $\zeta$-equations are equivalent to a lhs in $prg$, thus, if they overlapped, so would these lhss in $prg$, contrary to fact.
\end{proof}
\end{proposition}

\section{Example extractions}

The two following extractions will be used in the next chapter.

\subsection{Destructor extraction}

The extraction $des\_extract(q)$ of a single destructor targeting some copattern $q$ can be defined as follows: It is the $\pi$-extraction targeting $q$ for the extraction projection $\pi$ defined below.

\[
\pi(`` fun(\overline{p}) ") = `` fun(\overline{p}) "
\]
\[
\pi(`` q.des(\overline{p}) ") = `` q "
\]

Since copatterns without destructors aren't affected, this extraction is only meant to be used for copatterns with at least one destructor. 

\subsection{Constructor extraction}

In this section, the extraction of a single constructor is defined. Unlike the extraction of a single destructor from the previous section, for constructor extraction it needs to be specified which constructor should be extracted. Define the extraction $con_n\_extract(q)$ of the $n$-th constructor in $q$ (from left to right) as the $\pi_n$-extraction targeting $q$ with the extraction projection $\pi_n$ defined below.

Let $x_q$ be a variable different from all in $q$.
\[
\pi_n(q) = q \text{ with the $n$-th constructor application } con(...) \text{ replaced by $x_q$ }
\]

\subsection{Extracting all patterns out of a single-destructor copattern}

This extraction extracts all patterns, that is, all constructor calls, out of a copattern with only a single destructor. Define this extraction $extract\_patterns$ as the $\pi$-extraction targeting $q$ with the extraction projection $\pi$ defined below.

\[
\pi(`` fun(\overline{p}).des(\overline{p'}) ") = `` fun(\overline{x}, \overline{x'}) "
\]