\chapter{Extraction transformations}

Most of the steps that comprise the transformations of the following chapters are \textit{extractions}. Roughly speaking, this means that they decrease, in some way, the syntactic complexity of the program. It is usually desirable that they also preserve its semantics to some degree. This will be shown to be the case for all \textit{well-behaved} extractions, as defined in the following section.

As an example, consider the extraction of a destructor out of the following program fragment:
\begin{lstlisting}
fun().des().des() = t
\end{lstlisting}

\begin{lstlisting}
fun().des() = aux()
aux().des() = t
\end{lstlisting}
The resulting program fragment doesn't contain any copatterns with two destructors, unlike the original, thus the transformation has decreased the syntactic complexity in this way. The transformation is well-behaved because, roughly, in the transformed program there is a way to go from the original equation's lhs to its rhs: $\mathtt{fun().des().des()} \longrightarrow \mathtt{aux().des()} \longrightarrow \mathtt{t}$.

\section{Well-behaved extractions}

Well-behaved extractions are induced by a function $f$ that specifies how to decrease the syntactic complexity. Before defining well-behavedness, an equivalence relation for terms of the transformed program and depending on $f$ needs to be defined, and then a way to uniformly express extractions ``from the outside in'', such as the destructor extraction exemplified above.

In short, well-behaved extraction preserve semantic properties by introducing an equation which leads from a, syntactically, more complex to a less complex term, where both terms are meant to represent, semantically, the same ``object''. This ``sameness'' is expressed by an equivalence relation that underlies the definition of well-behaved extractions.

\begin{definition}[$f$-equivalence]
For a function $f$ from copatterns to copatterns, define the corresponding \textit{$f$-equivalence} $\sim_f$ as the smallest congruence relation between terms, with respect to their syntactic structure, for which it holds:
\[
\langle \langle q \rangle^f \rangle^{\sim}[\sigma] \sim \langle q \rangle^f[\sigma]
\]
for every copattern $q$ with $\langle q \rangle^f \neq q$, and every substitution $\sigma$.
\end{definition}

Changing the ``outer'' form of the copattern, i.e., removing or adding destructors, is dual to changing its ``inner'' form, i.e., replacing patterns with other patterns. For the purpose of defining well-behaved extractions of the latter type, substitutions will be used. For the purpose of defining well-behaved extraction of the former, the notion of \textit{co-substitution} is introduced.

\begin{definition}[Co-substitution]
A function $\sigma$ from copatterns to copatterns is called a \textit{co-substitution}, if, for every copattern $q$, it is defined as follows:
\[
\sigma(q) = q.\overline{des(\overline{p})},
\]
for some $\overline{des(\overline{p})}$ possibly depending on the $q$.
\end{definition}

Finally, well-behaved extraction functions can be defined. More precisely, it is defined what it means for a function $f$ to induce an extraction function, and then when such an extraction function induced by $f$ is well-behaved.

\begin{definition}[Lens induced by $f$]
The \textit{lens induced by $f$} is a pair of functions \texttt{get} and \texttt{putback} defined as follows:
\begin{itemize}
\item $\mathtt{get}(q^c) = \langle q^c \rangle^f$
\item $\mathtt{putback}(q^a, q^c) = \sigma^{q^c}_f(q^a)$
\end{itemize}
\end{definition}

\begin{definition}[Extraction function induced by $f$]
The extraction function induced by a function $f$, from copatterns to copatterns, is a function $e$ from equations to the union of equations and pairs of equations such that, for all equations $r$:
\begin{itemize}
\item If $\langle q_r \rangle^f \neq q_r$, then $\langle r \rangle^e = \big\langle \epsilon, \zeta_r \big\rangle$ with:
\begin{itemize}
\item $q_\epsilon = \mathtt{get}(q_r)$,
\item $t_\epsilon = \langle q_\epsilon \rangle^{\sim}$,
\item $q_{\zeta_r} = \mathtt{putback}(t_\epsilon, q_r)$,
\item $t_{\zeta_r} = t_r$.
\end{itemize}

\item If $\langle q_r \rangle^f \neq q_r$, then $\langle r \rangle^e = r$.
\end{itemize}
Here, the pair of \texttt{get} and \texttt{putback} is the lens induced by $f$.
\end{definition}

\begin{definition}[Well-behaved extractions]
The extraction function $e$ induced by a function $f$ is said to be \textit{well-behaved} if, for every copattern $q$, there is a substitution or co-substitution $\sigma^q_f$ such that:
\[
\sigma^q_f(\langle q \rangle^f) = q
\]
\end{definition}

\section{Lifting extractions to programs}

An extraction function was defined to have equations as its domain and range. As it will be used to transform programs into programs, however, it is necessary to define how it should be lifted to that domain. The definition of the lifting function is straightforward; all it does is apply the extraction to each equation $r$ of the program, replacing it with $\epsilon$ if it exists and otherwise leaving $r$ unchanged, and collecting the $\zeta_r$ for each changed equation $r$ in an auxiliary function definition.

...

\section{Proof of bisimulation}

Every well-behaved extraction function, as induced by some function $f$, preserves the semantics of programs in a kind of weak bisimulation. In this section, the precise kind of the bisimulation will be defined and this statement will be proved.

