\chapter{Extraction transformations}

Most of the steps that comprise the transformations of the following chapters are \textit{extractions}. Roughly speaking, this means that they decrease, in some way, the syntactic complexity of the program, while preserving its semantics to some degree.

As an example, consider the extraction of a destructor out of the following program fragment:
\begin{lstlisting}
fun().des().des() = t
\end{lstlisting}

\begin{lstlisting}
fun().des() = aux()
aux().des() = t
\end{lstlisting}
The resulting program fragment doesn't contain any copatterns with two destructors, unlike the original, thus the transformation has decreased the syntactic complexity in this way. The transformation preserves the semantics because, roughly, in the transformed program there is a way to go from the original equation's lhs to its rhs: $\mathtt{fun().des().des()} \longrightarrow \mathtt{aux().des()} \longrightarrow \mathtt{t}$.

\section{Extractions}

Extractions are induced by a function $\pi$ that specifies how to decrease the syntactic complexity. Before defining extractions a way is described to uniformly express extractions ``from the outside in'', such as the destructor extraction exemplified above.

Changing the ``outer'' form of the copattern, i.e., removing or adding destructors, is dual to changing its ``inner'' form, i.e., replacing patterns with other patterns. For the purpose of defining extractions of the latter type, substitutions will be used. For the purpose of defining extractions of the former, the notion of \textit{co-substitution} is introduced.

\begin{definition}[Co-substitution]
A function $\sigma$ from copatterns to copatterns is called a \textit{co-substitution}, if, for every copattern $q$, it is defined as follows:
\[
\sigma(q) = q.\overline{des(\overline{p})},
\]
for some $\overline{des(\overline{p})}$ possibly depending on the $q$.
\end{definition}

Finally, extraction functions can be defined. More precisely, it is defined what it means to be an extraction projection, and for such an extraction projection $\pi$, a $\pi$-extraction targeting a copattern $q$.

\begin{definition}[Extraction projection]
A function $\pi$ from copatterns to copatterns is called an extraction projection if for every copattern $q$ there exists a (co-)substitution $\sigma^q_\pi$ such that $\sigma^q_\pi(\langle q \rangle^\pi) = q$.
\end{definition}

\begin{definition}[$\pi$-lens]
The $\pi$-lens, for an extraction projection $\pi$, is the lens defined as follows:
\[
\mathtt{get} = \pi
\]
\[
\mathtt{putback}(q^a, q^c) = \sigma^{q^c}_\pi(q^a),
\]
where the $\sigma^{q^c}_\pi$ is the (co)-substitution for $q^c$ and $\pi$ as given in the definition of the extraction projection.
\end{definition}

\begin{definition}[$\pi$-extraction targeting $q$]
A function $e$ is called a $\pi$-extraction targeting $q$, when it is a function from equations to the union of equations and pairs of equations such that, for all equations $r$:
\begin{itemize}
\item If $\mathtt{get}(q_r) = \mathtt{get}(q)$, then $\langle r \rangle^e = \big\langle \epsilon, \zeta_r \big\rangle$ with
\begin{itemize}
\item $q_\epsilon = \mathtt{get}(q_r)$,
\item $t_\epsilon = \langle q_\epsilon \rangle^{aux}$,
\item $q_{\zeta_r} = \mathtt{putback}(t_\epsilon, q_r)$,
\item $t_{\zeta_r} = t_r$.
\end{itemize}

\item If $\mathtt{get}(q_r) \neq \mathtt{get}(q)$, then $\langle r \rangle^e = r$.
\end{itemize}
Here, the pair of \texttt{get} and \texttt{putback} is the $\pi$-lens. $\langle \cdot \rangle^{aux}$ is the call to the auxiliary function that corresponds to the given copattern, defined as follows for copatterns $q$:
\[
\langle q \rangle^{aux} = aux(\langle q \rangle^{vars}),
\]
where $aux$ is a fresh function name.\footnote{In practice this means that is in undeclared in the program that is transformed by the extraction lifted to programs (see next section).}
\end{definition}

\section{Lifting extractions to programs}

An extraction function was defined to have equations as its domain and range. As it will be used to transform programs into programs, however, it is necessary to define how it should be lifted to that domain. The definition of the lifting function is straightforward; all it does is apply the extraction to each equation $r$ of the program, replacing it with $\epsilon$ if it exists and otherwise leaving $r$ unchanged, and collecting the $\zeta_r$ for each changed equation $r$ in an auxiliary function definition.

...

\section{Bisimulation}

Every $\pi$-extraction function targeting a $q$ preserves the semantics of programs in a kind of weak bisimulation. There are two equivalent characterizations of this bisimulation. One is given and proved in the first subsection, the other follows in the second subsection and is shown to be equivalent to the first.

Before that, a function $\langle \cdot \rangle^{aux^{-1}}$ from terms to terms needs to be defined that serves as the opposite of $\langle \cdot \rangle^{aux}$. It is defined as replacing calls to the auxiliary function as generated by $\langle \cdot \rangle^{aux}$ with the original copatterns with their variables instantiated accordingly. This corresponds to what Setzer et al. call a \textit{back-interpretation} in the framework of their paper ``Unnesting of copatterns''; this term will be used here as well when referring to $\langle \cdot \rangle^{aux^{-1}}$.

\subsection{Using a modified value judgement}

For the definition of this weak bisimulation, a modification, using the back-interpretation, of the value judgement needs to be defined. For any term $t$ with names declared in $\langle prg \rangle$, let
\[
\langle prg \rangle \vdash'_v t :\iff prg \vdash_v \langle t \rangle^{aux^{-1}}.
\]

From this it immediately follows that $\mathcal{E}$ is an evaluation context with respect to this modified value judgement for $\langle prg \rangle$ if and only if $\langle \mathcal{E} \rangle^{aux^{-1}}$ is an evaluation context with respect to the original value judgement for $prg$. It also means that, for a term $t$ with all names declared in $\langle prg \rangle$, the immediate subterms of $t$ are values with this judgement for $\langle prg \rangle$ if and only if the immediate subterms of $\langle t \rangle^{aux^{-1}}$ are values with the original judgement for $prg$.

Write $\longrightarrow'$ for the reduction relation $\longrightarrow$ with its value judgement modified in this way. The weak bisimulation is defined as follows, for every $s,t$ with all of their names declared in $prg$:
\begin{equation}
s \longrightarrow_{prg}^* t \iff s {\longrightarrow'}_{\langle prg \rangle}^* t
\end{equation}
This statement is now proved using Theorem 4 of ``Unnesting of copatterns'' (Setzer et al.).

\begin{proposition}
The weak bisimulation statement (2.1) holds for any transformation defined as $liftp(e)$ for some $\pi$-extraction targeting a $q$.

\begin{proof}
By Theorem 4 of ``Unnesting of copatterns'' (Setzer et al.), it suffices to show the statements (SN1) and (SN2), defined there along with the theorem. For this, set $\textrm{Int} = \langle \cdot \rangle^{aux^{-1}}$. Note that this unconversion is compatible with evaluation contexts, i.e.,
\[
\langle \mathcal{E}[s'] \rangle^{aux^{-1}} = \langle \mathcal{E} \rangle^{aux^{-1}}[\langle s' \rangle^{aux^{-1}}],
\]
because $\langle \cdot \rangle^{aux}$ converts to function calls, not destructor calls.
Further, set $m$ as the number of calls to the function targeted in the transformation, i.e., that of $q_\epsilon$.

(SN1): We know that $s = \mathcal{E}[s'] = \mathcal{E}[q_r[\sigma]] \longrightarrow_{prg} \mathcal{E}[t_r[\sigma]] = t$, for some evaluation context $\mathcal{E}$ of $prg$ and some equation $r$ of $prg$.
\[
s = \mathcal{E}[s'] = \mathcal{E}[\sigma^{q_r}_\pi(q_\epsilon)[\sigma]]
\]
\[
\longrightarrow'_{\langle prg \rangle} \mathcal{E}[\sigma^{q_r}_\pi(t_\epsilon)[\sigma]] = \mathcal{E}[q_{\zeta_r}[\sigma]]
\]
\[
\longrightarrow'_{\langle prg \rangle} \mathcal{E}[t_{\zeta_r}[\sigma]] = \mathcal{E}[t_r[\sigma]] = t
\]

(SN2): Three cases will be distinguished: The reduction in $\langle prg \rangle$ can use either an equation taken over unchanged from $prg$ (1.), it can use a $\zeta_r$ (2.), or it can use $\epsilon$. Each case makes use of an evaluation context $\mathcal{E}$ of the reduction relation for $\langle prg \rangle$.
\begin{enumerate}
\item We know $s = \mathcal{E}[s'] = \mathcal{E}[q_r[\sigma]] \longrightarrow'_{\langle prg \rangle} \mathcal{E}[t_r[\sigma]] = t$. The desired reduction sequence can be given as follows:
\[
\langle \mathcal{E}[s'] \rangle^{aux^{-1}} = \langle \mathcal{E} \rangle^{aux^{-1}}[\langle s' \rangle^{aux^{-1}}] = \langle \mathcal{E} \rangle^{aux^{-1}}[\langle q_r \rangle^{aux^{-1}}[\langle \sigma \rangle^{aux^{-1}}]]
\]
\[
 \longrightarrow_{prg} \langle \mathcal{E} \rangle^{aux^{-1}}[t_r[\langle t_r \rangle^{aux^{-1}}]] = \langle \mathcal{E} \rangle^{aux^{-1}}[\langle t_r \rangle^{aux^{-1}}[\langle \sigma \rangle^{aux^{-1}}]] = \langle \mathcal{E} \rangle^{aux^{-1}}[\langle t_r[\sigma] \rangle^{aux^{-1}}]
\]
\[
= \langle \mathcal{E}[t_r[\sigma]] \rangle^{aux^{-1}} = \langle t \rangle^{aux^{-1}}.
\]

\item We know $s = \mathcal{E}[s'] = \mathcal{E}[q_{\zeta_r}[\sigma]] \longrightarrow'_{\langle prg \rangle} \mathcal{E}[t_{\zeta_r}[\sigma]] = t$. The desired reduction sequence can be given as follows:
\[
\langle \mathcal{E}[s'] \rangle^{aux^{-1}} = \langle \mathcal{E} \rangle^{aux^{-1}}[\langle s' \rangle^{aux^{-1}}] = \langle \mathcal{E} \rangle^{aux^{-1}}[\langle q_{\zeta_r} \rangle^{aux^{-1}}[\langle \sigma \rangle^{aux^{-1}}]] = \langle \mathcal{E} \rangle^{aux^{-1}}[q_r[\langle \sigma \rangle^{aux^{-1}}]]
\]
\[
\longrightarrow_{prg} \langle \mathcal{E} \rangle^{aux^{-1}}[t_r[\langle t_r \rangle^{aux^{-1}}]] = \langle \mathcal{E} \rangle^{aux^{-1}}[\langle t_r \rangle^{aux^{-1}}[\langle \sigma \rangle^{aux^{-1}}]] = \langle \mathcal{E} \rangle^{aux^{-1}}[\langle t_r[\sigma] \rangle^{aux^{-1}}]
\]
\[
= \langle \mathcal{E}[t_r[\sigma]] \rangle^{aux^{-1}} = \langle t \rangle^{aux^{-1}}.
\]

\item We know $s = \mathcal{E}[s'] = \mathcal{E}[q_\epsilon[\sigma]] \longrightarrow'_{\langle prg \rangle} \mathcal{E}[t_\epsilon[\sigma]] = t$. In this case, instead of giving a reduction sequence, the other side of the disjunction will be shown to hold.

Because $\langle q_\epsilon \rangle^{aux^{-1}} = \langle t_\epsilon \rangle^{aux^{-1}}$, it is $\langle s \rangle^{aux^{-1}} = \langle t \rangle^{aux^{-1}}$.

And because $q_\epsilon$ is of the function that is targeted in the transformation, and $t_\epsilon$ isn't, it is $m(q_\epsilon) > m(t_\epsilon)$, and consequently $m(s) > m(t)$.
\end{enumerate}
\end{proof}
\end{proposition}

\subsection{Using the back-interpretation directly}

This characterization of the bisimulation, equivalent to the first, directly uses the back-interpretation. In short, extractions preserve semantic properties by introducing an equation which leads from a, syntactically, more complex to a less complex term, where both terms are meant to represent, semantically, the same ``object''. This ``sameness'' is expressed by one being the back-interpretation of the other.

The bisimulation is characterized as follows:
\[
s {\longrightarrow}_{prg}^* t \iff s \longrightarrow^*_{\langle prg \rangle} \widetilde{t}, \text{ with } \langle \widetilde{t} \rangle^{aux^{-1}} = t
\]
In order to prove it, it suffices to show that this characterization is equivalent to the first characterization:
\begin{equation}
s {\longrightarrow'}_{\langle prg \rangle}^* t \iff s \longrightarrow^*_{\langle prg \rangle} \widetilde{t},
\end{equation}
for every $s, t$ with names declared in $prg$.

First, we prove the $`` \Rightarrow "$ direction of (2.2) and some lemmas used in this proof. In the following lemmas, all reductions are meant with respect to program $\langle prg \rangle$. Define $\longrightarrow^{aux} = \{(a,b) : a \longrightarrow b \land \langle a \rangle^{aux^{-1}} = \langle b \rangle^{aux^{-1}}\}$, where $\longrightarrow$ has a strategy which is allowed to choose any redex.

\begin{restatable}[Commuting-diamond property]{lemma}{cdpaux}

For all terms $a,b,c$ it holds that:
\[
a {\longrightarrow'} b \land a \longrightarrow^{aux} c \implies \exists d . b {\longrightarrow^{aux}}^* d \land c {\longrightarrow'}^= d
\]

\end{restatable}
\begin{proof}
Left for the appendix.
\end{proof}

\begin{corollary}

For all terms $a,b,c$ and $n \in \mathbb{N}$ it holds that:
\[
a {\longrightarrow'}^n b \land a {\longrightarrow^{aux}}^* c \implies \exists d . b {\longrightarrow^{aux}}^* d \land c {\longrightarrow'}^{\leq n} d
\]

\begin{proof}

By induction on $n$.

\end{proof}

\end{corollary}

\begin{lemma}

For a term $t$ with $\vdash'_v t$ and $\not\vdash_v t$, there is a term $t'$ and reductions
\[
t \longrightarrow t'
\]
and
\[
t \longrightarrow^{aux} t'.
\]

\begin{proof}

By induction on the structure of $t$. We distinguish two cases.
\begin{enumerate}
\item All immediate subterms of $t$ are values under judgement $\vdash_v$. We will show that $t$ matches $q_\epsilon$, and from this the desired reductions immediately follow. Because $\not\vdash_v t$, by the definition of the value judgement $t$ itself has to match some lhs $q$ of $\langle prg \rangle$. Since $\vdash'_v t$, there cannot be an equation $q'$ of $prg$ with $\langle q \rangle^{aux^{-1}} = q$. By the definition of extractions, the only equation of $\langle prg \rangle$ for which this can be the case is $q = q_\epsilon$. Consequently, we have some $\sigma$ that substitutes only with terms which are values under judgement $\vdash_v$ and the desired reductions
\[
t = q_\epsilon[\sigma] \longrightarrow t_\epsilon[\sigma], t = q_\epsilon[\sigma] \longrightarrow^{aux} t_\epsilon[\sigma].
\]

\item There are immediate subterms of $t$ which aren't values under judgement $\vdash_v$. Obtain a generic evaluation context $\mathcal{GE} \in \textrm{EC}_<$ by replacing all immediate subterms $t_i$ of $t$ with placeholders $p_i$, with $p_1, ..., p_n$ ordered according to the order $<$ of $\textrm{EC}_<$. Let $k$ be the smallest index for which $\not\vdash_v t_k$. We define $\mathcal{E} \in \textrm{EC}_<[\vdash_v]$ as the instance of $\mathcal{GE}$ which assigns, for all $i \neq k$, $t_i$ to placeholder $p_i$, and the hole $[]$ to placeholder $p_k$. By the induction hypothesis, we have a term $t'_k$ and reductions
\[
t_k \longrightarrow t'_k, t_k \longrightarrow^{aux} t'_k.
\]
Consequently, we have the desired term as $t' = \mathcal{E}[t'_k]$ and reductions
\[
t = \mathcal{E}[t_k] \longrightarrow \mathcal{E}[t'_k] = t', t = \mathcal{E}[t_k] \longrightarrow^{aux} \mathcal{E}[t'_k] = t'.
\]
\end{enumerate}

\end{proof}

\end{lemma}

\begin{lemma}

For all terms $a,b$ it holds that:

When $a \longrightarrow' b$ but $a \not\longrightarrow b$, then there is a $c$ with $a \longrightarrow c$ and $a \longrightarrow^{aux} c$.

\begin{proof}

There is an evaluation context $\mathcal{E} \in \textrm{EC}_<[\vdash'_v]$, with $\mathcal{E} \not\in \textrm{EC}_<[\vdash_v]$, and terms $a^0, b^0$, such that $a = \mathcal{E}[a^0], b = \mathcal{E}[b^0]$. Because $\mathcal{E} \not\in \textrm{EC}_<[\vdash_v]$, there has to be a subterm $t$ of $\mathcal{E}$ with $\not\vdash_v t$. Consider the generic evaluation context $\mathcal{GE} \in \textrm{EC}_<$ that $\mathcal{E}$ is an instance of, and the order $<$ on its placeholders. Construct an evaluation context $\mathcal{E}' \in \textrm{EC}_<[\vdash_v]$ from $\mathcal{E}$ by replacing the smallest placeholder of $\mathcal{GE}$, that $\mathcal{E}$ replaced with a term $t$ with $\not\vdash_v t$, by the hole $[]$. By Lemma 3.3.2, we have a term $t_{aux}$ and reductions
\[
t \longrightarrow t_{aux}, t \longrightarrow^{aux} t_{aux},
\]
and consequently there is a $c = \mathcal{E}'[t_{aux}]$ with
\[
\mathcal{E}'[t] \longrightarrow \mathcal{E}'[t_{aux}], \mathcal{E}'[t] \longrightarrow^{aux} \mathcal{E}'[t_{aux}].
\]

\end{proof}

\end{lemma}

\begin{lemma}[Left-to-right direction of (2.2)]
\[
s {\longrightarrow'}_{\langle prg \rangle}^* t \implies s \longrightarrow^*_{\langle prg \rangle} \widetilde{t}
\]

\begin{proof}

By induction on the length $n$ of $s {\longrightarrow'}_{\langle prg \rangle}^* t$.

\begin{itemize}
\item $n = 0$. Then it is $s = t$, thus simply choose $\widetilde{t} = t$.

\item $n = n' + 1$. By induction on the number $k$ of calls in $s$ to the function targeted in the transformation, i.e., that of $q_\epsilon$.
\begin{itemize}
\item $k = 0$. Consider the first step $s {\longrightarrow'}_{\langle prg \rangle} s_1$ of $s {\longrightarrow'}_{\langle prg \rangle}^* t$. Because there are no calls to the function of $q_\epsilon$ in $s$, for all subterms $s^0$ of $s$ it is $\vdash'_v s^0$ iff $\vdash_v s^0$. Consequently, the reduction step from $s$ to $s_1$ is also possible with reduction relation $\longrightarrow_{\langle prg \rangle}$, that is, $s \longrightarrow_{\langle prg \rangle} s_1$. By the outer induction hypothesis, we have the rest of the desired reduction sequence $s_1 \longrightarrow_{\langle prg \rangle} \widetilde{t}$.

\item $k = k' + 1$. Again, consider the first step $s {\longrightarrow'}_{\langle prg \rangle} s_1$ of the original sequence. We distinguish two cases.
\begin{enumerate}
\item $s \longrightarrow_{\langle prg \rangle} s_1$. With this, we have the first step of the desired reduction sequence. By the outer induction hypothesis, we have the rest of the desired reduction sequence $s_1 \longrightarrow_{\langle prg \rangle} \widetilde{t}$.

\item $s \not\longrightarrow_{\langle prg \rangle} s_1$. By Lemma 3.3.3, we have an $s_{aux}$ with $s \longrightarrow^{aux} s_{aux}$ and $s \longrightarrow_{\langle prg \rangle} s_{aux}$. By Corollary 3.3.1, we have a $\widetilde{t}'$ with (a) $s_{aux} \longrightarrow^{\leq n} \widetilde{t}'$ and (b) $t {\longrightarrow^{aux}}^* \widetilde{t}'$ and thus $\langle \widetilde{t}' \rangle^{aux^{-1}} = t$. Apply the inner induction hypothesis to $s_{aux} \longrightarrow^{\leq n} \widetilde{t}'$ to obtain the desired sequence.
\end{enumerate}
\end{itemize}
\end{itemize}

\end{proof}

\end{lemma}

To show that the $`` \Leftarrow ''$ direction of (2.2) holds, a translation for sequences $s \longrightarrow^*_{\langle prg \rangle} \widetilde{t}$ to sequences $s \longrightarrow^*_{prg} t$ is given (called Back-Translation). The correctness of the translation is verified, i.e., it is proven that each step of the sequences returned by the function is indeed a reduction step of the relevant reduction, and that the desired equality or equivalence between the end points of the original and the translated sequence holds. This correctness proofs are left for the appendix.

\begin{definition}[Back-Translation]
The Back-Translation is given by a conversion function $uc$ from sequences to sequences. It is defined recursively as follows.
\[
  uc((s_0, ..., s_n))=\begin{cases}
               (), &\text{ if } n = 0 \\
               uc((s_0, ..., s_{n-1})), &\text{ if } s_{n-1} \neq s_n \land \langle s_{n-1} \rangle^{aux^{-1}} = \langle s_n \rangle^{aux^{-1}} \\
               uc((s_0, ..., s_{n-1})) \cdot (\langle s_n \rangle^{aux^{-1}}), &\text{ otherwise} \\
            \end{cases}
\]

\end{definition}

\begin{restatable}[Correctness of the Back-Translation]{lemma}{correctb}
The Back-Translation is correct, that is, when
\[
s = s_1 {\longrightarrow}_{\langle prg \rangle} ... {\longrightarrow}_{\langle prg \rangle} s_n \text{ with } \langle s_n \rangle^{aux^{-1}} = t
\]
then
\[
s^{uc}_1 \longrightarrow_{prg} ... \longrightarrow_{prg} s^{uc}_m t
\]
for the sequence $(s^{uc}_1, ..., s^{uc}_m) := uc((s_1, ..., s_n))$.
\end{restatable}
\begin{proof}
Left for the appendix.
\end{proof}

\begin{lemma}[Right-to-left direction of (2.2)]
\[
 s \longrightarrow^*_{\langle prg \rangle} \widetilde{t} \implies s {\longrightarrow'}_{\langle prg \rangle}^* t
\]

\begin{proof}
``$\Rightarrow$'': 

``$\Leftarrow$'': Assume that there is a sequence of the form on the left-hand side. By the Back-Translation, there also is a sequence $s \longrightarrow_{prg} t$, and by (2.1) the desired sequence $s {\longrightarrow'}_{\langle prg \rangle} t$ obtains.

\end{proof}
\end{lemma}

Combining Lemma 2.3.3 and Lemma 2.3.5, statement (2.2) obtains.
\begin{proposition}
Statement (2.2), that is
\[
s {\longrightarrow'}_{\langle prg \rangle}^* t \iff s \longrightarrow^*_{\langle prg \rangle} \widetilde{t},
\]
for every $s, t$ with names declared in $prg$, holds.
\end{proposition}

\section{Absence of overlaps}

Since, in the context of this work, it is always presupposed that a program to be transformed has no overlapping copatterns, a desirable property of the transformed program is that it doesn't have overlapping copatterns, as well. This is now shown to be the case whenever $q_\epsilon$ doesn't overlap with any lhs of an equation taken over unchanged from $prg$.

\begin{proposition}
For any well-behaved extraction function lifted to programs, $\langle \cdot \rangle$, it is the case that if $prg$ has no overlapping copatterns and $q_\epsilon$ doesn't overlap with any lhs of an equation taken over unchanged from $prg$, then $\langle prg \rangle$ has no overlapping copatterns, as well.

\begin{proof}
First, the equations of the transformed are classified. There are three kinds of them: Those taken over unchanged over from $prg$ (indicated as $r$ in the table below), those which are an $\epsilon$ in a transformation result ($\epsilon$), and those which are, also in such a transformation result, a $\zeta_r$ ($\zeta$). The table below shows all possible combinations; its fields are filled with the number of the proof that lhss of equations of the respective kinds don't overlap.

\begin{tabular}{ l | c | c | r }  & r & $\epsilon$ & $\zeta$ \\ \hline r & (1) &  &  \\ \hline $\epsilon$ & (2) & (3) &  \\ \hline $\zeta$ & (4) & (5) & (6) \\ \hline \end{tabular}

\textit{ad} (1): Both equations are present in $prg$, thus their lhss don't overlap.

\textit{ad} (2): By assumption.

\textit{ad} (3): By the definition of the $q$-extraction, there is only one $\epsilon$-equation in the transformed program.

\textit{ad} (4): The $\zeta$-equation has a function name not declared in $prg$, unlike the $r$-equation.

\textit{ad} (5): The $\zeta$-equation has a function name not declared in $prg$, unlike the $\epsilon$-equation.

\textit{ad} (6): The lhss of each of the $\zeta$-equations are equivalent to a lhs in $prg$, thus, if they overlapped, so would these lhss in $prg$, contrary to fact.
\end{proof}
\end{proposition}

\section{Example extractions}

The two following extractions will be used in the next chapter.

\subsection{Destructor extraction}

The extraction $des\_extract(q)$ of a single destructor targeting some copattern $q$ can be defined as follows: It is the $\pi$-extraction targeting $q$ for the extraction projection $\pi$ defined below.

\[
\pi(`` fun(\overline{p}) ") = `` fun(\overline{p}) "
\]
\[
\pi(`` q.des(\overline{p}) ") = `` q "
\]

Since copatterns without destructors aren't affected, this extraction is only meant to be used for copatterns with at least one destructor. 

\subsection{Constructor extraction}

In this section, the extraction of a single constructor is defined. Unlike the extraction of a single destructor from the previous section, for constructor extraction it needs to be specified which constructor should be extracted. Define the extraction $con_n\_extract(q)$ of the $n$-th constructor in $q$ (from left to right) as the $\pi_n$-extraction targeting $q$ with the extraction projection $\pi_n$ defined below.

Let $x_q$ be a variable different from all in $q$.
\[
\pi_n(q) = q \text{ with the $n$-th constructor application } con(...) \text{ replaced by $x_q$ }
\]

\subsection{Extracting all patterns out of a single-destructor copattern}

This extraction extracts all patterns, that is, all constructor calls, out of a copattern with only a single destructor. Define this extraction $extract\_patterns$ as the $\pi$-extraction targeting $q$ with the extraction projection $\pi$ defined below.

\[
\pi(`` fun(\overline{p}).des(\overline{p'}) ") = `` fun(\overline{x}, \overline{x'}) "
\]