% !TEX root = main.tex
\chapter{Extraction transformations}

The automatic de- and refunctionalization algorithms presented in chapter 4 are made up of several ingredients, some of which are part of the motivation for the contents of this chapter. In the preprocessing phase of either algorithm, there are steps which eliminate destructors or constructors from copatterns. In this chapter, we generalize this concept to arbitrary \textit{extractions}.

Extractions are (a way to describe) transformations, that roughly speaking, decrease, in some way, the syntactic complexity of the program, while preserving its semantics to some degree. We think that this might also be interesting independent of its application for automatic de- and refunctionalization in chapter 4. For instance, the user might want to decrease the syntactic complexity, e.g., by reducing the maximum number of destructors appearing in any lhs, in order to better understand the program. This situation is comparable to certain refactorings in object-oriented programming, like moving methods or extracting classes.

As an example, consider the extraction of a destructor out of the following program fragment.
\begin{lstlisting}
fun().des().des() = t
\end{lstlisting}
The result is shown below; the transformation has introduced an auxiliary function, which is the reason why we chose the name ``extraction'': Some syntactic component is ``extracted'' into a newly added part of the program.
\begin{lstlisting}
fun().des() = aux()
aux().des() = t
\end{lstlisting}
The resulting program fragment doesn't contain any copatterns with two destructors, unlike the original, thus the transformation has decreased the syntactic complexity in this way. The transformation preserves the semantics because, roughly, in the transformed program there is a way to go from the original equation's lhs to its rhs: $\mathtt{fun().des().des()} \longrightarrow \mathtt{aux().des()} \longrightarrow \mathtt{t}$.

In the next section, we formally define extractions and how they are applied to programs (\autoref{sec:extrdef}). Then we show how extractions preserve the semantics of a program in a weak bisimulation (\autoref{sec:extrbis}). The preservation of the semantics depends upon the absence of overlapping equations both in the original and in the resulting program; therefore, the section that follows shows under which circumstances extractions don't introduce overlaps (\autoref{sec:extrovl}). Then we show that extractions preserve the well-typedness of programs (\autoref{sec:extrpwt}). Finally, we give some example extractions, two of which are used in chapter 4 (\autoref{sec:extrex}).

\section{Extractions}
\label{sec:extrdef}

Extractions are induced by a function $\pi$ that specifies how to decrease the syntactic complexity. Broadly, extractions fall into two classes: extractions that change the ``outer'' form, and those which change the ``inner'' form. Before defining extractions, we describe a way to uniformly express extractions that change the ``outer'' form, such as the destructor extraction exemplified above.

Changing the ``outer'' form of the copattern, i.e., removing or adding destructors, is dual to changing its ``inner'' form, i.e., replacing patterns with other patterns. For the purpose of defining extractions of the latter type, substitutions will be used. For the purpose of defining extractions of the former, we introduce the notion of \textit{co-substitution}.

\begin{definition}[Co-substitution]
A function $\sigma$ from copatterns to copatterns is called a \textit{co-substitution}, if, for every copattern $q$, it is defined as follows:
\[
\sigma(q) = q.\overline{des(\overline{p})},
\]
for some $\overline{des(\overline{p})}$ possibly depending on the $q$.
\end{definition}

Finally, extraction functions can be defined. More precisely, it is defined what it means to be an extraction projection, and for such an extraction projection $\pi$, a $\pi$-extraction targeting a set of equations $T$. For this, we make use of the notion of lenses, as defined by Foster et al.\cite{foster05combinators}.

\begin{definition}[Extraction projection]
A function $\pi$ from copatterns to copatterns is called an extraction projection if for every copattern $q$ there exists a (co-)substitution $\sigma^q_\pi$ such that $\sigma^q_\pi(\langle q \rangle^\pi) = q$.
\end{definition}

\begin{definition}[$\pi$-lens]
The $\pi$-lens, for an extraction projection $\pi$, is the lens defined as follows:
\[
\mathtt{get} = \pi
\]
\[
\mathtt{putback}(q^a, q^c) = \sigma^{q^c}_\pi(q^a),
\]
where the $\sigma^{q^c}_\pi$ is the (co)-substitution for $q^c$ and $\pi$ as given in the definition of the extraction projection.
\end{definition}

\begin{definition}[$\pi$-extraction target]
A $\pi$-extraction target $T$ is a set of equations such that $\pi(q) = \pi(q')$ for any two lhss $q, q'$ of equations in $T$.
\end{definition}

\begin{definition}[$\pi$-extraction targeting $T$]
A $\pi$-extraction targeting an extraction target $T$ is a triple consisting of
\begin{itemize}
\item an equation $\epsilon$ with lhs $q_\epsilon = \mathtt{get}(q)$ and rhs $t_\epsilon = \langle q_\epsilon \rangle^{aux}$, for some lhs $q$ of an equation in $T$,
\item a function $\zeta$ with domain $T$, defined as follows: $\zeta_r := ``\mathtt{putback}(t_\epsilon, q_r) = t_r  "$, and
\item the following signature for the auxiliary function $aux$ (its equations are the image of $\zeta$): $`` \textrm{\textbf{function }} aux(\tau_1, ..., \tau_n): \sigma "$, with $\tau_1, ..., \tau_n$ the types inferred, using the signatures of $prg$, for the variables in $q_\epsilon$, in the order that they appear in in $q_\epsilon$, and $\sigma$ the type inferred, also using the signatures of $prg$, for $q_\epsilon$.
\end{itemize}

Here $aux$ is a fresh function name\footnote{In practice this means that is in undeclared in the program that is transformed by the extraction lifted to programs (see next section).}. The pair of \texttt{get} and \texttt{putback} is the $\pi$-lens. $\langle \cdot \rangle^{aux}$ is the call to the auxiliary function that corresponds to the given copattern, defined as follows for copatterns $q$:
\[
\langle q \rangle^{aux} = aux(\langle q \rangle^{vars}),
\]
where $\langle q \rangle^{vars}$ denotes the list of all variables of $q$ in the order that they appear in in $q$.
\end{definition}

\subsection{Applying extractions to programs}

An extraction function was defined as a triple. As it will be used to transform programs into programs, however, it is necessary to define how it should be applied to programs. The definition of the apply function is straightforward; all it does is replace each targeted lhs in the program with $\epsilon$, leaving all other lhss unchanged, and collecting the $\zeta_r$ for each targeted equation $r$ in an auxiliary function definition.

Let $e = (\epsilon, \zeta, sig)$ be an extraction targeting $T$, with
\[
def_T = `` \textrm{\textbf{function }} fun(\tau_1, ..., \tau_n): \sigma \textrm{\textbf{ where }} eqns "
\]
the function definition that contains the lhss of $T$.

\begin{alignat*}{4}
\langle prg \rangle^{apply(e)} & = &&\{ && \textrm{\textbf{function }} fun(\tau_1, ..., \tau_n): \sigma \textrm{\textbf{ where }} \{ r ~ | ~ r \in eqns, r \not\in T \} \cup \{ \epsilon \}, \\
& && && \textrm{\textbf{function }} sig \textrm{\textbf{ where }} \{ \zeta_r ~ | ~ r \in T \} \} \\
& \cup && \{ def \in prg ~ | ~ def \neq def_T \} \span\span\span\span
\end{alignat*}

\section{Bisimulation}
\label{sec:extrbis}

Every extraction preserves the semantics of programs in a kind of weak bisimulation, assuming that the neither the original nor the transformed program have overlapping equations. There are two equivalent characterizations of this bisimulation. One is given and proved in the first subsection, the other follows in the second subsection and is shown to be equivalent to the first. For this section, we will always assume that neither the original nor the transformed program have overlapping equations.

Before we start, we define a function $\langle \cdot \rangle^{aux^{-1}}$ from terms to terms that serves as the opposite of $\langle \cdot \rangle^{aux}$. It is defined as replacing calls to the auxiliary function as generated by $\langle \cdot \rangle^{aux}$ with the original copatterns with their variables instantiated accordingly. This corresponds to what Setzer et al.\cite{setzer14unnesting} call a \textit{back-interpretation}; we will use this term here as well when referring to $\langle \cdot \rangle^{aux^{-1}}$.

\subsection{Using a modified value judgement}

For the definition of this weak bisimulation, we modify the value judgement, using the back-interpretation, in the following way. For any term $t$ with names declared in $\langle prg \rangle$, let
\[
\langle prg \rangle \vdash'_v t :\iff prg \vdash_v \langle t \rangle^{aux^{-1}}.
\]

From this definition it immediately follows that $\mathcal{E}$ is an evaluation context with respect to this modified value judgement for $\langle prg \rangle$ if and only if $\langle \mathcal{E} \rangle^{aux^{-1}}$ is an evaluation context with respect to the original value judgement for $prg$. It also means that, for a term $t$ with all names declared in $\langle prg \rangle$, the immediate subterms of $t$ are values with this judgement for $\langle prg \rangle$ if and only if the immediate subterms of $\langle t \rangle^{aux^{-1}}$ are values with the original judgement for $prg$.

Write $\longrightarrow'$ for the reduction relation $\longrightarrow$ with its value judgement modified in this way. The weak bisimulation is defined as follows, for every $s,t$ with all of their names declared in $prg$:
\begin{equation}
\label{eq:bisim1}
s \longrightarrow_{prg}^* t \iff s {\longrightarrow'}_{\langle prg \rangle}^* t
\end{equation}
This statement is now proved using Theorem 4 of Setzer et al.\cite{setzer14unnesting}

\begin{proposition}
\label{prop:bisim1}
The weak bisimulation statement~\ref{eq:bisim1} holds for any transformation defined as $apply(e)$ for some $\pi$-extraction $e$ targeting a $T$.

\begin{proof}
By Theorem 4 of Setzer et al.\cite{setzer14unnesting}, it suffices to show the statements (SN1) and (SN2), defined there along with the theorem. For this, set $\textrm{Int} = \langle \cdot \rangle^{aux^{-1}}$. Note that this unconversion is compatible with evaluation contexts, i.e.,
\[
\langle \mathcal{E}[s'] \rangle^{aux^{-1}} = \langle \mathcal{E} \rangle^{aux^{-1}}[\langle s' \rangle^{aux^{-1}}],
\]
because $\langle \cdot \rangle^{aux}$ converts to function calls, not destructor calls.
Further, set $m$ as the number of calls to the function targeted in the transformation, i.e., that of $q_\epsilon$.

(SN1): We know that $s = \mathcal{E}[s'] = \mathcal{E}[q_r[\sigma]] \longrightarrow_{prg} \mathcal{E}[t_r[\sigma]] = t$, for some evaluation context $\mathcal{E}$ of $prg$ and some equation $r$ of $prg$.
\begin{flalign*}
&s = \mathcal{E}[s'] = \mathcal{E}[\sigma^{q_r}_\pi(q_\epsilon)[\sigma]] \\
\longrightarrow'_{\langle prg \rangle} ~ &\mathcal{E}[\sigma^{q_r}_\pi(t_\epsilon)[\sigma]] = \mathcal{E}[q_{\zeta_r}[\sigma]] \\
\longrightarrow'_{\langle prg \rangle} ~ &\mathcal{E}[t_{\zeta_r}[\sigma]] = \mathcal{E}[t_r[\sigma]] = t
\end{flalign*}

(SN2): Three cases will be distinguished: The reduction in $\langle prg \rangle$ can use either an equation taken over unchanged from $prg$ (1.), it can use a $\zeta_r$ (2.), or it can use $\epsilon$ (3.). Each case makes use of an evaluation context $\mathcal{E}$ of the reduction relation for $\langle prg \rangle$.
\begin{enumerate}
\item We know $s = \mathcal{E}[s'] = \mathcal{E}[q_r[\sigma]] \longrightarrow'_{\langle prg \rangle} \mathcal{E}[t_r[\sigma]] = t$. The desired reduction sequence can be given as follows:
\begin{flalign*}
&\langle \mathcal{E}[s'] \rangle^{aux^{-1}} \\
=~& \langle \mathcal{E} \rangle^{aux^{-1}}[\langle s' \rangle^{aux^{-1}}] \\
=~& \langle \mathcal{E} \rangle^{aux^{-1}}[\langle q_r \rangle^{aux^{-1}}[\langle \sigma \rangle^{aux^{-1}}]] \\
=~& \langle \mathcal{E} \rangle^{aux^{-1}}[q_r[\langle \sigma \rangle^{aux^{-1}}]] \\
 \longrightarrow_{prg}~& \langle \mathcal{E} \rangle^{aux^{-1}}[t_r[\langle \sigma \rangle^{aux^{-1}}]] \\
=~& \langle \mathcal{E} \rangle^{aux^{-1}}[\langle t_r \rangle^{aux^{-1}}[\langle \sigma \rangle^{aux^{-1}}]] \\
=~& \langle \mathcal{E} \rangle^{aux^{-1}}[\langle t_r[\sigma] \rangle^{aux^{-1}}] \\
=~& \langle \mathcal{E}[t_r[\sigma]] \rangle^{aux^{-1}} \\
=~& \langle t \rangle^{aux^{-1}}.
\end{flalign*}

\item We know $s = \mathcal{E}[s'] = \mathcal{E}[q_{\zeta_r}[\sigma]] \longrightarrow'_{\langle prg \rangle} \mathcal{E}[t_{\zeta_r}[\sigma]] = t$. The desired reduction sequence can be given as follows:
\begin{flalign*}
&\langle \mathcal{E}[s'] \rangle^{aux^{-1}} \\
=~& \langle \mathcal{E} \rangle^{aux^{-1}}[\langle s' \rangle^{aux^{-1}}] \\
=~& \langle \mathcal{E} \rangle^{aux^{-1}}[\langle q_{\zeta_r} \rangle^{aux^{-1}}[\langle \sigma \rangle^{aux^{-1}}]] \\
=~& \langle \mathcal{E} \rangle^{aux^{-1}}[q_r[\langle \sigma \rangle^{aux^{-1}}]] \\
\longrightarrow_{prg}~& \langle \mathcal{E} \rangle^{aux^{-1}}[t_r[\langle t_r \rangle^{aux^{-1}}]] \\
=~& \langle \mathcal{E} \rangle^{aux^{-1}}[\langle t_r \rangle^{aux^{-1}}[\langle \sigma \rangle^{aux^{-1}}]] \\
=~& \langle \mathcal{E} \rangle^{aux^{-1}}[\langle t_r[\sigma] \rangle^{aux^{-1}}] \\
=~& \langle \mathcal{E}[t_r[\sigma]] \rangle^{aux^{-1}} = \langle t \rangle^{aux^{-1}}.
\end{flalign*}

\item We know $s = \mathcal{E}[s'] = \mathcal{E}[q_\epsilon[\sigma]] \longrightarrow'_{\langle prg \rangle} \mathcal{E}[t_\epsilon[\sigma]] = t$. In this case, instead of giving a reduction sequence, the other side of the disjunction will be shown to hold.

Because $\langle q_\epsilon \rangle^{aux^{-1}} = \langle t_\epsilon \rangle^{aux^{-1}}$, it is $\langle s \rangle^{aux^{-1}} = \langle t \rangle^{aux^{-1}}$.

And because $q_\epsilon$ is of the function that is targeted in the transformation, and $t_\epsilon$ isn't, it is $m(q_\epsilon) > m(t_\epsilon)$, and consequently $m(s) > m(t)$. \qedhere
\end{enumerate}
\end{proof}
\end{proposition}

\subsection{Using the back-interpretation directly}

This characterization of the bisimulation, which is equivalent to the first, directly uses the back-interpretation. In short, extractions preserve semantic properties by introducing an equation which leads from a, syntactically, more complex to a less complex term, where both terms are meant to represent, semantically, the same ``object''. This ``sameness'' is expressed by one being the back-interpretation of the other.

The bisimulation is characterized as follows:
\[
s {\longrightarrow}_{prg}^* t \iff s \longrightarrow^*_{\langle prg \rangle} \widetilde{t}, \text{ with } \langle \widetilde{t} \rangle^{aux^{-1}} = t
\]
In order to prove it, it suffices to show that this characterization is equivalent to the first characterization:
\begin{equation}
s {\longrightarrow'}_{\langle prg \rangle}^* t \iff s \longrightarrow^*_{\langle prg \rangle} \widetilde{t},
\end{equation}
for every $s, t$ with names declared in $prg$.

First, we show that the $`` \Rightarrow "$ direction of (2.2), as expressed in the following lemma, holds. In the following, all reductions are meant with respect to program $\langle prg \rangle$.

\begin{lemma}[$`` \Rightarrow "$ direction of (2.2)]
\label{lem:prop2lr}
\[
s {\longrightarrow'}_{\langle prg \rangle}^* t \implies s \longrightarrow^*_{\langle prg \rangle} \widetilde{t}
\]
\end{lemma}

In order to prove this, we define a counterpart to the $\longrightarrow'$ reduction relation: Let
\[
\longrightarrow^{aux} = \{(a,b) : a \longrightarrow^{all} b ~ \land ~ \langle a \rangle^{aux^{-1}} = \langle b \rangle^{aux^{-1}}\},
\]
where $\longrightarrow^{all}$ is the reduction relation for $\langle prg \rangle$ whose underlying value set is the set of all terms, i.e., the reduction is allowed to choose any redex. Where $\longrightarrow'$ can ``sidestep'' reductions which don't change the back-interpretation of a term, $\longrightarrow^{aux}$ is purely ``interpretative'', that is, it only allows reductions to terms with equal back-interpretation. This is expressed formally in the following lemmas relating $\longrightarrow'$ and $\longrightarrow^{aux}$, one concerning their commutation, the other what we call their complementarity. The $`` \Rightarrow "$ direction of (2.2) follows from these, as shown below.

\begin{restatable}[Commutation]{lemma}{cdpaux}
\label{lem:cdpaux}

For all terms $a,b,c$ it holds that:
\[
a {\longrightarrow'} b ~ \land ~ a \longrightarrow^{aux} c \implies \exists d . b {\longrightarrow^{aux}}^* d ~ \land ~ c {\longrightarrow'}^= d
\]

\end{restatable}
\begin{proof}
Left for the appendix.
\end{proof}

\begin{corollary}
\label{cor:cdpauxcor}

For all terms $a,b,c$ and $n \in \mathbb{N}$ it holds that:
\[
a {\longrightarrow'}^n b ~ \land ~ a {\longrightarrow^{aux}}^* c \implies \exists d . b {\longrightarrow^{aux}}^* d ~ \land ~ c {\longrightarrow'}^{\leq n} d
\]

\begin{proof}

By induction on $n$ and on the length of $a {\longrightarrow^{aux}}^* c$.

\end{proof}

\end{corollary}

\begin{restatable}[Complementarity]{lemma}{compl}
\label{lem:compl}

For all terms $a,b$ it holds that:

When $a \longrightarrow' b$ but $a \not\longrightarrow b$, then there is a $c$ with $a \longrightarrow c$ and $a \longrightarrow^{aux} c$.

\end{restatable}

\begin{proof}
Left for the appendix.
\end{proof}

\begin{proof}[Proof of \autoref{lem:prop2lr}]

By induction on the length $n$ of $s {\longrightarrow'}_{\langle prg \rangle}^* t$. For $n = 0$ it is $s = t$, thus simply choose $\widetilde{t} = t$. Now, consider the case that $n = n'+1$ for some $n' \geq 0$. We proceed by induction on the number $k$ of calls in $s$ to the function targeted in the transformation, i.e., that of $q_\epsilon$.

\begin{itemize}
\item $k = 0$. Consider the first step $s {\longrightarrow'}_{\langle prg \rangle} s_1$ of $s {\longrightarrow'}_{\langle prg \rangle}^* t$. Because there are no calls to the function of $q_\epsilon$ in $s$, for all subterms $s^0$ of $s$ it is $\vdash'_v s^0$ iff $\vdash_v s^0$. Consequently, the reduction step from $s$ to $s_1$ is also possible with reduction relation $\longrightarrow_{\langle prg \rangle}$, that is, $s \longrightarrow_{\langle prg \rangle} s_1$. By the outer induction hypothesis, we have the rest of the desired reduction sequence $s_1 \longrightarrow_{\langle prg \rangle} \widetilde{t}$.

\item $k = k' + 1$. Again, consider the first step $s {\longrightarrow'}_{\langle prg \rangle} s_1$ of the original sequence. We distinguish two cases.
\begin{enumerate}
\item $s \longrightarrow_{\langle prg \rangle} s_1$. With this, we have the first step of the desired reduction sequence. By the outer induction hypothesis, we have the rest of the desired reduction sequence $s_1 \longrightarrow_{\langle prg \rangle} \widetilde{t}$.

\item $s \not\longrightarrow_{\langle prg \rangle} s_1$. By \autoref{lem:compl}, we have an $s_{aux}$ with $s \longrightarrow^{aux} s_{aux}$ and $s \longrightarrow_{\langle prg \rangle} s_{aux}$. By \autoref{cor:cdpauxcor}, we have a $\widetilde{t}'$ with (a) $s_{aux} \longrightarrow^{\leq n} \widetilde{t}'$ and (b) $t {\longrightarrow^{aux}}^* \widetilde{t}'$ and thus $\langle \widetilde{t}' \rangle^{aux^{-1}} = t$. Apply the inner induction hypothesis to $s_{aux} \longrightarrow^{\leq n} \widetilde{t}'$ to obtain the desired sequence.
\end{enumerate}
\end{itemize}

\end{proof}

%%-- under construction
Now, we show the $`` \Leftarrow ''$ direction of (3.2).

\begin{lemma}[$`` \Leftarrow "$ direction of (3.2)]
\label{lem:prop2rl}
\[
s \longrightarrow^*_{\langle prg \rangle} \widetilde{t} \implies s {\longrightarrow'}_{\langle prg \rangle}^* t
\]
\end{lemma}

Again, we begin by showing a commutation lemma; but this time, we operate on the inverse relations of $\longrightarrow'$ and $\longrightarrow^{aux}$. The commutation of the inverses is different and non-standard, since it has an intermediate sequence of $\longrightarrow^{aux}$-steps, as shown in diagram TODO.

\begin{restatable}[Commutation (inverses)]{lemma}{comminv}
\label{lem:comminv}

For all terms $b,c,d$ it holds that:
\[
c {\longrightarrow'} d ~ \land ~ b \longrightarrow^{aux} d \implies \exists a, a' . a {\longrightarrow^{aux}}^* c ~ \land ~ a {\longrightarrow'}^= a' {\longrightarrow^{aux}}^*  b
\]

\end{restatable}
\begin{proof}
Left for the appendix.
\end{proof}

\begin{corollary}
\label{cor:comminvcor}

For all terms $b,c,d$ it holds that:
\[
c \longrightarrow' d ~ \land ~ b {\longrightarrow^{aux}}^* d \implies \exists a, a' . a {\longrightarrow^{aux}}^* c ~ \land ~ a {\longrightarrow'}^= a' {\longrightarrow^{aux}}^* b
\]

\begin{proof}

By induction on the length of $b {\longrightarrow^{aux}}^* d$.

\end{proof}
\end{corollary}

\begin{proof}[Proof of \autoref{lem:prop2rl}]

First, note that this direction is logically equivalent to the statement
\[
\forall s, t \in \textrm{Term}_{prg}, \widetilde{t} \in \textrm{Term}_{\langle prg \rangle}. (s \longrightarrow^* \widetilde{t} ~ \land ~ \langle \widetilde{t} \rangle^{aux^{-1}} = t) \implies s {\longrightarrow'}^* t.
\]
We prove this statement by induction on the length $n$ of $s \longrightarrow^* \widetilde{t}$. For $n = 0$ it is $s = \widetilde{t}$; since $s \in \textrm{Term}_{prg}$ it is $\langle s \rangle^{aux^{-1}} = s$ and it follows that $s = \langle \widetilde{t} \rangle^{aux^{-1}} = t$. 

Now, consider the case that $n = n'+1$ for some $n' \geq 0$. Since $\longrightarrow \subseteq \longrightarrow'$ it is $s_1 \longrightarrow' \widetilde{t}$; and since $t = \langle \widetilde{t} \rangle^{aux^{-1}}$ it is $t {\longrightarrow^{aux}}^* \widetilde{t}$. Thus, by \autoref{cor:comminvcor} we have terms $t', t''$ with $t' {\longrightarrow'}^= t'' {\longrightarrow^{aux}}^* t$ and $t' {\longrightarrow^{aux}}^* s_1$. Because $t \in \textrm{Term}_{prg}$ it must be $t'' = t$. By the induction hypothesis we have a reduction sequence $s {\longrightarrow'}^* \langle s_1 \rangle^{aux^{-1}}$. Since it is $t' {\longrightarrow'}^= t$, and $t$ contains no calls to $aux$, neither does $t'$. Thus, from $t' {\longrightarrow^{aux}}^* s_1$ it follows that $\langle s_1 \rangle^{aux^{-1}} = t'$. By combining the thus known sequences and equalities, we get the desired sequence $s {\longrightarrow'}^* \langle s_1 \rangle^{aux^{-1}} = t' {\longrightarrow'}^= t$.

\end{proof}

Combining \autoref{lem:prop2lr} and \autoref{lem:prop2rl}, statement (3.2) obtains.
\begin{proposition}
\label{prop:bisim2}
Statement (3.2), that is
\[
s {\longrightarrow'}_{\langle prg \rangle}^* t \iff s \longrightarrow^*_{\langle prg \rangle} \widetilde{t},
\]
for every $s, t$ with names declared in $prg$, holds.
\end{proposition}

\section{Absence of overlaps}
\label{sec:extrovl}

For the bisimulation, we presupposed that the transformed program has no overlapping copatterns. Here, we shown that this is the case whenever $q_\epsilon$ doesn't overlap with any lhs of an equation taken over unchanged from $prg$.

\begin{proposition}
For any well-behaved extraction function lifted to programs, $\langle \cdot \rangle$, it is the case that if $prg$ has no overlapping copatterns and $q_\epsilon$ doesn't overlap with any lhs of an equation taken over unchanged from $prg$, then $\langle prg \rangle$ has no overlapping copatterns, as well.

\begin{proof}
First, the equations of the transformed are classified. There are three kinds of them: Those taken over unchanged over from $prg$ (indicated as $r$ in the table below), those which are an $\epsilon$ in a transformation result ($\epsilon$), and those which are, also in such a transformation result, a $\zeta_r$ ($\zeta$). The table below shows all possible combinations; its fields are filled with the number of the proof that lhss of equations of the respective kinds don't overlap.

\begin{tabular}{ l | c | c | r }  & r & $\epsilon$ & $\zeta$ \\ \hline r & (1) &  &  \\ \hline $\epsilon$ & (2) & (3) &  \\ \hline $\zeta$ & (4) & (5) & (6) \\ \hline \end{tabular}

\textit{ad} (1): Both equations are present in $prg$, thus their lhss don't overlap.

\textit{ad} (2): By assumption.

\textit{ad} (3): By the definition of the $q$-extraction, there is only one $\epsilon$-equation in the transformed program.

\textit{ad} (4): The $\zeta$-equation has a function name not declared in $prg$, unlike the $r$-equation.

\textit{ad} (5): The $\zeta$-equation has a function name not declared in $prg$, unlike the $\epsilon$-equation.

\textit{ad} (6): The lhss of each of the $\zeta$-equations are equivalent to a lhs in $prg$, thus, if they overlapped, so would these lhss in $prg$, contrary to fact.
\end{proof}
\end{proposition}

\section{Preservation of well-typedness}
\label{sec:extrpwt}

In this section, we show that extractions preserve well-typedness, i.e., that when applying an extraction to a well-typed program a well-typed program obtains. Let $prg$ be the original program, and $\langle prg \rangle$ the program after extraction. By definition a program is well-typed when all of its equations $eqn$ are well-typed, i.e., it is $\Sigma \vdash eqn \textrm{ ok}$, where $\Sigma$ are the signatures of the program. Let $\Sigma_{prg}$ be the signatures of $prg$, and $\Sigma_{\langle prg \rangle}$ be the signatures of $\langle prg \rangle$. Since extractions only add one function signature, but leave the other signatures unchanged, it is $\Sigma_{prg} \subseteq \Sigma_{\langle prg \rangle}$. The following thus holds trivially.

\begin{fact}
\label{fac:wtpfac}
All typing judgements still hold when replacing $\Sigma_{prg}$ with $\Sigma_{\langle prg \rangle}$.
\end{fact}

Like in \autoref{sec:extrovl}, we distinguish three kinds of equations in $\langle prg \rangle$: those left unchanged, the $\epsilon$ equation, and the $\zeta_r$ equations. For each of these kinds, we show a lemma stating that all of its equations are well-typed, under the assumption that $prg$  is well-typed (and thus all its equations). Combining these, the preservation of well-typedness follows.

\begin{lemma}
Assuming that $prg$ is well-typed, all equations $eqn$ of $\langle prg \rangle$ taken over unchanged from $prg$ are well-typed.

\begin{proof}
By assumption, we know that $\Sigma_{prg} \vdash eqn \textrm{ ok}$. By \autoref{fac:wtpfac} it follows that $\Sigma_{\langle prg \rangle} \vdash eqn \textrm{ ok}$.
\end{proof}
\end{lemma}

\begin{lemma}
Assuming that $prg$ is well-typed, the equation $\epsilon$ of $\langle prg \rangle$, as specified in the definition of extractions, is well-typed.

\begin{proof}
We want to show that $\Sigma \vdash `` q_\epsilon = t_\epsilon " \textrm{ ok}$. By the judgement for well-typedness of equations this means that we have to show the following.
\begin{enumerate}
\item $q_\epsilon$ is either a function or destructor call, with the respective function or destructor signature present in $\Sigma_{\langle prg \rangle}$; let the return type given in this signature be $\sigma$.

\item For each variable $x$ in $q_\epsilon$ it is $q_\epsilon \vdash_{\Sigma_{\langle prg \rangle}} x : \sigma_x$ for some type $\sigma_x$.

\item It is $\{ x : \sigma_x ~ | ~ x \in V(q_\epsilon) \} \vdash_{\Sigma_{\langle prg \rangle}} t_\epsilon : \sigma$.
\end{enumerate}
The first point is easy to see. Since there is at least one left-hand side $q \in T$ (part of the extraction target) of $prg$, that we know is the result of applying a (co-)substitution to $q_\epsilon$, and $prg$ is well-typed, there must be a signature in $\Sigma_{prg}$ which holds the return type $\sigma$ for the destructor or function call that $q_\epsilon$ is. Consequently, since $\Sigma_{prg} \subseteq \Sigma_{\langle prg \rangle}$, we also have the return type $\sigma$ for the destructor or function call $q_\epsilon$ in the signature set $\Sigma_{\langle prg \rangle}$ of $\langle prg \rangle$.

For the second point, again consider an lhs $q \in T$. Because it is possible to infer, using $\Sigma_{prg}$, a type for all variables in $q$, which results from applying a (co-)substitution to $q_\epsilon$, it also is possible to infer a type for all variables in $q_\epsilon$ using $\Sigma_{prg}$, and, by \autoref{fac:wtpfac}, also using $\Sigma_{\langle prg \rangle}$.

The third point follows immediately from the definition of the signature of $aux$, as it is $t_\epsilon = aux(x_1, ..., x_n)$, with the $x_i$ being the variables of $q_\epsilon$, and the argument types of $aux$ are defined to be the types inferred for the variables of $q_\epsilon$, and the return type of $aux$ is defined to be the type $\sigma$ inferred for $q_\epsilon$.
\end{proof}
\end{lemma}

\begin{lemma}
Assuming that $prg$ is well-typed, the equations $\zeta_r$, as specified in the definition of extractions, are well-typed.

\begin{proof}
First, note that by definition $aux$ is the function definition that $\zeta_r$ is a part of. We want to show that $\Sigma \vdash `` q_{\zeta_r} = t_{\zeta_r} " \textrm{ ok}$. By the judgement for well-typedness of equations this means that we have to show the following.
\begin{enumerate}
\item $q_{\zeta_r}$ is either a function or destructor call, with the respective function or destructor signature present in $\Sigma_{\langle prg \rangle}$; let the return type given in this signature be $\sigma$.

\item For each variable $x$ in $q_{\zeta_r}$ it is $q_{\zeta_r} \vdash_{\Sigma_{\langle prg \rangle}} x : \sigma_x$ for some type $\sigma_x$.

\item It is $\{ x : \sigma_x ~ | ~ x \in V(q_{\zeta_r}) \} \vdash_{\Sigma_{\langle prg \rangle}} t_{\zeta_r} : \sigma$.
\end{enumerate}
For the first point, we distinguish between the possible forms of $q_{\zeta_r}$. It can either be a function call, then we infer its type as the return type $\sigma$ of $aux$, given in the signature for $aux$, which is present in $\Sigma_{\langle prg \rangle}$. Or it can be a destructor call; by definition, $q_{\zeta_r}$ is the result of a (co-)substitution applied to $t_\epsilon$, and since $q_{\zeta_r}$ contains a destructor call and $t_\epsilon$ doesn't, the destructor must be a part of the co-substitution. Since $q_r \in T$ results from $q_\epsilon$ by applying this co-substitution to it, and $prg$ is well-typed, we know that the respective destructor signature is present in $\Sigma_{prg}$, and thus in $\Sigma_{\langle prg \rangle}$.

Concerning the second point, first note that the variables of $q_{\zeta_r}$ are exactly those of $q_r$. The types inferred for the variables in $q_{\zeta_r}$ are identical to the types inferred for them in $q_r$. To see this, combine the facts that the argument types of $aux$ are defined to be the types inferred for the variables of $q_\epsilon$, the return type of $aux$ is defined to be the type inferred for $q_\epsilon$, and that $q_r$ is the result of applying a (co-)substitution to $q_\epsilon$. Thus we know the following. For each variable $x$ of $q_r$ and $q_{\zeta_r}$, let $\sigma_x$ be the type inferred for it in $prg$, i.e., it is $q_r \vdash_{\Sigma_{prg}} x : \sigma_x$. Since the types inferred for the variables in $q_{\zeta_r}$ are identical to the types inferred for them in $q_r$, it also is $q_{\zeta_r} \vdash_{\Sigma_{\langle prg \rangle}} x : \sigma_x$.

For the third point, we use the fact that the types inferred for the variables of $q_{\zeta_r}$ and $q_r$ are identical in $prg$ and $\langle prg \rangle$, as established for the second point. Since $prg$ is well-typed, and thus it is $\Sigma_{prg} \vdash `` q_r = t_r " \textrm{ ok}$, it also is $\{ x : \sigma_x ~ | ~ x \in V(q_r) \} \vdash_{\Sigma_{prg}} t_r$, and by \autoref{fac:wtpfac} it is $\{ x : \sigma_x ~ | ~ x \in V(q_r) \} \vdash_{\Sigma_{\langle prg \rangle}} t_r$.
\end{proof}
\end{lemma}

\section{Example extractions}
\label{sec:extrex}

The two following extractions will be used in the next chapter.

\subsection{Destructor extraction}

The extraction $des\_extract(q)$ of a single destructor targeting some copattern $q$ can be defined as follows: It is the $\pi$-extraction (targeting some $T$) for the extraction projection $\pi$ defined below.

\[
\pi(`` fun(\overline{p}) ") = `` fun(\overline{p}) "
\]
\[
\pi(`` q.des(\overline{p}) ") = `` q "
\]

Since copatterns without destructors aren't affected, this extraction is only meant to be used for copatterns with at least one destructor. We now show that $\pi$ is an extraction projection, by giving a cosubstitution $\sigma^q_\pi$ for each $q$ such that $\sigma^q_\pi(\pi(q)) = q$: Set
\[
\sigma^q_\pi(q') = \begin{cases}
                              q'.des(\overline{p}) &,\text{ if } q = q''.des(\overline{p}) \\
                              q' &,\text{ otherwise}
                              \end{cases}
\]

\subsection{Constructor extraction}

We define a family $\textsf{ExtractCon}(p)$ of extractions of single constructors. The parameter $p$ stands for the position of the constructor call to be extracted in the targeted copatterns. For such a position $p$, $\textsf{ExtractCon}(p)$ is defined as the $\pi$-extraction (targeting some $T$) for the extraction projection $\pi$ defined below.

For a copattern $q$, let $x_q$ be a variable different from all in $q$.
\[
\pi(q) = \begin{cases}
             q[con(\overline{p}) \mapsto x_q]_p &\text{ if } q \text{ has a constructor call at position } p \\
             q &\text{ otherwise}
             \end{cases}
\]

We now show why $\pi$ is an extraction projection, that is, we give, for each $q$, a substitution $\sigma^q_\pi$ such that $\sigma^q_\pi(\pi(q)) = q$: Set $\sigma^q_\pi := \{x_q \mapsto \overline{p}\}$.

\subsection{Extracting all patterns out of a single-destructor copattern}

This extraction extracts all patterns, that is, all constructor calls, out of a copattern with only a single destructor. Define this extraction $extract\_patterns$ as the $\pi$-extraction targeting $q$ with the extraction projection $\pi$ defined below.

\[
\pi(`` fun(\overline{p}).des(\overline{p'}) ") = `` fun(\overline{x}, \overline{x'}) "
\]