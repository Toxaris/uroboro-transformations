% -*- Mode:TeX -*-

%% IMPORTANT: The official thesis specifications are available at:
%%            http://libraries.mit.edu/archives/thesis-specs/
%%
%%            Please verify your thesis' formatting and copyright
%%            assignment before submission.  If you notice any
%%            discrepancies between these templates and the 
%%            MIT Libraries' specs, please let us know
%%            by e-mailing thesis@mit.edu

%% The documentclass options along with the pagestyle can be used to generate
%% a technical report, a draft copy, or a regular thesis.  You may need to
%% re-specify the pagestyle after you \include  cover.tex.  For more
%% information, see the first few lines of mitthesis.cls. 

%\documentclass[12pt,vi,twoside]{mitthesis}
%%
%%  If you want your thesis copyright to you instead of MIT, use the
%%  ``vi'' option, as above.
%%
%\documentclass[12pt,twoside,leftblank]{mitthesis}
%%
%% If you want blank pages before new chapters to be labelled ``This
%% Page Intentionally Left Blank'', use the ``leftblank'' option, as
%% above. 

\documentclass[12pt,twoside]{mitthesis}
\usepackage{lgrind}
%% These have been added at the request of the MIT Libraries, because
%% some PDF conversions mess up the ligatures.  -LB, 1/22/2014
\usepackage{cmap}
\usepackage[T1]{fontenc}
\pagestyle{drafthead}

\usepackage[utf8]{inputenc}

\usepackage{amsmath}
\usepackage{amssymb}
\usepackage{amsthm}
\usepackage{bussproofs}
\usepackage{framed}
\usepackage{listings}
\usepackage{fixltx2e}
\usepackage{thmtools, thm-restate}
\usepackage{hyperref}
\usepackage{csquotes}

\usepackage{graphicx}
\usepackage[labelfont=bf]{caption}
\usepackage{subcaption}

\usepackage{float}
%\floatstyle{boxed} 
\restylefloat{figure}

\usepackage{courier}

\lstset{basicstyle=\footnotesize\ttfamily,breaklines=true}
\lstset{framextopmargin=50pt,frame=bottomline}

\usepackage{tikz}

\newtheorem{theorem}{Theorem}[section]
\newtheorem{proposition}{Proposition}[section]
\newtheorem{lemma}{Lemma}[section]
\newtheorem{fact}{Fact}[section]
\newtheorem{corollary}{Corollary}[section]
\declaretheorem[style=definition,name=Definition,numberwithin=section,qed=$\Diamond$]{definition}
\declaretheorem[style=definition,name=Remark,numberwithin=section]{remark}
\declaretheorem[style=definition,name=Algorithm,qed=$\Diamond$]{algorithm}

\setcounter{secnumdepth}{3}

%% This bit allows you to either specify only the files which you wish to
%% process, or `all' to process all files which you \include.
%% Krishna Sethuraman (1990).

%\typein [\files]{Enter file names to process, (chap1,chap2 ...), or `all' to
%process all files:}
%\def\all{all}
%\ifx\files\all \typeout{Including all files.} \else \typeout{Including only \files.} \includeonly{\files} \fi

% Autoref configuration (by Tillmann)
% =====================

\def\sectionautorefname{Section}
\let\subsectionautorefname=\sectionautorefname
\let\subsubsectionautorefname=\sectionautorefname % added by me
\def\lemmaautorefname{Lemma}

\let\oldautoref=\autoref
\renewcommand{\autoref}[1]
  {\mbox{\oldautoref{#1}}}


\def\chapterautorefname{Chapter}

\begin{document}

% -*-latex-*-
% 
% For questions, comments, concerns or complaints:
% thesis@mit.edu
% 
%
% $Log: cover.tex,v $
% Revision 1.8  2008/05/13 15:02:15  jdreed
% Degree month is June, not May.  Added note about prevdegrees.
% Arthur Smith's title updated
%
% Revision 1.7  2001/02/08 18:53:16  boojum
% changed some \newpages to \cleardoublepages
%
% Revision 1.6  1999/10/21 14:49:31  boojum
% changed comment referring to documentstyle
%
% Revision 1.5  1999/10/21 14:39:04  boojum
% *** empty log message ***
%
% Revision 1.4  1997/04/18  17:54:10  othomas
% added page numbers on abstract and cover, and made 1 abstract
% page the default rather than 2.  (anne hunter tells me this
% is the new institute standard.)
%
% Revision 1.4  1997/04/18  17:54:10  othomas
% added page numbers on abstract and cover, and made 1 abstract
% page the default rather than 2.  (anne hunter tells me this
% is the new institute standard.)
%
% Revision 1.3  93/05/17  17:06:29  starflt
% Added acknowledgements section (suggested by tompalka)
% 
% Revision 1.2  92/04/22  13:13:13  epeisach
% Fixes for 1991 course 6 requirements
% Phrase "and to grant others the right to do so" has been added to 
% permission clause
% Second copy of abstract is not counted as separate pages so numbering works
% out
% 
% Revision 1.1  92/04/22  13:08:20  epeisach

% NOTE:
% These templates make an effort to conform to the MIT Thesis specifications,
% however the specifications can change.  We recommend that you verify the
% layout of your title page with your thesis advisor and/or the MIT 
% Libraries before printing your final copy.
\title{Automatic Program Transformations for a Language with Copattern Matching}

\author{Julian Jabs}
% If you wish to list your previous degrees on the cover page, use the 
% previous degrees command:
%       \prevdegrees{A.A., Harvard University (1985)}
% You can use the \\ command to list multiple previous degrees
%       \prevdegrees{B.S., University of California (1978) \\
%                    S.M., Massachusetts Institute of Technology (1981)}
\department{Mathematisch-Naturwissenschaftliche Fakultät}

% If the thesis is for two degrees simultaneously, list them both
% separated by \and like this:
% \degree{Doctor of Philosophy \and Master of Science}
\degree{Master of Science in Computer Science}

% As of the 2007-08 academic year, valid degree months are September, 
% February, or June.  The default is June.
\degreemonth{September}
\degreeyear{2015}
\thesisdate{September 30, 2015}

%% By default, the thesis will be copyrighted to MIT.  If you need to copyright
%% the thesis to yourself, just specify the `vi' documentclass option.  If for
%% some reason you want to exactly specify the copyright notice text, you can
%% use the \copyrightnoticetext command.  
%\copyrightnoticetext{\copyright IBM, 1990.  Do not open till Xmas.}

% If there is more than one supervisor, use the \supervisor command
% once for each.
\supervisor{Prof. Dr.-Ing. Klaus Ostermann}
\secondexaminer{Prof. Dr. Torsten Grust}

% This is the department committee chairman, not the thesis committee
% chairman.  You should replace this with your Department's Committee
% Chairman.
%\chairman{Arthur C. Smith}{Chairman, Department Committee on Graduate Theses}

% Make the titlepage based on the above information.  If you need
% something special and can't use the standard form, you can specify
% the exact text of the titlepage yourself.  Put it in a titlepage
% environment and leave blank lines where you want vertical space.
% The spaces will be adjusted to fill the entire page.  The dotted
% lines for the signatures are made with the \signature command.
\maketitle

% The abstractpage environment sets up everything on the page except
% the text itself.  The title and other header material are put at the
% top of the page, and the supervisors are listed at the bottom.  A
% new page is begun both before and after.  Of course, an abstract may
% be more than one page itself.  If you need more control over the
% format of the page, you can use the abstract environment, which puts
% the word "Abstract" at the beginning and single spaces its text.

%% You can either \input (*not* \include) your abstract file, or you can put
%% the text of the abstract directly between the \begin{abstractpage} and
%% \end{abstractpage} commands.

% First copy: start a new page, and save the page number.
\cleardoublepage
% Uncomment the next line if you do NOT want a page number on your
% abstract and acknowledgments pages.
% \pagestyle{empty}
\setcounter{savepage}{\thepage}
\begin{abstractpage}
% $Log: abstract.tex,v $
% Revision 1.1  93/05/14  14:56:25  starflt
% Initial revision
% 
% Revision 1.1  90/05/04  10:41:01  lwvanels
% Initial revision
% 
%
%% The text of your abstract and nothing else (other than comments) goes here.
%% It will be single-spaced and the rest of the text that is supposed to go on
%% the abstract page will be generated by the abstractpage environment.  This
%% file should be \input (not \include 'd) from cover.tex.
Rendel et al. present two functional programming languages on which Reynolds' defunctionalization and Danvy's refunctionalization, respectively, are totally defined. These languages, called the Codata and the Data Fragment, support, respectively, Abel's copattern matching and its dual, the usual pattern matching of functional languages. Rendel et al. intend these languages to be fragments of a common language, called Uroboro, that has both pattern and copattern matching. In this thesis, we formally define Uroboro and we develop automatic program transformations for this language. We extend Rendel et al.'s automatic de- and refunctionalization for the Codata and Data Fragments to all of Uroboro. We also identify a generalization, which we call extraction, of some of the steps which make up these transformations. As a by-product of our work, we shine some light on an asymmetry between pattern and copattern matching.

\end{abstractpage}

% Additional copy: start a new page, and reset the page number.  This way,
% the second copy of the abstract is not counted as separate pages.
% Uncomment the next 6 lines if you need two copies of the abstract
% page.
% \setcounter{page}{\thesavepage}
% \begin{abstractpage}
% % $Log: abstract.tex,v $
% Revision 1.1  93/05/14  14:56:25  starflt
% Initial revision
% 
% Revision 1.1  90/05/04  10:41:01  lwvanels
% Initial revision
% 
%
%% The text of your abstract and nothing else (other than comments) goes here.
%% It will be single-spaced and the rest of the text that is supposed to go on
%% the abstract page will be generated by the abstractpage environment.  This
%% file should be \input (not \include 'd) from cover.tex.
Rendel et al. present two functional programming languages on which Reynolds' defunctionalization and Danvy's refunctionalization, respectively, are totally defined. These languages, called the Codata and the Data Fragment, support, respectively, Abel's copattern matching and its dual, the usual pattern matching of functional languages. Rendel et al. intend these languages to be fragments of a common language, called Uroboro, that has both pattern and copattern matching. In this thesis, we formally define Uroboro and we develop automatic program transformations for this language. We extend Rendel et al.'s automatic de- and refunctionalization for the Codata and Data Fragments to all of Uroboro. We also identify a generalization, which we call extraction, of some of the steps which make up these transformations. As a by-product of our work, we shine some light on an asymmetry between pattern and copattern matching.

% \end{abstractpage}

\cleardoublepage

\section*{Acknowledgments}

This is the acknowledgements section.  You should replace this with your
own acknowledgements.

%%%%%%%%%%%%%%%%%%%%%%%%%%%%%%%%%%%%%%%%%%%%%%%%%%%%%%%%%%%%%%%%%%%%%%
% -*-latex-*-

% Some departments (e.g. 5) require an additional signature page.  See
% signature.tex for more information and uncomment the following line if
% applicable.
% \include{signature}
\pagestyle{plain}
  % -*- Mode:TeX -*-
%% This file simply contains the commands that actually generate the table of
%% contents and lists of figures and tables.  You can omit any or all of
%% these files by simply taking out the appropriate command.  For more
%% information on these files, see appendix C.3.3 of the LaTeX manual. 
\tableofcontents
\newpage
\listoffigures
%\newpage
%\listoftables


\chapter{Introduction}
\label{ch:intro}

Functional programming seeks to bring the benefits of mathematical functions over to the world of programming. In mathematics, a function's return value can only depend on the input, not on when -- or more generally, under which ``circumstances'' -- the function was called. This is in sharp contrast to imperative programming languages, which have a concept of \textit{mutable} state. For instance, a variable might be assigned different values at different times during the execution of a program, and a function might depend upon the value of this variable. A purely functional programming language doesn't allow the programmer to work with mutable state, which means that functions defined in such a language can be understood in the same way a mathematical function can. This clarity makes (purely) functional languages potentially less error-prone than imperative languages, since the programmer doesn't need to consider the mutable state, e.g., of variables.

Many functional programming languages, in particular, languages with static type-checking, have the language construct of \textit{data type definitions}. When type-checking is performed statically, i.e., before the execution of a program, the type-checker must be able to assign a type to each expression that appears in the program. Types can be regarded as sets of values; looking at it the other way around, values which are of a given type are said to \textit{inhabit} the type. With data type definitions, the programmer can specify -- or rather postulate -- which values are to inhabit the thus defined type. This is done by giving a number of \textit{constructors} for the type. For instance, a type for natural numbers can be defined by giving a constructor for the zero value, and one for successors, which takes an argument -- the predecessor -- which is itself a natural number.

In functional programming, a function definition often consists of a number of equations. This is strongly analogous to mathematics; a left-hand side (lhs) of such an equation specifies which inputs are to be considered, and the right-hand side specifies the corresponding output. Thus, there needs to be a way to introspect inputs. This brings us back to the data type definitions and their constructors. On the lhs of some equation for, e.g., a function $fun(\mathbb{N})$ with a natural number as input, the programmer can specify that he wishes the equation to define the output corresponding to, e.g., the zero value, or a successor, by using the respective constructor of the data type for $\mathbb{N}$, possibly together with variables or further constructors for its arguments. In symbols, for constructors $zero()$ and $succ(\mathbb{N})$, the lhs that captures the zero value input of $fun$ is $fun(zero())$, and the one that captures the input of a successor, with the predecessor argument left unspecified, it $fun(succ(x))$, where $x$ is a variable. This concept is called \textit{pattern matching}, and the constructors and variables used in this way to introspect the input are called \textit{patterns}. For many concepts, it is interesting to also explore their \textit{dual}. The dual of patterns are the so-called copatterns. Basically, they allow to introspect, or observe, the output of a function; more on this later.

Another important feature of many functional programming languages is that they have \textit{first-class functions}. This is to say that functions can be used just like other objects, e.g., a function can be passed as an argument to another function. This can be very useful, e.g., to define filters or maps over collections. A language with first-class functions is also called \textit{higher-order}, in contrast to languages lacking this feature, which are called \textit{first-order}.

Sometimes, a first-order specification of some function can be beneficial over an equivalent higher-order specification, and sometimes it is the other way around. This can depend upon whether human understandability or a certain property advantageous for machine processing is desired. For this reason, researchers are interested in finding and automatizing transformations which can bring a higher-order program into a semantically equivalent first-order form, or the other way around. Transforming a higher-order program (with first-class functions) into an equivalent first-order program (without first-class functions) is called defunctionalization, the opposite direction is called refunctionalization.

In this work, we develop automatic program transformations for a specific language with copattern matching, called \textit{Uroboro}. One major goal is to support automatic de- and refunctionalization for all of Uroboro. Uses for these kinds of transformations have been presented by Reynolds and Danvy, among others. Uroboro is interesting for the purpose of automatizing such transformations, because, as Rendel et al.\cite{rendel15automatic} have shown, copatterns and refunctionalization are related. Rendel et al. have already developed automatic de- and refunctionalization between two fragments of Uroboro, called the Data and Codata Fragment. Based on their work, we contribute the following.
\begin{itemize}
\item We formally define the full language Uroboro, which is an extension of both the Data and the Codata Fragment.

\item We have developed and/or adapted the necessary transformation steps, such that one can now automatically de- and refunctionalize all of Uroboro; these transformations preserve the semantics of programs -- see the next point.

\item For some of the steps, we have identified a generalization; we call such transformations \textit{extractions}. We also show how extractions preserve the semantics of programs; we essentially use this result when proving that our automatic de- and refunctionalization preserve semantics.

\item When developing automatic refunctionalization, we have encountered a complication which doesn't have an analogue in defunctionalization; we have identified the general problem underlying this, shining a light on an asymmetry between patterns and copatterns.
\end{itemize}
We intend the transformations of this work to be a foundation for an entire toolbox of automatic transformations for Uroboro.

In this introduction, we first give an informal overview of de- and refunctionalization. Next, we briefly describe copatterns. We then outline how Uroboro relates these two concepts and why we consider this language interesting. Finally, we sum up the contents of this thesis.

\textit{Defunctionalization} is a technique to transform higher-order programs into semantically equivalent first-order programs. It was first described by Reynolds\cite{reynolds72definitional}, who uses it as one of several tools to compile interpreters. We illustrate defunctionalization with the following example program, presented in a Haskell-like pseudocode.

\begin{lstlisting}

filterNats :: ((Nat -> Bool), [Nat]) -> [Nat]
filterNats (f, x:xs)
  | f x = x:(filterNats (f, xs))
  | otherwise = filterNats (f, xs)
filterNats (_, []) = []

even :: Nat -> Bool
even Zero = True
even Succ(n) = not (even n)

main :: [Nat]
main = filterNats (even, [1, 2, 3, 4, 5])

\end{lstlisting}

This program is higher-order because the first argument of \texttt{filterNats} has a function type (\texttt{Nat -> Bool}). Defunctionalization does the following:
\begin{enumerate}
\item  A data type \texttt{NatBoolFun} for this function type is introduced, with a constructor \texttt{Even} for \texttt{even}.
\item A function \texttt{apply :: (NatBoolFun, Nat) -> Bool} is introduced, which has an equation for every equation in \texttt{even}. The left-hand side \texttt{even $p$}, where $p$ is some pattern, of an equation of \texttt{even} is replaced by \texttt{apply (Even, $p$)}, and the respective right-hand sides are identical (preliminarily).
\item Each call to \texttt{even} is replaced by the corresponding call to \texttt{apply}, and each occurrence of \texttt{even} in an argument position is replaced by \texttt{Even}.
\item The function definition for \texttt{even} is removed.
\end{enumerate}
The result is presented below.

\begin{lstlisting}

data NatBoolFun = Even

apply :: (NatBoolFun, Nat) -> Bool
apply (Even, Zero) = True
apply (Even, Succ(n)) = not (apply (Even, n))

filterNats :: (NatBoolFun, [Nat]) -> [Nat]
filterNats (f, x:xs)
  | f x = x:(filterNats (f, xs))
  | otherwise = filterNats (f, xs)
filterNats (_, []) = []

main :: [Nat]
main = filterNats (Even, [1, 2, 3, 4, 5])

\end{lstlisting}

This transformation is easily automatized. In general, the procedure described above is applied to all occurrences of function types in the program.

\textit{Refunctionalization} is the left inverse of defunctionalization, first described by Danvy and Millikin\cite{danvy09refunctionalization}. Its goal is to undo defunctionalization. When one knows where to find the function that serves the role of \texttt{apply} in the example above, this is easily done by just reverting the steps described above. However, in general there can be multiple functions which have the correct form for this, and could all possibly be \texttt{apply} functions. In its originally described form, refunctionalization therefore only works on programs which are the result of the defunctionalization of some other program.

However, it would be desirable to have a transformation that transforms \textit{all} first-order programs into semantically equivalent higher-order programs, like defunctionalization does it the other way around. This is where another concept comes into play: copatterns and codata.

\textit{Copatterns} are the dual to patterns, in the following sense. Where patterns allow one to distinguish between different inputs to a function, with copatterns one can observe the output. The need to observe output arises naturally when describing infinite structures. Copatterns are thus connected to \textit{codata}. Where data type definitions are used for finite structures like natural numbers, codata type definitions are used for infinite structures like streams. Consider the following example program:

\begin{lstlisting}

codata Stream = {head :: Nat, tail :: Stream}

repeat :: Nat -> Stream
head (repeat n) = n
tail (repeat n) = repeat n

\end{lstlisting}

The codata type definition for \texttt{Stream} gives the ways, called \textit{destructors}, in which a stream can be observed, or \textit{destructed}. One is fetching its head, which is a natural number, the other is fetching the tail, which again is a stream. The function \texttt{repeat} describes a stream that always returns the same number when fetching its head. For this, the left-hand sides of the definition for \texttt{repeat} use copatterns; one applies the destructor \texttt{head}, the other the destructor \texttt{tail}.

Uroboro is a functional programming language. Instead of first-class functions, Uroboro has its generalization, codata types, and copattern matching. We first show how two fragments of Uroboro relate codata and copatterns with refunctionalization. Then we motivate why Uroboro is interesting -- in general, and in particular for our purpose of automatic program transformations.

How do copatterns and codata relate to refunctionalization? Rendel et al.\cite{rendel15automatic} show how a program in a certain first-order language -- which they call the Data Fragment -- can be automatically transformed to a language -- which they call the Codata Fragment -- with copattern matching, and vice versa. The authors intend these two languages to be fragments of Uroboro. The key idea of Rendel et al. is that codata can be regarded as a generalization of first-class functions. This solves the problem of the multiple apply functions described above. Consider the following example program in the Data Fragment.

\begin{lstlisting}

data Nat where
  Zero(): Nat
  Succ(Nat): Nat

function add(Nat, Nat): Nat where
  add(Zero(), n) = n
  add(Succ(m), n) = Succ(add(m, n))

function multiply(Nat, Nat): Nat where
  multiply(Zero(), n) = Zero()
  multiply(Succ(m), n) = add(multiply(m, n), n)

\end{lstlisting}

Here, both \texttt{add} and \texttt{multiply} could possibly play the role of apply functions. The refunctionalization of Rendel et al. does the following:
\begin{enumerate}
\item The data type definition \texttt{Nat} is replaced by a codata type definition which has a destructor for each function whose first argument has type \texttt{Nat}.
\item For each constructor of \texttt{Nat}, a function definition is added, the equations of which correspond to the equations of all function definitions (here, both \texttt{add} and \texttt{multiply}) where the first argument of the left-hand side pattern is this constructor. Instead of a pattern with a constructor, the left-hand sides of these are now copatterns with the corresponding destructor. The right-hand sides are likewise transformed.
\end{enumerate}
The result of refunctionalization is shown below.

\begin{lstlisting}

codata Nat where
  Nat.add(Nat): Nat
  Nat.multiply(Nat): Nat

function Zero(): Nat where
  Zero().add(n) = n
  Succ(m).add(n) = Succ(m).add(n)

function Succ(Nat): Nat where
  Zero().multiply(n) = Zero()
  Succ(m).multiply(n) = m.multiply(n).add(n)

\end{lstlisting}
The Codata Fragment's syntax for destructors, \texttt{fun(s).des} is taken from Abel et al.\cite{abel13copatterns} and is syntactical sugar for the form \texttt{des(fun(s))} used above.

Rendel et al.s\cite{rendel15automatic} refunctionalization for the Data Fragment finds its reverse in their defunctionalization for the Codata Fragment. As an example, consider the result of refunctionalization of the example program above. Defunctionalizing this program leads back to the original program. Rendel et al. show that, when one views programs as two-dimensional matrices, defunctionalizing the Codata Fragment and refunctionalizing the Data Fragment can be regarded as one and the same function, namely matrix transposition.

Rendel et al. intend the Data Fragment and the Codata Fragment to be fragments of a common language, called Uroboro. In the \hyperref[ch:uro]{following chapter}, we formally define this language in full for the first time. Here, we motivate why we think Uroboro is interesting in general and for our purpose, automatic program transformations.

In our opinion, Uroboro is well suited for automatic program transformations which are related to the conversion between first-order and higher-order programs. There are two major reasons for this:
\begin{itemize}
\item Uroboro isn't truly ``higher-order'', i.e., it has no first-class functions and no function types. Functions and destructors have fixed-length argument lists, i.e., there can be no ``underapplication'' (but this can be emulated with destructors). Destructors in Uroboro are thus somewhat similar to methods in object-oriented programming languages.

\item Uroboro is symmetric in the sense that it has both data type definitions and its dual codata type definition, along with patterns and its dual copatterns. This is interesting for automatic transformations, since, as described before, the Data Fragment and the Codata Fragment of Uroboro are connected by refunctionalization and defunctionalization.
\end{itemize}
We aren't aware of any existing languages which have both of these properties.

As stated before, codata generalizes first-class functions. For this reason, Uroboro doesn't lose expressiveness when compared to languages with first-class functions but without parametric polymorphism. Thus, in our opinion, Uroboro, or rather a future version of it that includes parametric polymorphism, has the potential to replace certain higher-order languages. We give more details on this in \autoref{sec:reluro}.

We now sum up the remaining contents of this work.

In \autoref{ch:uro}, we define the language Uroboro. We give its syntax in \autoref{sec:urosyn}, typing in \autoref{sec:urostatsem}, and finally its dynamic semantics in a small-step operational way \autoref{sec:urosos}. Then we discuss coverage (\autoref{sec:cc}) and type soundness (\autoref{sec:urots}). We also show how the reduction relation of the full Uroboro extends that of the Data and Codata Fragment (\autoref{sec:codfragext}).

In \autoref{ch:extr}, we introduce the notion of \textit{extractions}. Extractions are basic automatic program transformations that decrease the syntactic complexity of the program while preserving its reduction semantics in some way. Syntactic complexity is decreased in the left-hand sides of some of the program's equations, such that the new lhs is the result of what we call a extraction projection applied to the old lhs. An extraction projection is a function which is the inverse of either a substitution or its dual, a co-substitution. The reduction semantics are preserved in a weak bisimulation; this is achieved by introducing an auxiliary function. The precise definition of extractions is given in \autoref{sec:extrdef}; example extractions are given in \autoref{sec:extrex}. The proof of bisimulation follows in \autoref{sec:extrbis}. Under which circumstances an important property, the absence of overlaps, holds in programs that are the result of extraction, is stated and proven in \autoref{sec:extrovl}. The preservation of well-typedness is discussed in \autoref{sec:extrpwt}.

In \autoref{ch:derefunc}, we give an automatic defunctionalization and refunctionalization for the full Uroboro. Both are made up of two phases. First, the program needs to be simplified. We describe this process, called unnesting, in \autoref{sec:unn}. After this, core defunctionalization and core refunctionalization follow, described in \autoref{sec:coredefunc} and \autoref{sec:corerefunc}, respectively.

In \autoref{ch:impl}, we describe how we implemented our program transformations in Haskell. We also give an example application for defunctionalization and one for refunctionalization, both of which are known from the literature, but have only been transformed manually before.

\autoref{ch:rel} considers related and future work.

The \hyperref[ch:concl]{seventh and final chapter} concludes.

\chapter{The language Uroboro}

In this chapter we formally define the language Uroboro. It is a more or less straightforward generalization of the Data and Codata Fragments of Rendel et al.\cite{rendel15}; we therefore orient ourselves on their presentation. We will also sometimes orient ourselves on Abel et al.\cite{abel13copatterns}, which also define a language with copatterns, albeit including first-class functions unlike Uroboro. We first define the syntax of Uroboro (\autoref{sec:urosyn}), then its typing (\autoref{sec:urostatsem}), and finally its dynamic semantics (\autoref{sec:urosos}), which we present as small-step operational semantics.

In the introduction, we have already shown example programs from the Data and Codata Fragments. Before we start with the definitions, we give an example Uroboro program that is neither in the Data nor in the Codata Fragments, i.e., it contains both data and codata types.

\begin{lstlisting}



\end{lstlisting}

\section{Syntax}
\label{sec:urosyn}

We assume the following domains of symbols.

\begin{align*}
&\sigma, \tau = &\textrm{type names}\\
&con = &\textrm{constructor names}\\
&des = &\textrm{destructor names}\\
&fun = &\textrm{function names}\\
&x, y = &\textrm{variable names}\\
\end{align*}

The \textit{terms} of Uroboro are variables, and constructor calls, destructor calls, and function calls to lists of terms.

\[
s, t ::= x ~ | ~ fun(t^*) ~ | ~ con(t^*) ~ | ~ s.des(t^*)
\]

The destructor call syntax is syntactical sugar for $des(s, t^*)$, meant to easier distinguish between the destructed (or observed) term and the other arguments of the destructor. A subset of terms is the set of copatterns, which we denote $\mathbf{Cop}$; a copattern is either a function call applied to patterns, or a destructor destructing another copattern and otherwise applied to patterns.

\begin{align*}
&q ::= fun(p^*) ~ | ~ q.des(p^*) \\
&p ::= x ~ | ~ con(p^*)
\end{align*}

\textit{Programs} are defined to be sets of \textit{definitions}. A definition can either be a data type definition, a codata type definition, or a function definition.

\[
prg ::= (ddef ~ | ~ coddef ~ | ~  fdef)^*
\]

\textit{Data (type) definitions} consist of the name of the data type together with a set of constructor signatures.

\begin{align*}
&ddef &::= \textbf{data } \sigma \textbf{ where } csig^* &\\
&csig &::= con(\tau^*): \tau &\\
\end{align*}

\textit{Codata (type) definitions} consist of the name of the codata type together with a set of destructor signatures.

\begin{align*}
&coddef &::= \textbf{codata } \sigma \textbf{ where } dsig^*\\
&dsig &::= \sigma.des(\tau^*): \tau\\
\end{align*}

\textit{Function definitions} consist of the signature of the function together with a set of \textit{equations}. Equations have a left-hand side, which is a copattern, and a right-hand side, which is some term.

\begin{align*}
&fdef &::= \textbf{function } fun(\tau^*): \tau \textbf{ where } eqn^*\\
&eqn &::= q = t
\end{align*}

Note that we consider the collection of equations in a function definition to be a \textit{set}. This means that we establish no order on the equations. We don't consider order-dependent programs in this work; suffice it to say that a variant of Uroboro augmented with order would require a change to the semantics. In general, this is a problem not yet considered for languages with copattern matching; for instance, the language of Abel et al.\cite{abel13copatterns} also doesn't have an order on equations.

\section{Static semantics}
\label{sec:urostatsem}

As is the case with the copattern language of Abel et al.\cite{abel13copatterns}, there are two kinds of types in Uroboro. \textit{Positive types}, which are those defined by data type definitions, and \textit{negative types}, which are those defined by codata type definitions. Like Rendel et al. do for their fragments, we give a type judgement for terms, and one to judge whether equations are well-typed.  Like them, we let $\Sigma$ be the signatures of a program, that is, its function signatures, codata signatures, and data signatures, as defined above, and we let $\Gamma$ be a typing context, i.e., a set of type assignments to variables, and write
\begin{itemize}
\item $\Gamma \vdash_{\Sigma} t : \tau$ to mean that term $t$ is well-typed with type $\tau$ under typing context $\Gamma$ and signature $\Sigma$, and

\item $\Sigma \vdash eqn \textrm{ ok}$ to mean that equation $eqn$ is well-typed under signature $\Sigma$.
\end{itemize}
We give the following rules for the type judgement for terms, which extend those of the Data and Codata Fragments. In fact, we simply combine the rules of the Data and Codata Fragments.

\begin{prooftree}
\AxiomC{$``x : \tau" \in \Gamma$}
\RightLabel{Var}
\UnaryInfC{$\Gamma \vdash_{\Sigma} x : \tau$}
\end{prooftree}

\begin{prooftree}
\AxiomC{$``fun(\tau_1, ..., \tau_n): \tau" \in \Sigma$}
\AxiomC{$\Gamma \vdash_{\Sigma} t_1 : \tau_1$}
\AxiomC{...}
\AxiomC{$\Gamma \vdash_{\Sigma} t_n : \tau_n$}
\RightLabel{Fun}
\QuaternaryInfC{$\Gamma \vdash_{\Sigma} fun(t_1, ..., t_n): \tau$}
\end{prooftree}

\begin{prooftree}
\AxiomC{$``con(\tau_1, ..., \tau_n): \tau" \in \Sigma$}
\AxiomC{$\Gamma \vdash_{\Sigma} t_1 : \tau_1$}
\AxiomC{...}
\AxiomC{$\Gamma \vdash_{\Sigma} t_n : \tau_n$}
\RightLabel{Con}
\QuaternaryInfC{$\Gamma \vdash_{\Sigma} con(t_1, ..., t_n): \tau$}
\end{prooftree}

\begin{prooftree}
\AxiomC{$``\sigma.des(\tau_1, ..., \tau_n): \tau" \in \Sigma$}
\AxiomC{$\Gamma \vdash_{\Sigma} s : \sigma$}
\AxiomC{$\Gamma \vdash_{\Sigma} t_1 : \tau_1$}
\AxiomC{...}
\AxiomC{$\Gamma \vdash_{\Sigma} t_n : \tau_n$}
\RightLabel{Des}
\QuinaryInfC{$\Gamma \vdash_{\Sigma} s.des(t_1, ..., t_n): \tau$}
\end{prooftree}

We give the following rules for the well-typedness judgement for equations, which generalize the rules for the Data and Codata Fragment. We write $V(q)$ to denote the variables occuring in $q$.

\begin{prooftree}
\AxiomC{$q = q'.des(p_1, ..., p_n)$}
\noLine
\UnaryInfC{$``\sigma.des(\tau_1, ..., \tau_n): \tau" \in \Sigma$}
\AxiomC{$(q \vdash_{\Sigma} x : \sigma_x)_{x \in V(q)}$}
\AxiomC{$\{ x : \sigma_x ~ | ~ x \in V(q) \} \vdash_{\Sigma} t : \tau$}
\RightLabel{Des}
\TrinaryInfC{$\Sigma \vdash q = t \textrm{ ok}$}
\end{prooftree}

\begin{prooftree}
\AxiomC{$q = fun(p_1, ..., p_n)$}
\noLine
\UnaryInfC{$``fun(\tau_1, ..., \tau_n): \tau" \in \Sigma$}
\AxiomC{$(q \vdash_{\Sigma} x : \sigma_x)_{x \in V(q)}$}
\AxiomC{$\{ x : \sigma_x ~ | ~ x \in V(q) \} \vdash_{\Sigma} t : \tau$}
\RightLabel{Fun}
\TrinaryInfC{$\Sigma \vdash q = t \textrm{ ok}$}
\end{prooftree}

Here, $q \vdash_{\Sigma} x : \sigma$, for $x \in V(q)$, is an auxiliary judgement which means that, within copattern $q$, the variable has type $\sigma$ under signature $\Sigma$. To put it another way, it is a type inference for variables. This judgement, in turn, uses an analogous judgement for patterns. The rules for both judgements are given below.

\begin{center}
\AxiomC{$x = p_i$}
\AxiomC{$``con(\tau_1, ..., \tau_n): \tau" \in \Sigma$}
\RightLabel{Con\textsubscript{Var}}
\BinaryInfC{$con(p_1, ..., p_n) \vdash_{\Sigma} x : \tau_i$}
\DisplayProof
\AxiomC{$x \in V(p_i), x \neq p_i$}
\AxiomC{$p_i \vdash_{\Sigma} x : \sigma$}
\RightLabel{Con\textsubscript{Con}}
\BinaryInfC{$con(p_1, ..., p_n) \vdash_{\Sigma} x : \sigma$}
\DisplayProof
\end{center}

\begin{prooftree}
\AxiomC{$x \in V(p_i)$}
\AxiomC{$p_i \vdash_{\Sigma} x : \sigma$}
\RightLabel{Fun}
\BinaryInfC{$fun(p_1, ..., p_n) \vdash_{\Sigma} x : \sigma$}
\end{prooftree}

\begin{center}
\AxiomC{$x \in V(p_i)$}
\AxiomC{$p_i \vdash_{\Sigma} x : \sigma$}
\RightLabel{Des\textsubscript{P}}
\BinaryInfC{$q.des(p_1, ..., p_n) \vdash_{\Sigma} x : \sigma$}
\DisplayProof
\AxiomC{$x \in V(q)$}
\AxiomC{$q \vdash_{\Sigma} x : \sigma$}
\RightLabel{Des\textsubscript{Q}}
\BinaryInfC{$q.des(p_1, ..., p_n) \vdash_{\Sigma} x : \sigma$}
\DisplayProof
\end{center}

We say that a program with signatures $\Sigma$ is well-typed when it is $\Sigma \vdash eqn \textrm{ ok}$ for all equations $eqn$ of the program.

\section{Dynamic semantics}
\label{sec:urosos}

In this section, we formally define the semantics of Uroboro. We present them as small-step operational semantics. Before defining the reduction relation, a few other definitions are in order. In Uroboro, a reduction is a term rewriting that uses one of the equations of the program. Therefore, we need to define what it means for a term to match an lhs, that is, a copattern; we do so in \autoref{ssec:copm}. For languages with codata, there is a special notion of value; we define what values are in Uroboro in \autoref{ssec:val}. Then we define the reduction relation (\autoref{ssec:red}). For some programs, terms can be stuck, that is, they are neither reducible nor considered a value. Such cases are excluded by copattern coverage, for which we give a judgement in \autoref{ssec:cc}. Finally, we consider how the semantics relate to the semantics of the Data and Codata Fragments (\autoref{ssec:codfragext}).

\subsection{Copattern matching}
\label{ssec:copm}

Here, we define what it means for a term to match a copattern. In short, a term $t$ matches a copattern $q$ when there is a substitution $\sigma$ such that $t = q[\sigma]$. We give a judgement for this, which orients itself on and borrows notations from Abel et al.\cite{abel13copatterns}, section 4. In order to do so, we also give an analogous judgement for patterns. In terms of the judgement for patterns, $t =^? p \searrow \sigma$ means that term t matches with pattern $p$ under substitution $\sigma$ (i.e. it is equivalent to $t = p[\sigma]$). The meaning of $t =^? q \searrow \sigma$, for copatterns $q$, is defined analogously. The rules for the pattern matching judgement are as follows.

\begin{prooftree}
\AxiomC{}
\RightLabel{PM\textsubscript{Var}}
\UnaryInfC{$t =^? x \searrow t/x$}
\end{prooftree}

\begin{prooftree}
\AxiomC{$t_i =^? p_i \searrow \sigma_i ~ \forall i \in \{1, ..., n\}$}
\RightLabel{PM\textsubscript{Con}}
\UnaryInfC{$con(t_1, ..., t_n) =^? con(p_1, ..., p_n) \searrow \sigma_1, ..., \sigma_n$}
\end{prooftree}

Using this judgement for patterns, we define the rules for the copattern matching judgement.

\begin{prooftree}
\AxiomC{$t_i =^? p_i \searrow \sigma_i ~ \forall i \in \{1, ..., n\}$}
\RightLabel{PM\textsubscript{App}}
\UnaryInfC{$fun(t_1, ..., t_n) =^? fun(p_1, ..., p_n) \searrow \sigma_1, ..., \sigma_n$}
\end{prooftree}

\begin{prooftree}
\AxiomC{$t =^? q \searrow \sigma$}
\AxiomC{$t_i =^? p_i \searrow \sigma_i ~ \forall i \in \{1, ..., n\}$}
\RightLabel{PM\textsubscript{Des}}
\BinaryInfC{$t.des(t_1, ...t_n) =^? q.des(p_1, ..., p_n) \searrow \sigma, \sigma_1, ..., \sigma_n$}
\end{prooftree}

\subsection{Values}
\label{ssec:val}

This section formalizes the notion of value for Uroboro. As is the case with the language of Abel et al.\cite{abel13copatterns}, whether a term is a value depends upon its type. Thus for the following judgement rules a type for every term will be presupposed.

$\vdash_v t$ means that the closed term $t$ is a value. This notation is borrowed from Abel et al.\cite{abel13copatterns}, section 5. Note that only closed terms are considered because only those are relevant for the semantics preservation.

The value judgement makes use of a ``no-match'' judgement. We write $t \neq^? q$ to mean that $t$ doesn't match $q$, i.e., it is $t \neq q[\sigma]$ for all substitutions $\sigma$. This judgement is complementary to the copattern matching judgement of the previous section. In order to define the rules of the ``no-match'' judgement for copatterns, as with the matching judgement, we give the rules of an analogous ``no-match'' judgement for patterns.

\begin{prooftree}
\AxiomC{$t_i \neq^? p_i$}
\RightLabel{NPM\textsubscript{Con/Mism}}
\UnaryInfC{$con(t_1, , ..., t_i, ..., t_n) \neq^? con(p_1, ..., p_i, ..., p_n)$}
\end{prooftree}

\begin{prooftree}
\AxiomC{$con \neq con'$}
\RightLabel{NPM\textsubscript{Con/Diff}}
\UnaryInfC{$con(t_1, , ..., t_n) \neq^? con'(p_1, ..., p_m)$}
\end{prooftree}

\begin{prooftree}
\AxiomC{$t_i \neq^? p_i$}
\RightLabel{NPM\textsubscript{App/Mism}}
\UnaryInfC{$fun(t_1, ..., t_i, ..., t_n) \neq^? fun(p_1, ..., p_i, ..., p_n)$}
\end{prooftree}

\begin{prooftree}
\AxiomC{$fun \neq fun'$}
\RightLabel{NPM\textsubscript{App/Diff}}
\UnaryInfC{$fun(t_1, ..., t_n) \neq^? fun'(p_1, ..., p_m)$}
\end{prooftree}

\begin{prooftree}
\AxiomC{$t \neq^? q$}
\RightLabel{NPM\textsubscript{Des/Mism}}
\UnaryInfC{$t.des(t_1, ..., t_n) \neq^? q.des(p_1, ..., p_n)$}
\end{prooftree}

\begin{prooftree}
\AxiomC{$t_i \neq^? p_i$}
\RightLabel{NPM\textsubscript{Des/MismArgs}}
\UnaryInfC{$t.des(t_1, ..., t_i, ..., t_n) \neq^? q.des(p_1, ..., p_i, ..., p_n)$}
\end{prooftree}

\begin{prooftree}
\AxiomC{$des \neq des'$}
\RightLabel{NPM\textsubscript{Des/Diff}}
\UnaryInfC{$t.des(t_1, ..., t_n) \neq^? q.des'(p_1, ..., p_m)$}
\end{prooftree}

\begin{prooftree}
\AxiomC{}
\RightLabel{NPM\textsubscript{Diff1}}
\UnaryInfC{$fun(...) \neq^? q.des(...)$}
\end{prooftree}

\begin{prooftree}
\AxiomC{}
\RightLabel{NPM\textsubscript{Diff2}}
\UnaryInfC{$t.des(...) \neq^? fun(...)$}
\end{prooftree}

Finally, using this ``no-match'' judgement, we define the rules of the value judgement. As said above, this judgement also depends upon the type of the term. In short, a term is a value whenever it is a either (a) constructor applied to values, or (b) a function or destructor applied to values such that the return type of the function or destructor is a negative type, i.e., a type defined by a codata type definition.

\begin{prooftree}
\AxiomC{$fun(t_1, ..., t_n) \neq^? q ~ \forall (q, t) \in \textrm{Rules}(prg)$}
\AxiomC{$\vdash_v t_1$}
\AxiomC{...}
\AxiomC{$\vdash_v t_n$}
\RightLabel{V\textsubscript{CodTFun}}
\QuaternaryInfC{$\vdash_v fun(t_1, ..., t_n)$, if $fun(t_1, ..., t_n)$ has codata type}
\end{prooftree}

\begin{prooftree}
\AxiomC{$t_0.des(t_1, ..., t_n) \neq^? q ~ \forall (q, t) \in \textrm{Rules}(prg)$}
\AxiomC{$\vdash_v t_0$}
\AxiomC{$\vdash_v t_1$}
\AxiomC{...}
\AxiomC{$\vdash_v t_n$}
\RightLabel{V\textsubscript{CodTDes}}
\QuinaryInfC{$\vdash_v t_0.des(t_1, ..., t_n)$, , if $t_0.des(t_1, ..., t_n)$ has codata type}
\end{prooftree}

\begin{prooftree}
\AxiomC{$\vdash_v t_1$}
\AxiomC{...}
\AxiomC{$\vdash_v t_n$}
\RightLabel{V\textsubscript{Con}}
\TrinaryInfC{$\vdash_v con(t_1, ..., t_n)$}
\end{prooftree}

\subsection{The reduction relation}
\label{ssec:red}

$\longrightarrow$ is the one-step reduction relation for closed terms, which is assumed to be used with respect to the rules $\textrm{Rules}(prg)$ of a program $prg$, where a rule $(q, t)$ consists of a copattern $q$ (left-hand side of the rule) and a term $t$ (right-hand side of the rule). The reduction depends upon a set of terms $\textrm{Val}$ which are considered to be values. Standardly, $\textrm{Val}$ is assumed to be the set of terms which are judged to be values by $\vdash_v t$. Sometimes, an alternative reduction judgement will be used, which will only differ in the underlying value set $\textrm{Val}$. This alternative reduction will then be defined by giving the alternative $\textrm{Val}$.

We define evaluation contexts for left-to-right, call-by-value evaluation.
\begin{alignat*}{2}
& v  \in &&\textrm{Val} \\
& \mathcal{E} ::= ~&& [] ~ | ~ fun(v^*, \mathcal{E}, t^*) ~ | ~ con(v^*, \mathcal{E}, t^*) ~ | ~ \mathcal{E}.des(t^*) ~ | ~ v.des(v^*, \mathcal{E}, t^*)
\end{alignat*}
We let $\mathbf{EC}$ denote the set of evaluation contexts for the standard value judgement $\vdash'_v$, and we let $\mathbf{EC}[V]$ denote the set of evaluation contexts for other value sets (or their respective judgements).

Contraction $\mapsto$ of terms using equations is defined as follows.

\begin{center}
if the immediate subterms of $t$ are values:
\AxiomC{$t =^? q \searrow \sigma$ with $(q, t') \in \textrm{Rules}(prg)$}
\RightLabel{Subst}
\UnaryInfC{$t \mapsto t'[\sigma]$}
\DisplayProof
\end{center}

Finally, we define the actual reduction relation by a congruence rule.
\begin{prooftree}
\AxiomC{$t \mapsto t'$}
\RightLabel{Cong}
\UnaryInfC{$\mathcal{E}[t] \longrightarrow \mathcal{E}[t']$}
\end{prooftree}

Write $\longrightarrow^=$ for the reflexive closure, $\longrightarrow^*$ for the reflexive and transitive closure of $\longrightarrow$. To make it clear that the reduction relation is meant with respect to a certain program $prg$, write $\longrightarrow^=_{prg}$, $\longrightarrow^*_{prg}$ and $\longrightarrow_{prg}$. When clear from the context or unimportant for the statement to make, this subscript will be omitted.

The well-known notions of \textit{context} and \textit{multi-hole context} are related to that of evaluation context. We will use them in some of the proofs of this work, and therefore give an Uroboro-specific definition for each of them here.

\begin{definition}[Context]
Contexts are defined by the following EBNF rule.
\[
\mathcal{E} ::= [] ~ | ~ fun(t^*, \mathcal{E}, t^*) ~ | ~ con(t^*, \mathcal{E}, t^*) ~ | ~ \mathcal{E}.des(t^*) ~ | ~ t.des(t^*, \mathcal{E}, t^*)
\]
\end{definition}

\begin{definition}[Multi-hole context]
Multi-hole contexts are defined by the following EBNF rule.
\[
\mathcal{E} ::= [] ~ | ~ fun((\mathcal{E} ~ | ~ t)^*) ~ | ~ con((\mathcal{E} ~ | ~ t)^*) ~ | ~ \mathcal{E}.des(t^*) ~ | ~ t.des((\mathcal{E} ~ | ~ t)^*)
\]
\end{definition}

Concerning evaluation contexts and contexts, the following holds, which will be used in the proof of \autoref{lem:cdpaux}.

\begin{fact}
\label{fac:chp21}
For any two contexts $\mathcal{C}, \mathcal{C}_0$ and any evaluation context $\mathcal{E} \in \mathbf{EC}[V]$ for some set of values $V$:
\[
\mathcal{E} = \mathcal{C}[\mathcal{C}_0] \implies \mathcal{C} \in \mathbf{EC}[V].
\]
\end{fact}

\subsection{Copattern coverage}

In general, for programs with \textit{coverage} it is guaranteed that no term is stuck. We adapt the judgement for \textit{copattern coverage} from Abel et al.\cite{abel13copatterns} for Uroboro. $fun \lhd | ~ Q$ means that the set of copatterns $Q$ \textit{covers} the function $fun$. Intuitively, when this set $Q$ is the set of left-hand sides of the function definition for $fun$, this function is fully specified. One can regard the consecutive application of the rules of the coverage judgement as a way of step-wise refining the specification of the function. The rule C\textsubscript{Head} stands for the start point, where the only left-hand side of the function neither has destructor nor constructor calls. The other two rules refine the specification of $fun$ by variable splitting (C\textsubscript{Con}), that is, refining based upon the input, or by result splitting (C\textsubscript{Des}), that is, refining based upon the output (or result). The terminology ``variable splitting'' and ``result splitting'' is also borrowed from Abel et al. and will become relevant in chapter 4.

These rules for the judgement are defined below. We assume a program and write $Dess_\tau$ for the set of destructors for codata type $\tau$, and $Cons_\tau$ for the set of constructors for data type $\tau$. We also write $q^\tau$ and $x^\tau$ to indicate that the type of the copattern $q$ or the variable $x$, respectively, has been inferred to be $\tau$. \autoref{sec:urostatsem} provides a way to infer types for variables, the type of a copattern is found out by looking it up in the respective function or destructor signature.

\begin{prooftree}
\AxiomC{}
\RightLabel{C\textsubscript{Head}}
\UnaryInfC{$fun \lhd | (fun(\overline{x}))$}
\end{prooftree}

\begin{prooftree}
\AxiomC{$fun \lhd | ~ Q ~ (q^\tau)$}
\RightLabel{C\textsubscript{Des}}
\UnaryInfC{$fun \lhd | ~ Q ~ (q.des(\overline{x^{des}}))_{des \in Dess_\tau}$}
\end{prooftree}

\begin{prooftree}
\AxiomC{$fun \lhd | ~ Q ~ (q(x^\tau))$}
\RightLabel{C\textsubscript{Con}}
\UnaryInfC{$fun \lhd | ~ Q ~ (q[x := con(\overline{y^{con}})])_{con \in Cons_\tau}$}
\end{prooftree}

A program has copattern coverage whenever all of its function definitions have coverage, that is, it is $fun \lhd | ~ Q$ for each function $fun$ and its set of left-hand sides $Q$.

In chapter 4, we will rely not only upon copattern coverage, but also on knowing how the coverage was derived by consecutive result and variable splitting. In other words, we need an algorithm for the judgement. Abel et al.\cite{abel13copatterns} hint, for their judgement, at a possible way to adapt other coverage algorithms for copatterns; to our knowledge, there currently is no such algorithm for copattern coverage checking, either by Abel et al. or others. For the time being, an exponential-time algorithm that goes through all derivation trees of a maximum depth will have to suffice. The maximum depth is the maximal number of constructors and destructors, added together, of all lhss of the function definition.

\subsection{As an extension of (Co)Data Fragment semantics}
\label{ssec:codfragext}

To conclude this chapter, we show how Uroboro's semantics relate to those of either the Data Fragment or of the Codata Fragment. Assuming well-typedness and copattern coverage, Uroboro's notion of reducibility is a conservative extension of either the Data Fragment's notion (\autoref{sssec:df}) or the Codata Fragment's notion (\autoref{sssec:codf}). Note that, when the well-typedness and coverage conditions are dropped, Uroboro's reducibility notion is still a conservative extension of the Data Fragment's notion.

\subsubsection{Data Fragment}
\label{sssec:df}

The evaluation context $\mathcal{E}$ conservatively extends the context for the Data Fragment, as explained below. The common congruence rule for both of this and their Codata Fragment is used here, as well. As a result of this and of the pattern matching rules given above, the notion of reduction for Uroboro of this work is a conservative extension of their Data Fragment reduction.

$\mathcal{E}$ uses a different notion of evaluation context than the usual, in that values are not syntactic. Determining whether something is an evaluation context is nonetheless unproblematic, as one can simply use the value judgement from \autoref{ssec:val}.

This work's evaluation context conservatively extends that for the Data Fragment, since the value judgement conservatively extends the syntactic notion of value for the Data Fragment: For this fragment, the Con rule is identical to the syntax rule (Figure 6b of Rendel et al.\cite{rendel15})
\[
u, v ::= con(v^*),
\]
while the V\textsubscript{CodTFun} can never be applied. This is because a function in the Data Fragment can only have data type unless it doesn't have a type defined in the program at all, as types can only be added by data definitions.

\subsubsection{Codata Fragment}
\label{sssec:codf}

The Codata Fragment's evaluation context notion is conservatively extended by this work's, as well. And here again, this is why the notion of reduction for all of Uroboro conservatively extends that for the Codata Fragment.

As with the Data Fragment, this conservation for evaluation context notions follows from that of the value notions. In their value rule for the Codata Fragment, a term is a value if and only if it is a function application with argument values. It will be demonstrated why the two directions of the equivalence hold for this work's value judgement, restricted to terms in the Codata Fragment, as well.

Consider first the ``if" direction. By the rules of the judgement, a function application with argument values is only a value if it has codata type and doesn't match any rule of the program. Assuming well-typedness, the first constraint is always fulfilled in the Codata Fragment, since there, a type can only be added by a codata definition. That the second constraint is always fulfilled because, in the Codata Fragment, there are no left-hand sides of rules which are hole patterns.

Second, consider the ``only if" direction. By the rules of the judgement, a value can also be a destructor application if it has codata type and doesn't match any rule of the program. But this case can be excluded in the Codata Fragment for programs with copattern coverage.

% !TEX root = main.tex
\chapter{Extraction transformations}
\label{ch:extr}

The automatic de- and refunctionalization algorithms presented in \autoref{ch:derefunc} are made up of several ingredients, some of which are part of the motivation for the contents of this chapter. In the preprocessing phase of either algorithm, there are steps which eliminate destructors or constructors from copatterns. In this chapter, we generalize this concept to arbitrary \textit{extractions}.

Extractions are (a way to describe) transformations, that roughly speaking, decrease, in some way, the syntactic complexity of the program, while preserving its semantics to some degree. We think that this might also be interesting independent of its application for automatic de- and refunctionalization in \autoref{ch:derefunc}. For instance, the user might want to decrease the syntactic complexity, e.g., by reducing the maximum number of destructors appearing in any lhs, in order to better understand the program. This situation is comparable to certain refactorings in object-oriented programming, like moving methods or extracting classes.

As an example, consider the extraction of a destructor out of the following program fragment.
\begin{lstlisting}
fun().des().des() = t
\end{lstlisting}
The result is shown below; the transformation has introduced an auxiliary function, which is the reason why we chose the name ``extraction'': Some syntactic component is ``extracted'' into a newly added part of the program.
\begin{lstlisting}
fun().des() = aux()
aux().des() = t
\end{lstlisting}
The resulting program fragment doesn't contain any copatterns with two destructors, unlike the original, thus the transformation has decreased the syntactic complexity in this way. The transformation preserves the semantics because, roughly, in the transformed program there is a way to go from the original equation's lhs to its rhs: $\mathtt{fun().des().des()} \longrightarrow \mathtt{aux().des()} \longrightarrow \mathtt{t}$.

In the next section, we formally define extractions and how they are applied to programs (\autoref{sec:extrdef}). We give some example extractions, two of which are used in \autoref{ch:derefunc} (\autoref{sec:extrex}). Then we show how extractions preserve the semantics of a program in a weak bisimulation (\autoref{sec:extrbis}). The preservation of the semantics depends upon the absence of overlapping equations both in the original and in the resulting program; therefore, the section that follows shows under which circumstances extractions don't introduce overlaps (\autoref{sec:extrovl}). Then we show that extractions preserve the well-typedness of programs (\autoref{sec:extrpwt}).

\section{Extractions}
\label{sec:extrdef}

Extractions are induced by a function $\pi$ that specifies how to decrease the syntactic complexity. Broadly, extractions fall into two classes: extractions that change the ``outer'' form, and those which change the ``inner'' form. Before defining extractions, we describe a way to uniformly express extractions that change the ``outer'' form, such as the destructor extraction exemplified above.

Changing the ``outer'' form of the copattern, i.e., removing or adding destructors, is dual to changing its ``inner'' form, i.e., replacing patterns with other patterns. For the purpose of defining extractions of the latter type, substitutions will be used. For the purpose of defining extractions of the former, we introduce the notion of \textit{co-substitution}.

\begin{definition}[Co-substitution]
A function $\sigma$ from copatterns to copatterns is called a \textit{co-substitution}, if, for every copattern $q$, it is defined as follows:
\[
\sigma(q) = q.\overline{des(\overline{p})},
\]
for some $\overline{des(\overline{p})}$ possibly depending on the $q$.
\end{definition}

Finally, extraction functions can be defined. More precisely, it is defined what it means to be an extraction projection, and for such an extraction projection $\pi$, a $\pi$-extraction targeting a set of equations $T$. For this, we make use of the notion of lenses, as defined by Foster et al.\cite{foster05combinators}.

\begin{definition}[Extraction projection]
\label{def:extrproj}
A function $\pi$ from copatterns to copatterns is called an extraction projection if for every copattern $q$ there exists a (co-)substitution $\sigma^q_\pi$ such that $\sigma^q_\pi(\langle q \rangle^\pi) = q$.
\end{definition}

\begin{definition}[$\pi$-lens]
The $\pi$-lens, for an extraction projection $\pi$, is the lens defined as follows:
\[
\mathtt{get} = \pi
\]
\[
\mathtt{putback}(q^a, q^c) = \sigma^{q^c}_\pi(q^a),
\]
where the $\sigma^{q^c}_\pi$ is the (co)-substitution for $q^c$ and $\pi$ as given in the definition of the extraction projection.
\end{definition}

\begin{definition}[$\pi$-extraction target]
A $\pi$-extraction target $T$ is a set of equations such that $\pi(q) = \pi(q')$ for any two lhss $q, q'$ of equations in $T$.
\end{definition}

\begin{definition}[$\pi$-extraction targeting $T$]
Let $aux$ be a fresh function name \footnote{In practice this means that is in undeclared in the program that is transformed by the extraction lifted to programs (see next section).} Let the pair of \textsf{get} and \textsf{putback} be the $\pi$-lens. For any copattern $q$, let $\langle q \rangle^{vars}$ denote the list of all variables of $q$ in the order that they appear in in $q$. Let $\langle \cdot \rangle^{aux}$ be the call to the auxiliary function that corresponds to the given copattern, defined as follows for copatterns $q$: $\langle q \rangle^{aux} = aux(\langle q \rangle^{vars})$. A \textit{$\pi$-extraction targeting an extraction target $T$} is a triple consisting of
\begin{itemize}
\item an equation $\epsilon$ with lhs $q_\epsilon = \textsf{get}(q)$ and rhs $t_\epsilon = \langle q_\epsilon \rangle^{aux}$, for some lhs $q$ of an equation in $T$,
\item a function $\zeta$ with domain $T$, defined as follows: $\zeta_r := ``\textsf{putback}(t_\epsilon, q_r) = t_r  "$, and
\item the following signature for the auxiliary function $aux$ (its equations are the image of $\zeta$): $`` \textrm{\textbf{function }} aux(\tau_1, ..., \tau_n): \sigma "$, with $\tau_1, ..., \tau_n$ the types inferred, using the signatures of $prg$, for the variables in $q_\epsilon$, in the order that they appear in in $q_\epsilon$, and $\sigma$ the type inferred, also using the signatures of $prg$, for $q_\epsilon$.
\end{itemize}
\end{definition}

\subsection{Applying extractions to programs}

An extraction function was defined as a triple. As it will be used to transform programs into programs, however, it is necessary to define how it should be applied to programs. The definition of the apply function is straightforward; all it does is replace each targeted lhs in the program with $\epsilon$, leaving all other lhss unchanged, and collecting the $\zeta_r$ for each targeted equation $r$ in an auxiliary function definition.

Let $e = (\epsilon, \zeta, sig)$ be an extraction targeting $T$, with
\[
def_T = `` \textrm{\textbf{function }} fun(\tau_1, ..., \tau_n): \sigma \textrm{\textbf{ where }} eqns "
\]
the function definition that contains the lhss of $T$.

\begin{alignat*}{4}
\langle prg \rangle^{apply(e)} & = &&\{ && \textrm{\textbf{function }} fun(\tau_1, ..., \tau_n): \sigma \textrm{\textbf{ where }} \{ r ~ | ~ r \in eqns, r \not\in T \} \cup \{ \epsilon \}, \\
& && && \textrm{\textbf{function }} sig \textrm{\textbf{ where }} \{ \zeta_r ~ | ~ r \in T \} \} \\
& \cup && \{ def \in prg ~ | ~ def \neq def_T \} \span\span\span\span
\end{alignat*}

\section{Prime extractions}
\label{sec:extrex}

In this section, we give some example extractions, destructor extraction and those from the family of constructor extractions. They are \textit{prime} extractions, in the sense that all other extractions' extraction projections are compositions of the underlying projections of destructor and constructor extractions.

\subsection{Destructor extraction}
\label{sec:desextr}

The extraction \textsf{ExtractDes} of a single destructor can be defined as follows: It is the $\pi$-extraction (targeting some $T$) for the extraction projection $\pi$ defined below.

\[
\pi(`` fun(\overline{p}) ") = `` fun(\overline{p}) "
\]
\[
\pi(`` q.des(\overline{p}) ") = `` q "
\]

Since copatterns without destructors aren't affected, this extraction is only meant to be used for copatterns with at least one destructor. We now show that $\pi$ is indeed an extraction projection, by giving a cosubstitution $\sigma^q_\pi$ for each $q$ such that $\sigma^q_\pi(\pi(q)) = q$.
\[
\sigma^{q^c}_\pi(q^a) = \begin{cases}
                              q^a.des(\overline{p}) &,\text{ if } q^c = q'.des(\overline{p}) \\
                              q^a &,\text{ otherwise}
                              \end{cases}
\]
With $\texttt{get} = \pi$ and $\texttt{putback}(q^a, q^c) = \sigma^{q^c}_\pi(q^a)$ we have the $\pi$-lens. We would like to illustrate with a little Haskell implementation how straightforward the two parts of the lens fit together.

\begin{lstlisting}
desExtrGet :: Cop -> Cop
desExtrGet (Des _ cop _) = cop

desExtrPutback :: Cop -> Cop -> Cop
desExtrPutback cop (Des des _ args) = Des des cop args
\end{lstlisting}

All extraction projections that are the counterparts to cosubstitutions can be defined as a composition of the destructor extraction projection: Simply extract the destructors one after the other.

As an example for destructor extraction, consider the function definition for \texttt{oneElemArray} from the program of \autoref{fig:ch2uroex}.
\begin{lstlisting}
function oneElemArray(Nat): Array where
  oneElemArray(n).get(Zero()) = n
  oneElemArray(n).get(Succ(m)) = Zero()
\end{lstlisting}
Targeting both equations for extraction of destructor \texttt{get} leads to the changed function definition for \texttt{oneElemArray} and an auxiliary function definition.
\begin{lstlisting}
function oneElemArray(Nat): Array where
  oneElemArray(n) = aux(n)

function aux(Nat): Array
  aux(n).get(Zero()) = n
  aux(n).get(Succ(m)) = Zero()
\end{lstlisting}
Note that this transformation didn't decrease the overall syntactic complexity of the program, since the complexity of \texttt{oneElemArray}, where the lhss contained one constructor and one destructor each, is simply transferred over to \texttt{aux}. One way to actually decrease the syntactic complexity of this example is the extraction of the constructor calls \texttt{Zero()} and \texttt{Succ(m)}; constructor extraction is defined in the next section.

\subsection{Constructor extraction}
\label{ssec:conextr}

We define a family $\textsf{ExtractCon}(\ell)$ of extractions of single constructors. The parameter $\ell$ stands for the position of the constructor call to be extracted in the targeted copatterns. We assume that $\ell$ actually points to a constructor call with only variables as arguments. For such a position $\ell$, $\textsf{ExtractCon}(\ell)$ is defined as the $\pi_\ell$-extraction (targeting some $T$) for the extraction projection $\pi_\ell$ defined below.

We assume that, in all copatterns, any variable is named according to its position. This can be  achieved by renaming variables according to a naming schema. For instance, the variables $fun(m, n)$ might be renamed such that the resulting term is $fun(x0, x1)$, where the position of each variable can be identified by its name. This approach is easily extended to destructor copatterns; for constructor calls, the schema needs to be applied recursively. Write $\textsf{name}(\ell)$ for the name given to a variable at position $\ell$ under such a naming schema. Using this, we define $\pi_\ell$ for any copattern $q$ which satisfies our assumptions that (a) $\ell$ points to a constructor call with only variables as arguments and (b) $q$ is named according to the schema used in \textsf{name}.
\[
\pi_\ell(q) := q[con(\overline{x}) \mapsto \textsf{name}(\ell)]_\ell
\]
The bracket notation means that the replacement occurs at position $\ell$.

We now show why $\pi_\ell$ is an extraction projection, that is, we give, for each $q$, a substitution $\sigma^q_{\pi_\ell}$ such that $\sigma^q_{\pi_\ell}(\pi_\ell(q)) = q$: Set $\sigma^q_{\pi_\ell} := \{\textsf{name}(\ell) \mapsto con(\overline{x})\}$.

All extraction projections that are the counterparts to substitutions can be defined as a composition of constructor extraction projections: Simply extract the constructors one after the other, from innermost to outermost.

As an example for constructor extraction, consider again the function definition for \texttt{oneElemArray} from the program of \autoref{fig:ch2uroex}.
\begin{lstlisting}
function oneElemArray(Nat): Array where
  oneElemArray(n).get(Zero()) = n
  oneElemArray(n).get(Succ(m)) = Zero()
\end{lstlisting}
Targeting both equations for extraction of the constructor calls at the argument position of \texttt{get} leads to the changed function definition for \texttt{oneElemArray} and an auxiliary function definition.
\begin{lstlisting}
function oneElemArray(Nat): Array where
  oneElemArray(n).get(m) = aux(n, m)

function aux(Nat, Nat): Array
  aux(n, Zero()) = n
  aux(n, Succ(m)) = Zero()
\end{lstlisting}
Note than in this example, other than destructor extraction, this transformation actually decreased the overall syntactic complexity of the program. The original definition for \texttt{oneElemArray} contained lhss with both destructor and constructor calls; the new definition has one lhs with one destructor, but no constructor call, and the auxiliary definition has two lhss which both have one constructor, but no destructor. Of course, the opposite case also exists, i.e., there are programs which cannot be syntactically simplified by constructor extraction, but by destructor extraction.

\section{Bisimulation}
\label{sec:extrbis}

Every extraction preserves the semantics of programs in a kind of weak bisimulation, assuming that neither the original nor the transformed program have overlapping equations. There are two equivalent characterizations of this bisimulation. One is given and proved in the first subsection, the other follows in the second subsection and is shown to be equivalent to the first. For this section, we will always assume that neither the original nor the transformed program have overlapping equations.

Before we start, we define a function $\langle \cdot \rangle^{aux^{-1}}$ from terms to terms that serves as the opposite of $\langle \cdot \rangle^{aux}$. It is defined as replacing calls to the auxiliary function as generated by $\langle \cdot \rangle^{aux}$ with the original copatterns with their variables instantiated accordingly. This corresponds to what Setzer et al.\cite{setzer14unnesting} call a \textit{back-interpretation}; we will use this term here as well when referring to $\langle \cdot \rangle^{aux^{-1}}$. We reuse some of their argumentation for the semantics preservation of their unnesting transformations; details on how our extractions are related to their transformations follow in \autoref{sec:unn} and \autoref{sec:relunn}.

\subsection{Using a modified value judgement}

For the definition of this weak bisimulation, we modify the value judgement, using the back-interpretation, in the following way. For any term $t$ with names declared in $\langle prg \rangle$, let
\[
\langle prg \rangle \vdash'_v t :\iff prg \vdash_v \langle t \rangle^{aux^{-1}}.
\]

From this definition it immediately follows that $\mathcal{E}$ is an evaluation context with respect to this modified value judgement for $\langle prg \rangle$ if and only if $\langle \mathcal{E} \rangle^{aux^{-1}}$ is an evaluation context with respect to the original value judgement for $prg$. It also means that, for a term $t$ with all names declared in $\langle prg \rangle$, the immediate subterms of $t$ are values with this judgement for $\langle prg \rangle$ if and only if the immediate subterms of $\langle t \rangle^{aux^{-1}}$ are values with the original judgement for $prg$.

Write $\longrightarrow'$ for the reduction relation $\longrightarrow$ with its value judgement modified in this way. This modified reduction relation can (but doesn't need to) ``sidestep'' reductions which don't change the back-interpretation of a term; as an example, consider a program with a function $fun$, where the extraction creates an auxiliary function $aux$ with an equation $\epsilon = ``fun() = aux()"$: Assume the term $fun().des()$ reduces in one step to some term $t$ with $\longrightarrow_{prg}$. It cannot do so with $\longrightarrow_{\langle prg \rangle}$ because it first has to reduce to $aux().des()$ by the $\epsilon$ equation. But since $\longrightarrow'_{\langle prg \rangle}$ considers $aux()$ a value because $\langle aux() \rangle^{-1} = fun()$ is a value in $prg$, this reduction can ``ignore'' $\epsilon$ and directly go to $t$.

The weak bisimulation is defined as follows, for every $s,t$ with all of their names declared in $prg$:
\begin{equation}
\label{eq:bisim1}
s \longrightarrow_{prg}^* t \iff s {\longrightarrow'}_{\langle prg \rangle}^* t
\end{equation}
In other words, $\longrightarrow'_{\langle prg \rangle}$ is a conservative extension of $\longrightarrow_{prg}$. This statement is now proved using Theorem 4a of Setzer et al.\cite{setzer14unnesting} Their theorem makes use of a set \textsf{Good} which limits the terms of $\langle prg \rangle$ for which there is a back-interpretation, but which is only relevant in a context with first-class functions. We therefore present Theorem 4a in a simplified form -- a special case -- which omits any mention of \textsf{Good}, because in our case \textsf{Good} is simply the entire set of terms of $\langle prg \rangle$.

\begin{theorem}[Setzer et al.]
\label{thm:setzer4a}
Let $(\mathcal{A}, \longrightarrow), (\mathcal{A}', \longrightarrow')$ be abstract reduction systems, i.e., it is $\longrightarrow \subseteq \mathcal{A} \times \mathcal{A}$ and $\longrightarrow' \subseteq \mathcal{A}' \times \mathcal{A}'$, such that $\mathcal{A} \subseteq \mathcal{A}'$. Let \textsf{int} be a back-interpretation from $\mathcal{A}'$ into $\mathcal{A}$, $\textsf{m} : \mathcal{A}' \to \mathbb{N}$. We define the following conditions for the back-interpretation:\\
(SN 1) $\forall a, a' \in \mathcal{A}. a \longrightarrow a' \implies a {\longrightarrow'}^{\geq 1} a'$\\
(SN 2) If $a \in \mathcal{A}, a' \in \mathcal{A}'$ and $a \longrightarrow' a'$ then $\textsf{int}(a) {\longrightarrow}^{\geq 1} \textsf{int}(a')$ or $\textsf{int}(a) = \textsf{int}(a') ~ \land ~ \textsf{m}(a) > \textsf{m}(a')$.

(SN 1), (SN 2) imply that $(\mathcal{A}', \longrightarrow')$ is a conservative extension of $(\mathcal{A}, \longrightarrow)$, i.e. it is $a \longrightarrow^* a' \iff a {\longrightarrow'}^* a'$ for all $a, a' \in \mathcal{A}$, preserving strong normalisation.
\end{theorem}

\begin{proposition}
\label{prop:bisim1}
The weak bisimulation statement~\ref{eq:bisim1} holds for any transformation defined as $apply(e)$ for some $\pi$-extraction $e$ targeting a $T$.

\begin{proof}
By \autoref{thm:setzer4a}, it suffices to show the statements (SN1) and (SN2). For this, set $\textsf{int} = \langle \cdot \rangle^{aux^{-1}}$. Note that this unconversion is compatible with evaluation contexts, i.e.,
\[
\langle \mathcal{E}[s'] \rangle^{aux^{-1}} = \langle \mathcal{E} \rangle^{aux^{-1}}[\langle s' \rangle^{aux^{-1}}],
\]
because $\langle \cdot \rangle^{aux}$ converts to function calls, not destructor calls. Analogously, $\langle \cdot \rangle^{aux^{-1}}$ is compatible with substitutions, i.e.,
\[
\langle q[\sigma] \rangle^{aux^{-1}} = \langle q \rangle^{aux^{-1}}[\langle \sigma \rangle^{aux^{-1}}],
\]
where $\langle \cdot \rangle^{aux^{-1}}$ has been straightforwardly lifted to substitutions by applying it to each right-hand side individually.
Further, set \textsf{m} as the number of calls to the function targeted in the transformation, i.e., that of $q_\epsilon$.

(SN 1): We know that $s = \mathcal{E}[q_r[\sigma]] \longrightarrow_{prg} \mathcal{E}[t_r[\sigma]] = t$, for some evaluation context $\mathcal{E}$ of $prg$ and some equation $r$ of $prg$. When $r \not\in T$, i.e., $r$ is taken over unchanged to $\langle prg \rangle$, the reduction $\mathcal{E}[q_r[\sigma]] {\longrightarrow'}_{\langle prg \rangle} \mathcal{E}[t_r[\sigma]]$ can be derived identically to $\mathcal{E}[q_r[\sigma]] \longrightarrow_{prg} \mathcal{E}[t_r[\sigma]]$. When $r \in T$, it is $q_r = \sigma^{q_r}_\pi(q_\epsilon)$ and $t_r = t_{\zeta_r}$, and we distinguish between whether $\sigma^{q_r}_\pi$ is a substitution or a co-substitution.

When $\sigma^{q_r}_\pi$ is a substitution, it is $\sigma^{q_r}_\pi(q_\epsilon)[\sigma] = q_\epsilon[\sigma \circ \sigma^{q_r}_\pi]$, and we can derive:
\begin{flalign*}
&s = \mathcal{E}[q_r[\sigma]] = \mathcal{E}[q_\epsilon[\sigma \circ \sigma^{q_r}_\pi]] \\
\longrightarrow'_{\langle prg \rangle} ~ &\mathcal{E}[t_\epsilon[\sigma \circ \sigma^{q_r}_\pi]] = \mathcal{E}[q_{\zeta_r}[\sigma]] \\
\longrightarrow'_{\langle prg \rangle} ~ &\mathcal{E}[t_{\zeta_r}[\sigma]] = \mathcal{E}[t_r[\sigma]] = t
\end{flalign*}

When $\sigma^{q_r}_\pi$ is a co-substitution, it is $\sigma^{q_r}_\pi(q_\epsilon) = q_\epsilon.\overline{des(\overline{p})}$, and thus $\mathcal{E}[q_\epsilon.\overline{des(\overline{p})}[\sigma]] = \mathcal{E}[[].\overline{des(\overline{p})}[\sigma]][q_\epsilon[\sigma]]$. Since $\mathcal{E}_{des} := \mathcal{E}[[].\overline{des(\overline{p})}[\sigma]]$ is $\mathcal{E}$ with the hole ``shrunken'' from the right, it is an evaluation context for the same value judgements as $\mathcal{E}$, and thus we can derive:
\begin{flalign*}
&s = \mathcal{E}[q_r[\sigma]] = \mathcal{E}_{des}[q_\epsilon[\sigma]] \\
\longrightarrow'_{\langle prg \rangle} ~ &\mathcal{E}_{des}[t_\epsilon[\sigma]] = \mathcal{E}[q_{\zeta_r}[\sigma]] \\
\longrightarrow'_{\langle prg \rangle} ~ &\mathcal{E}[t_{\zeta_r}[\sigma]] = \mathcal{E}[t_r[\sigma]] = t
\end{flalign*}

(SN 2): Three cases will be distinguished: The reduction in $\langle prg \rangle$ can use either an equation taken over unchanged from $prg$ (1.), it can use a $\zeta_r$ (2.), or it can use $\epsilon$ (3.). Each case makes use of an evaluation context $\mathcal{E}$ of the reduction relation for $\langle prg \rangle$.
\begin{enumerate}
\item We know $s = \mathcal{E}[s'] = \mathcal{E}[q_r[\sigma]] \longrightarrow'_{\langle prg \rangle} \mathcal{E}[t_r[\sigma]] = t$. The desired reduction sequence can be given as follows:
\begin{flalign*}
&\langle \mathcal{E}[s'] \rangle^{aux^{-1}} \\
=~& \langle \mathcal{E} \rangle^{aux^{-1}}[\langle s' \rangle^{aux^{-1}}] \\
=~& \langle \mathcal{E} \rangle^{aux^{-1}}[\langle q_r \rangle^{aux^{-1}}[\langle \sigma \rangle^{aux^{-1}}]] \\
=~& \langle \mathcal{E} \rangle^{aux^{-1}}[q_r[\langle \sigma \rangle^{aux^{-1}}]] \\
 \longrightarrow_{prg}~& \langle \mathcal{E} \rangle^{aux^{-1}}[t_r[\langle \sigma \rangle^{aux^{-1}}]] \\
=~& \langle \mathcal{E} \rangle^{aux^{-1}}[\langle t_r \rangle^{aux^{-1}}[\langle \sigma \rangle^{aux^{-1}}]] \\
=~& \langle \mathcal{E} \rangle^{aux^{-1}}[\langle t_r[\sigma] \rangle^{aux^{-1}}] \\
=~& \langle \mathcal{E}[t_r[\sigma]] \rangle^{aux^{-1}} \\
=~& \langle t \rangle^{aux^{-1}}.
\end{flalign*}

\item We know $s = \mathcal{E}[s'] = \mathcal{E}[q_{\zeta_r}[\sigma]] \longrightarrow'_{\langle prg \rangle} \mathcal{E}[t_{\zeta_r}[\sigma]] = t$. The desired reduction sequence can be given as follows:
\begin{flalign*}
&\langle \mathcal{E}[s'] \rangle^{aux^{-1}} \\
=~& \langle \mathcal{E} \rangle^{aux^{-1}}[\langle s' \rangle^{aux^{-1}}] \\
=~& \langle \mathcal{E} \rangle^{aux^{-1}}[\langle q_{\zeta_r} \rangle^{aux^{-1}}[\langle \sigma \rangle^{aux^{-1}}]] \\
=~& \langle \mathcal{E} \rangle^{aux^{-1}}[q_r[\langle \sigma \rangle^{aux^{-1}}]] \\
\longrightarrow_{prg}~& \langle \mathcal{E} \rangle^{aux^{-1}}[t_r[\langle \sigma \rangle^{aux^{-1}}]] \\
=~& \langle \mathcal{E} \rangle^{aux^{-1}}[\langle t_r \rangle^{aux^{-1}}[\langle \sigma \rangle^{aux^{-1}}]] \\
=~& \langle \mathcal{E} \rangle^{aux^{-1}}[\langle t_r[\sigma] \rangle^{aux^{-1}}] \\
=~& \langle \mathcal{E}[t_r[\sigma]] \rangle^{aux^{-1}} \\
=~& \langle \mathcal{E}[t_{\zeta_r}[\sigma]] \rangle^{aux^{-1}} = \langle t \rangle^{aux^{-1}}.
\end{flalign*}

\item We know $s = \mathcal{E}[s'] = \mathcal{E}[q_\epsilon[\sigma]] \longrightarrow'_{\langle prg \rangle} \mathcal{E}[t_\epsilon[\sigma]] = t$. In this case, instead of giving a reduction sequence, the other side of the disjunction will be shown to hold.

Because $\langle q_\epsilon \rangle^{aux^{-1}} = \langle t_\epsilon \rangle^{aux^{-1}}$, it is $\langle s \rangle^{aux^{-1}} = \langle t \rangle^{aux^{-1}}$.

And because $q_\epsilon$ is of the function that is targeted in the transformation, and $t_\epsilon$ isn't, it is $\textsf{m}(q_\epsilon) > \textsf{m}(t_\epsilon)$, and consequently $\textsf{m}(s) > \textsf{m}(t)$. \qedhere
\end{enumerate}
\end{proof}
\end{proposition}

\subsection{Using the back-interpretation directly}

This characterization of the bisimulation, which is equivalent to the first, directly uses the back-interpretation. In short, extractions preserve semantic properties by introducing an equation which leads from a, syntactically, more complex to a less complex term, where both terms are meant to represent, semantically, the same ``object''. This ``sameness'' is expressed by one being the back-interpretation of the other.

The bisimulation is characterized as follows:
\[
s {\longrightarrow}_{prg}^* t \iff s \longrightarrow^*_{\langle prg \rangle} \widetilde{t}, \text{ with } \langle \widetilde{t} \rangle^{aux^{-1}} = t
\]
In order to prove it, it suffices to show that this characterization is equivalent to the first characterization:
\begin{equation}
s {\longrightarrow'}_{\langle prg \rangle}^* t \iff s \longrightarrow^*_{\langle prg \rangle} \widetilde{t},
\end{equation}
for every $s, t$ with names declared in $prg$.

First, we show that the $`` \Rightarrow "$ direction of (2.2), as expressed in the following lemma, holds. In the following, all reductions are meant with respect to program $\langle prg \rangle$.

\begin{lemma}[$`` \Rightarrow "$ direction of (2.2)]
\label{lem:prop2lr}
\[
s {\longrightarrow'}_{\langle prg \rangle}^* t \implies s \longrightarrow^*_{\langle prg \rangle} \widetilde{t}
\]
\end{lemma}

In order to prove this, we define a counterpart to the $\longrightarrow'$ reduction relation: Let
\[
\longrightarrow^{aux} = \{(a,b) : a \longrightarrow^{all} b ~ \land ~ \langle a \rangle^{aux^{-1}} = \langle b \rangle^{aux^{-1}}\},
\]
where $\longrightarrow^{all}$ is the reduction relation for $\langle prg \rangle$ whose underlying value set is the set of all terms, i.e., the reduction is allowed to choose any redex. Where $\longrightarrow'$ can ``sidestep'' reductions which don't change the back-interpretation of a term, $\longrightarrow^{aux}$ is purely ``interpretative'', that is, it only allows reductions to terms with equal back-interpretation. This is expressed formally in the following lemmas relating $\longrightarrow'$ and $\longrightarrow^{aux}$, one concerning their commutation, the other what we call their complementarity. The $`` \Rightarrow "$ direction of (2.2) follows from these, as shown below.

\begin{figure}
\begin{subfigure}{0.3\textwidth}
\begin{tikzpicture}
\path (0,0) node(a) {} 
      (3,0) node(b) {}
      (0,-3) node(c) {}
      (3,-3) node(d) {};
\draw[->] (a) -- (b);
\draw[->,color=blue] (a) -- (c);
\draw[->,dashed,color=blue] (b) -- (d) node [midway, above, sloped] () {$*$};
\draw[->,dashed] (c) -- (d) node [midway, above, sloped] () {$=$};
\end{tikzpicture}
\caption{For \autoref{lem:cdpaux}}
\end{subfigure}
\begin{subfigure}{0.3\textwidth}
\begin{tikzpicture}
\path (0,0) node(a) {}
      (3,1) node (a') {}
      (3,0) node(b) {}
      (0,-3) node(c) {}
      (3,-3) node(d) {};
\draw[->,dashed] (a) -- (a') node [midway, above, sloped] () {$=$};
\draw[->,dashed,color=blue] (a') -- (b) node [midway, above, sloped] () {$*$};
\draw[->,dashed,color=blue] (a) -- (c) node [midway, above, sloped] () {$*$};
\draw[->,color=blue] (b) -- (d);
\draw[->] (c) -- (d);
\end{tikzpicture}
\caption{For \autoref{lem:comminv}}
\label{fig:comminv}
\end{subfigure}
\caption{The commutations of \autoref{lem:cdpaux} and \autoref{lem:comminv}: blue arrows represent $\longrightarrow^{aux}$, black arrows $\longrightarrow'$}
\end{figure}

\begin{restatable}[Commutation]{lemma}{cdpaux}
\label{lem:cdpaux}

For all terms $a,b,c$ it holds that:
\[
a {\longrightarrow'} b ~ \land ~ a \longrightarrow^{aux} c \implies \exists d . b {\longrightarrow^{aux}}^* d ~ \land ~ c {\longrightarrow'}^= d
\]

\end{restatable}
\begin{proof}
\hyperref[prf:cdpaux]{Left for the appendix.}
\end{proof}

\begin{corollary}
\label{cor:cdpauxcor}

For all terms $a,b,c$ and $n \in \mathbb{N}$ it holds that:
\[
a {\longrightarrow'}^n b ~ \land ~ a {\longrightarrow^{aux}}^* c \implies \exists d . b {\longrightarrow^{aux}}^* d ~ \land ~ c {\longrightarrow'}^{\leq n} d
\]

\begin{proof}

By induction on $n$ and on the length of $a {\longrightarrow^{aux}}^* c$.

\end{proof}

\end{corollary}

\begin{restatable}[Complementarity]{lemma}{compl}
\label{lem:compl}

For all terms $a,b$ it holds that:

When $a \longrightarrow' b$ but $a \not\longrightarrow b$, then there is a $c$ with $a \longrightarrow c$ and $a \longrightarrow^{aux} c$.

\end{restatable}

\begin{proof}
\hyperref[prf:compl]{Left for the appendix.}
\end{proof}

\begin{proof}[Proof of \autoref{lem:prop2lr}]

By induction on the length $n$ of $s {\longrightarrow'}_{\langle prg \rangle}^* t$. For $n = 0$ it is $s = t$, thus simply choose $\widetilde{t} = t$. Now, consider the case that $n = n'+1$ for some $n' \geq 0$. We proceed by induction on the number $k$ of calls in $s$ to the function targeted in the transformation, i.e., that of $q_\epsilon$.

\begin{itemize}
\item $k = 0$. Consider the first step $s {\longrightarrow'}_{\langle prg \rangle} s_1$ of $s {\longrightarrow'}_{\langle prg \rangle}^* t$. Because there are no calls to the function of $q_\epsilon$ in $s$, for all subterms $s^0$ of $s$ it is $\vdash'_v s^0$ iff $\vdash_v s^0$. Consequently, the reduction step from $s$ to $s_1$ is also possible with reduction relation $\longrightarrow_{\langle prg \rangle}$, that is, $s \longrightarrow_{\langle prg \rangle} s_1$. By the outer induction hypothesis, we have the rest of the desired reduction sequence $s_1 \longrightarrow_{\langle prg \rangle} \widetilde{t}$.

\item $k = k' + 1$. Again, consider the first step $s {\longrightarrow'}_{\langle prg \rangle} s_1$ of the original sequence. We distinguish two cases.
\begin{enumerate}
\item $s \longrightarrow_{\langle prg \rangle} s_1$. With this, we have the first step of the desired reduction sequence. By the outer induction hypothesis, we have the rest of the desired reduction sequence $s_1 \longrightarrow_{\langle prg \rangle} \widetilde{t}$.

\item $s \not\longrightarrow_{\langle prg \rangle} s_1$. By \autoref{lem:compl}, we have an $s_{aux}$ with $s \longrightarrow^{aux} s_{aux}$ and $s \longrightarrow_{\langle prg \rangle} s_{aux}$. By \autoref{cor:cdpauxcor}, we have a $\widetilde{t}'$ with (a) $s_{aux} \longrightarrow^{\leq n} \widetilde{t}'$ and (b) $t {\longrightarrow^{aux}}^* \widetilde{t}'$ and thus $\langle \widetilde{t}' \rangle^{aux^{-1}} = t$. Apply the inner induction hypothesis to $s_{aux} \longrightarrow^{\leq n} \widetilde{t}'$ to obtain the desired sequence.
\end{enumerate}
\end{itemize}

\end{proof}

%%-- under construction
Now, we show the $`` \Leftarrow ''$ direction of (3.2).

\begin{lemma}[$`` \Leftarrow "$ direction of (3.2)]
\label{lem:prop2rl}
\[
s \longrightarrow^*_{\langle prg \rangle} \widetilde{t} \implies s {\longrightarrow'}_{\langle prg \rangle}^* t
\]
\end{lemma}

Again, we begin by showing a commutation lemma; but this time, we operate on the inverse relations of $\longrightarrow'$ and $\longrightarrow^{aux}$. The commutation of the inverses is different and non-standard, since it has an intermediate sequence of $\longrightarrow^{aux}$-steps, as shown in \autoref{fig:comminv}.

\begin{restatable}[Commutation (inverses)]{lemma}{comminv}
\label{lem:comminv}

For all terms $b,c,d$ it holds that:
\[
c {\longrightarrow'} d ~ \land ~ b \longrightarrow^{aux} d \implies \exists a, a' . a {\longrightarrow^{aux}}^* c ~ \land ~ a {\longrightarrow'}^= a' {\longrightarrow^{aux}}^*  b
\]

\end{restatable}
\begin{proof}
\hyperref[prf:comminv]{Left for the appendix.}
\end{proof}

\begin{corollary}
\label{cor:comminvcor}

For all terms $b,c,d$ it holds that:
\[
c \longrightarrow' d ~ \land ~ b {\longrightarrow^{aux}}^* d \implies \exists a, a' . a {\longrightarrow^{aux}}^* c ~ \land ~ a {\longrightarrow'}^= a' {\longrightarrow^{aux}}^* b
\]

\begin{proof}

By induction on the length of $b {\longrightarrow^{aux}}^* d$.

\end{proof}
\end{corollary}

\begin{proof}[Proof of \autoref{lem:prop2rl}]

First, note that this direction is logically equivalent to the statement
\[
\forall s, t \in \textrm{Term}_{prg}, \widetilde{t} \in \textrm{Term}_{\langle prg \rangle}. (s \longrightarrow^* \widetilde{t} ~ \land ~ \langle \widetilde{t} \rangle^{aux^{-1}} = t) \implies s {\longrightarrow'}^* t.
\]
We prove this statement by induction on the length $n$ of $s \longrightarrow^* \widetilde{t}$. For $n = 0$ it is $s = \widetilde{t}$; since $s \in \textrm{Term}_{prg}$ it is $\langle s \rangle^{aux^{-1}} = s$ and it follows that $s = \langle \widetilde{t} \rangle^{aux^{-1}} = t$. 

Now, consider the case that $n = n'+1$ for some $n' \geq 0$. Since $\longrightarrow \subseteq \longrightarrow'$ it is $s_1 \longrightarrow' \widetilde{t}$; and since $t = \langle \widetilde{t} \rangle^{aux^{-1}}$ it is $t {\longrightarrow^{aux}}^* \widetilde{t}$. Thus, by \autoref{cor:comminvcor} we have terms $t', t''$ with $t' {\longrightarrow'}^= t'' {\longrightarrow^{aux}}^* t$ and $t' {\longrightarrow^{aux}}^* s_1$. Because $t \in \textrm{Term}_{prg}$ it must be $t'' = t$. By the induction hypothesis we have a reduction sequence $s {\longrightarrow'}^* \langle s_1 \rangle^{aux^{-1}}$. Since it is $t' {\longrightarrow'}^= t$, and $t$ contains no calls to $aux$, neither does $t'$. Thus, from $t' {\longrightarrow^{aux}}^* s_1$ it follows that $\langle s_1 \rangle^{aux^{-1}} = t'$. By combining the thus known sequences and equalities, we get the desired sequence $s {\longrightarrow'}^* \langle s_1 \rangle^{aux^{-1}} = t' {\longrightarrow'}^= t$.

\end{proof}

Combining \autoref{lem:prop2lr} and \autoref{lem:prop2rl}, statement (3.2) obtains.
\begin{proposition}
\label{prop:bisim2}
Statement (3.2), that is
\[
s {\longrightarrow'}_{\langle prg \rangle}^* t \iff s \longrightarrow^*_{\langle prg \rangle} \widetilde{t},
\]
for every $s, t$ with names declared in $prg$, holds.
\end{proposition}

\section{Absence of overlaps}
\label{sec:extrovl}

For the bisimulation, we presupposed that the transformed program has no overlapping copatterns. Here, we shown that this is the case whenever $q_\epsilon$ doesn't overlap with any lhs of an equation taken over unchanged from $prg$.

\begin{proposition}
For any well-behaved extraction function lifted to programs, $\langle \cdot \rangle$, it is the case that if $prg$ has no overlapping copatterns and $q_\epsilon$ doesn't overlap with any lhs of an equation taken over unchanged from $prg$, then $\langle prg \rangle$ has no overlapping copatterns, as well.

\begin{proof}
First, the equations of the transformed are classified. There are three kinds of them: Those taken over unchanged over from $prg$ (indicated as $r$ in the table below), those which are an $\epsilon$ in a transformation result ($\epsilon$), and those which are, also in such a transformation result, a $\zeta_r$ ($\zeta$). The table below shows all possible combinations; its fields are filled with the number of the proof that lhss of equations of the respective kinds don't overlap.

\begin{tabular}{ l | c | c | r }  & r & $\epsilon$ & $\zeta$ \\ \hline r & (1) &  &  \\ \hline $\epsilon$ & (2) & (3) &  \\ \hline $\zeta$ & (4) & (5) & (6) \\ \hline \end{tabular}

\textit{ad} (1): Both equations are present in $prg$, thus their lhss don't overlap.

\textit{ad} (2): By assumption.

\textit{ad} (3): By the definition of the $q$-extraction, there is only one $\epsilon$-equation in the transformed program.

\textit{ad} (4): The $\zeta$-equation has a function name not declared in $prg$, unlike the $r$-equation.

\textit{ad} (5): The $\zeta$-equation has a function name not declared in $prg$, unlike the $\epsilon$-equation.

\textit{ad} (6): The lhss of each of the $\zeta$-equations are equivalent to a lhs in $prg$, thus, if they overlapped, so would these lhss in $prg$, contrary to fact.
\end{proof}
\end{proposition}

\section{Preservation of well-typedness}
\label{sec:extrpwt}

In this section, we show that extractions preserve well-typedness, i.e., that when applying an extraction to a well-typed program a well-typed program obtains. Let $prg$ be the original program, and $\langle prg \rangle$ the program after extraction. By definition a program is well-typed when all of its equations $eqn$ are well-typed, i.e., it is $\Sigma \vdash eqn \textrm{ ok}$, where $\Sigma$ are the signatures of the program. Let $\Sigma_{prg}$ be the signatures of $prg$, and $\Sigma_{\langle prg \rangle}$ be the signatures of $\langle prg \rangle$. Since extractions only add one function signature, but leave the other signatures unchanged, it is $\Sigma_{prg} \subseteq \Sigma_{\langle prg \rangle}$. The following thus holds trivially.

\begin{fact}
\label{fac:wtpfac}
All typing judgements still hold when replacing $\Sigma_{prg}$ with $\Sigma_{\langle prg \rangle}$.
\end{fact}

Like in \autoref{sec:extrovl}, we distinguish three kinds of equations in $\langle prg \rangle$: those left unchanged, the $\epsilon$ equation, and the $\zeta_r$ equations. For each of these kinds, we show a lemma stating that all of its equations are well-typed, under the assumption that $prg$  is well-typed (and thus all its equations). Combining these, the preservation of well-typedness follows.

\begin{lemma}
Assuming that $prg$ is well-typed, all equations $eqn$ of $\langle prg \rangle$ taken over unchanged from $prg$ are well-typed.

\begin{proof}
By assumption, we know that $\Sigma_{prg} \vdash eqn \textrm{ ok}$. By \autoref{fac:wtpfac} it follows that $\Sigma_{\langle prg \rangle} \vdash eqn \textrm{ ok}$.
\end{proof}
\end{lemma}

\begin{lemma}
\label{lem:wtpeps}
Assuming that $prg$ is well-typed, the equation $\epsilon$ of $\langle prg \rangle$, as specified in the definition of extractions, is well-typed.

\begin{proof}
We want to show that $\Sigma_{\langle prg \rangle} \vdash `` q_\epsilon = t_\epsilon " \textrm{ ok}$. By the judgement for well-typedness of equations this means that we have to show the following.
%\begin{enumerate}
%\item $q_\epsilon$ is either a function or destructor call, with the respective function or destructor signature present in $\Sigma_{\langle prg \rangle}$; let the return type given in this signature be $\sigma$.
%
%\item For each variable $x$ in $q_\epsilon$ it is $q_\epsilon \vdash_{\Sigma_{\langle prg \rangle}} x : \sigma_x$ for some type $\sigma_x$.
%
%\item It is $\{ x : \sigma_x ~ | ~ x \in V(q_\epsilon) \} \vdash_{\Sigma_{\langle prg \rangle}} t_\epsilon : \sigma$.
%\end{enumerate}
%The first point is easy to see. Since there is at least one left-hand side $q \in T$ (part of the extraction target) of $prg$, that we know is the result of applying a (co-)substitution to $q_\epsilon$, and $prg$ is well-typed, there must be a signature in $\Sigma_{prg}$ which holds the return type $\sigma$ for the destructor or function call that $q_\epsilon$ is. Consequently, since $\Sigma_{prg} \subseteq \Sigma_{\langle prg \rangle}$, we also have the return type $\sigma$ for the destructor or function call $q_\epsilon$ in the signature set $\Sigma_{\langle prg \rangle}$ of $\langle prg \rangle$.
%
%For the second point, again consider an lhs $q \in T$. Because it is possible to infer, using $\Sigma_{prg}$, a type for all variables in $q$, which results from applying a (co-)substitution to $q_\epsilon$, it also is possible to infer a type for all variables in $q_\epsilon$ using $\Sigma_{prg}$, and, by \autoref{fac:wtpfac}, also using $\Sigma_{\langle prg \rangle}$.
%
%The third point follows immediately from the definition of the signature of $aux$, as it is $t_\epsilon = aux(x_1, ..., x_n)$, with the $x_i$ being the variables of $q_\epsilon$, and the argument types of $aux$ are defined to be the types inferred for the variables of $q_\epsilon$, and the return type of $aux$ is defined to be the type $\sigma$ inferred for $q_\epsilon$.

\begin{enumerate}
\item $\Sigma_{\langle prg \rangle} \vdash q_\epsilon : \tau, \Gamma$ for some type $\tau$ and some typing context $\Gamma$, and

\item $\Gamma \vdash_{\Sigma} t_\epsilon : \tau$.
\end{enumerate}
First, consider the first point. There is at least one left-hand side $q \in T$ (part of the extraction target) of $prg$, that we know is the result of applying a (co-)substitution to $q_\epsilon$. Since $prg$ is well-typed, it is $\Sigma_{prg} \vdash q : \tau'; \Gamma'$, and by \autoref{fac:wtpfac} also $\Sigma_{\langle prg \rangle} \vdash q : \tau'; \Gamma'$. We can rebuild this derivation into a derivation for $\Sigma_{\langle prg \rangle} \vdash q_\epsilon : \tau, \Gamma$. When $q_\epsilon$ results from $q$ by a co-substitution, cut off the ``Des'' steps of the derivation until arriving at the statement concerning $q_\epsilon$. When $q_\epsilon$ results from $q$ by a substitution, we rebuild the derivations of the auxiliary judgements used for type inference of variables in patterns. Replace each such derivation for a statement $\tau \vdash_{\Sigma} p : \Gamma$, that concerns a pattern $p$ in $q$ which is replaced by some variable $x$ in $q_\epsilon$, with a derivation for $\tau \vdash_{\Sigma} x : x:\tau$ by the ``Var'' rule. Then adapt the (typing contexts in the) derivation steps following this accordingly.

Now, consider the second point. $t_\epsilon$ was defined as $aux(x_1, ..., x_n)$, with the $x_i$ being exactly the variables of $q_\epsilon$. The signature of $aux$ is defined as $`` aux(\tau_1, ..., \tau_n): \tau "$ with $\tau_i$ being precisely defined to be the types inferred for the $x_i$ in $q_\epsilon$, i.e., those that $\Gamma$ assigns to the $x_i$. Thus we can derive $\Gamma \vdash_{\Sigma} t_\epsilon : \tau$ using the ``Var'' rule for these variables, followed by the ``Fun'' rule for $aux$.
\end{proof}
\end{lemma}

\begin{lemma}
Assuming that $prg$ is well-typed, the equations $\zeta_r$, as specified in the definition of extractions, are well-typed.

\begin{proof}
First, note that by definition $aux$ is the function definition that $\zeta_r$ is a part of. We want to show that $\Sigma \vdash `` q_{\zeta_r} = t_{\zeta_r} " \textrm{ ok}$. By the judgement for well-typedness of equations this means that we have to show the following.
%\begin{enumerate}
%\item $q_{\zeta_r}$ is either a function or destructor call, with the respective function or destructor signature present in $\Sigma_{\langle prg \rangle}$; let the return type given in this signature be $\sigma$.
%
%\item For each variable $x$ in $q_{\zeta_r}$ it is $q_{\zeta_r} \vdash_{\Sigma_{\langle prg \rangle}} x : \sigma_x$ for some type $\sigma_x$.
%
%\item It is $\{ x : \sigma_x ~ | ~ x \in V(q_{\zeta_r}) \} \vdash_{\Sigma_{\langle prg \rangle}} t_{\zeta_r} : \sigma$.
%\end{enumerate}
%For the first point, we distinguish between the possible forms of $q_{\zeta_r}$. It can either be a function call, then we infer its type as the return type $\sigma$ of $aux$, given in the signature for $aux$, which is present in $\Sigma_{\langle prg \rangle}$. Or it can be a destructor call; by definition, $q_{\zeta_r}$ is the result of a (co-)substitution applied to $t_\epsilon$, and since $q_{\zeta_r}$ contains a destructor call and $t_\epsilon$ doesn't, the destructor must be a part of the co-substitution. Since $q_r \in T$ results from $q_\epsilon$ by applying this co-substitution to it, and $prg$ is well-typed, we know that the respective destructor signature is present in $\Sigma_{prg}$, and thus in $\Sigma_{\langle prg \rangle}$.
%
%Concerning the second point, first note that the variables of $q_{\zeta_r}$ are exactly those of $q_r$. The types inferred for the variables in $q_{\zeta_r}$ are identical to the types inferred for them in $q_r$. To see this, combine the facts that the argument types of $aux$ are defined to be the types inferred for the variables of $q_\epsilon$, the return type of $aux$ is defined to be the type inferred for $q_\epsilon$, and that $q_r$ is the result of applying a (co-)substitution to $q_\epsilon$. Thus we know the following. For each variable $x$ of $q_r$ and $q_{\zeta_r}$, let $\sigma_x$ be the type inferred for it in $prg$, i.e., it is $q_r \vdash_{\Sigma_{prg}} x : \sigma_x$. Since the types inferred for the variables in $q_{\zeta_r}$ are identical to the types inferred for them in $q_r$, it also is $q_{\zeta_r} \vdash_{\Sigma_{\langle prg \rangle}} x : \sigma_x$.
%
%For the third point, we use the fact that the types inferred for the variables of $q_{\zeta_r}$ and $q_r$ are identical in $prg$ and $\langle prg \rangle$, as established for the second point. Since $prg$ is well-typed, and thus it is $\Sigma_{prg} \vdash `` q_r = t_r " \textrm{ ok}$, it also is $\{ x : \sigma_x ~ | ~ x \in V(q_r) \} \vdash_{\Sigma_{prg}} t_r$, and by \autoref{fac:wtpfac} it is $\{ x : \sigma_x ~ | ~ x \in V(q_r) \} \vdash_{\Sigma_{\langle prg \rangle}} t_r$.

\begin{enumerate}
\item $\Sigma_{\langle prg \rangle} \vdash q_{\zeta_r} : \tau; \Gamma$ for some type $\tau$ and some typing context $\Gamma$, and

\item $\Gamma \vdash_{\Sigma_{\langle prg \rangle}} t_{\zeta_r} : \tau$.
\end{enumerate}
Since $prg$ is well-typed, by the derivation of the well-typedness for $r$ we know that $\Sigma_{prg} \vdash q_r : \tau'; \Gamma'$ for some type $\tau'$ and some typing context $\Gamma'$ with a type assignment for each variable in $q_r$; the same holds for $\Sigma_{\langle prg \rangle}$ by \autoref{fac:wtpfac}. Now, we consider the two points above in turn: We will show that they hold for $\Gamma = \Gamma'$ and $\tau = \tau'$.

For the first point, we note that it is $q_{\zeta_r} = \textsf{putback}(t_\epsilon, q_r) = \sigma^{q_r}_\pi(t_\epsilon)$, and we distinguish whether $\sigma^{q_r}_\pi$, from now on shortened to $\sigma$, is a substitution or a co-substitution.
\begin{itemize}
\item When $\sigma$ is a substitution, it is $q_{\zeta_r} = aux(\sigma(x_1), ..., \sigma(x_n))$. Since, by definition, $t_\epsilon$ has the same variables as $q_\epsilon$, and it is $\sigma(q_\epsilon) = q_r$, we know that each $\sigma(x_i)$ is a subterm of $q_r$ and that the variables of $q_{\zeta_r}$ are exactly those of $q_r$. From the derivation of $\Sigma_{\langle prg \rangle} \vdash q_r : \tau'; \Gamma'$ we know that it is $\tau_i \vdash_{\Sigma_{\langle prg \rangle}} \sigma(x_i) : \Gamma_i$ for some type $\tau_i$ and some typing context $\Gamma_i$. Each $\tau_i$ is the $i$-th argument type of $aux$ because these argument types were defined to be the types inferred for the variables in $q_\epsilon$ and it is $q_r = \sigma(q_\epsilon)$. Since the variables of $q_{\zeta_r} = \sigma(x_1), ..., \sigma(x_n)$ are exactly those of $q_r$, we know that any variable of $q_r$ is contained in some $\sigma(x_i)$, and thus it must be $\Gamma' = \Gamma_1, ..., \Gamma_n$. The function $aux$ has return type $\tau'$ since its return type is defined to be the return type of whatever function or destructor $q_\epsilon$ is a call to and $q_r = \sigma(q_\epsilon)$ is also a call to this same function or destructor. Apply the ``Fun'' rule for $aux$ on top of the derivations for the $\tau_i \vdash_{\Sigma_{prg}} \sigma(x_i) : \Gamma_i$ to derive $\Sigma_{\langle prg \rangle} \vdash q_{\zeta_r} : \tau'; \Gamma'$.

\item When $\sigma$ is a co-substitution, it is $q_{\zeta_r} = aux(x_1, ..., x_n).\overline{des(\overline{p})}$ and $q_r = q_\epsilon.\overline{des(\overline{p})}$. Derive $\Sigma_{\langle prg \rangle} \vdash q_{\zeta_r} : \tau'; \Gamma'$ as follows. Start with the ``Var'' rules for each $x_i$ to derive $\tau_i \vdash_{\Sigma_{\langle prg \rangle}} x_i : x_i:\tau_i$, where $\tau_i$ is the $i$-th argument type of $aux$; note that the argument types of $aux$ were defined to be the types inferred for the variables in $q_\epsilon$ and are thus identical to the types inferred for these variables by $\Sigma_{\langle prg \rangle} \vdash q_r : \tau'; \Gamma'$ (1). Then use the ``Fun'' rule and follow it up with the ``Des'' rule steps at the end of the derivation of $\Sigma_{\langle prg \rangle} \vdash q_r : \tau'; \Gamma'$. This infers the types for the rest of the variables in $q_r$ (i.e., those not in $\{x_1, ..., x_n\}$) identically to $\Sigma_{\langle prg \rangle} \vdash q_r : \tau'; \Gamma'$ (2). Combining (1) and (2) we know that we have derived $\Sigma_{\langle prg \rangle} \vdash q_{\zeta_r} : \tau'; \Gamma'$.
\end{itemize}

Now, consider the second point. From the derivation of the well-typedness of $r$ we also know that $\Gamma \vdash_{\Sigma_{prg}} t_r : \tau$; by \autoref{fac:wtpfac} it follows that $\Gamma \vdash_{\Sigma_{\langle prg \rangle}} t_r : \tau$. Since $t_{\zeta_r}$ was defined to be $t_r$, we have the desired statement.
\end{proof}
\end{lemma}

\chapter{Automatic de- and refunctionalization}

\section{Common pretransformations}

...

\subsection{From order-dependent to order-independent}

...

\subsection{Aligning patterns}

This step is necessary for destructor extraction to work correctly.

The purpose of $align\_patterns$ is to bring the patterns inside the copatterns within one function definition to the same level of specification. Consider the following example:

\begin{lstlisting}

fun().d1(c1()).d1() = t_1
fun().d1(c2()).d1() = t_2
fun().d1(x).d2() = t_3

\end{lstlisting}

Here, the third equation's left-hand side has a catch-all pattern (variable) in its first destructor call $d1$. The first destructor in the first and second equations' left-hand sides is also $d1$, but the patterns there are the constructors $c1()$ and $c2()$, respectively; they have a higher level of specification than the catch-all pattern. For destructor extraction to work correctly, the catch-all pattern therefore must be split. Such splitting steps are what $align\_patterns$ is doing. The result of $align\_patterns$ applied to the above function definition is:

\begin{lstlisting}

fun().d1(c1()).d1() = t_1
fun().d1(c2()).d1() = t_2
fun().d1(c1()).d2() = t_3
fun().d1(c2()).d2() = t_3

\end{lstlisting}

First, we formally define what it means for a (co-)pattern to be \textit{aligned} with another (co-)pattern.

\begin{definition}[Aligning patterns]
Two patterns $p_1, p_2$ of the same type align if they are (a) both variables, (b) both constructor calls of different constructors, or (c) constructor calls of the same constructor $con$ such that
\[
p_1 = con(p'_1, ..., p'_n), p_2 = con(p''_1, ..., p''_n),
\]
where, for all $i \in \{1, ..., n\}$, $p'_i$ and $p''_i$ align.
\end{definition}

\begin{definition}[Aligning copatterns]
Two copatterns $q_1, q_2$ of the same function definition, for a function $fun$, align if the respective arguments of their largest common destructor call chains align.
\end{definition}

Now, we define the algorithm that aligns patterns; we present a calculus for it. We write $q \rhd Q \searrow \Sigma$ when the algorithm produces a result $\Sigma$ for a copattern $q$ and a copattern set $Q$ that $q$ is to be aligned with. This $\Sigma$ is a set of substitutions $\sigma$ such that $q[\sigma]$ is aligned with all elements of $Q$ and such that $\bigcup_{\sigma \in \Sigma} q[\sigma]$ is equivalent to $q$. Similarly, we write $q \rhd q' \searrow \Sigma$ for the sub-algorithm used for aligning $q$ with an individual copattern $q'$.

\begin{prooftree}
\AxiomC{}
\RightLabel{QA\textsubscript{Empty}}
\UnaryInfC{$q \rhd \emptyset \searrow \mathbf{1}$}
\end{prooftree}

\begin{prooftree}
\AxiomC{$q \rhd Q \searrow \Sigma$}
\AxiomC{$q \rhd q' \searrow \Sigma'$}
\RightLabel{QA\textsubscript{Head}}
\BinaryInfC{$q \rhd q' : Q \searrow \Sigma' \circ \Sigma$}
\end{prooftree}

\begin{prooftree}
\AxiomC{$\overline{p \rhd p' \searrow \Sigma}$}
\RightLabel{qA}
\UnaryInfC{$fun(\overline{p}) \rhd fun(\overline{p'}) \searrow \prod_i \Sigma_i$}
\end{prooftree}

$p \rhd p' \searrow \Sigma$ means that the sub-algorithm for pattern alignment, presented below, produces the set of substitutions $\Sigma$.

\begin{prooftree}
\AxiomC{}
\RightLabel{PA\textsubscript{VarVar}}
\UnaryInfC{$x \rhd y \searrow \mathbf{1}$}
\end{prooftree}

\begin{prooftree}
\AxiomC{$\overline{y^{\mathit{fresh}} \rhd p \searrow \Sigma}$}
\RightLabel{PA\textsubscript{VarCon}}
\UnaryInfC{$x \rhd con^\tau(\overline{p}) \searrow \bigoplus_{con' \in \textrm{Cons}_\tau} x \mapsto con'(\overline{\Sigma(y^{\mathit{fresh}})})$}
\end{prooftree}

\begin{prooftree}
\AxiomC{}
\RightLabel{PA\textsubscript{ConVar}}
\UnaryInfC{$con(\overline{p}) \rhd x \searrow \mathbf{1}$}
\end{prooftree}

\begin{prooftree}
\AxiomC{$\overline{p \rhd p' \searrow \Sigma}$}
\RightLabel{PA\textsubscript{ConCon\textsubscript{$=$}}}
\UnaryInfC{$con(\overline{p}) \rhd con(\overline{p'}) \searrow \prod_i \Sigma_i$}
\end{prooftree}

\begin{prooftree}
\AxiomC{$con \neq con'$}
\RightLabel{PA\textsubscript{ConCon\textsubscript{$\neq$}}}
\UnaryInfC{$con(\overline{p}) \rhd con'(\overline{p'}) \searrow \mathbf{1}$}
\end{prooftree}

The following lemma captures that this algorithm can actually be used for its intended purpose.

\begin{lemma}
For any set of copatterns $Q$ and any two of its elements $q_1, q_2$ with $q_1 \rhd Q \searrow \Sigma_1, q_2 \rhd Q \searrow \Sigma_2$ for some $\Sigma_1, \Sigma_2$, all elements of $\Sigma_1(q_1)$ align with all elements of $\Sigma_1(q_2)$.
\end{lemma}

To prove this, we first show a lemma for the sub-algorithm for pattern alignment.

\begin{lemma}
For any two patterns $p_1, p_2$ with $p_1 \rhd p_2 \searrow \Sigma_1, p_2 \rhd p_1 \searrow \Sigma_2$ for some $\Sigma_1, \Sigma_2$, all elements of $\Sigma_1(p_1)$ align with all elements of $\Sigma_2(p_2)$.

\begin{proof}
By induction on the derivation of $p_1 \rhd p_2 \searrow \Sigma$.

\textbf{PA\textsubscript{VarVar}}: The only element of $\Sigma$ is the identity substitution, thus the only element of $\Sigma(p_1)$ is $p_1 = x$ itself, and it is $p_2 = y$. Because $p_1$ and $p_2$ are both variables, $p_2 \rhd p_1 \searrow \Sigma$ can only be derived by the rule PA\textsubscript{VarVar}. Like with the other derivation, we can conclude that the only element of $\Sigma(p_2)$ is $p_2 = y$ itself; it follows that elements of $\Sigma(p_1)$ align with all elements of $\Sigma(p_2)$ since both $p_1$ and $p_2$ are variables.

\textbf{PA\textsubscript{ConVar}}: The only element of $\Sigma$ is the identity substitution thus the only element of $\Sigma(p_1)$ is $p_1 = con(\overline{p})$ itself, and it is $p_2 = y$.

TODO: rest
\end{proof}
\end{lemma} 

Now, we get back to the proof of Lemma 4.1.1.

\begin{proof}[Proof of Lemma 4.1.1]
By induction on the derivation of $q_1 \rhd Q \searrow \Sigma_1$.

TODO
\end{proof}

Aligning of patterns is lifted to programs in the following, straightforward way; for this, let $Q_{def}$ denote the lhss of function definition $def$.
\begin{alignat*}{4}
\langle prg \rangle^{align\_patterns} & = &&\{ \textrm{\textbf{function }} fun(\tau_1, ..., \tau_n): \sigma \textrm{\textbf{ where }} \\
& &&\quad \{ `` q' = t " ~ | ~ q' \in \Sigma(q), q \rhd Q_{def} \searrow \Sigma, `` q = t " \in eqns \} \\
& && | ~ def = `` fun (\tau_1, ..., \tau_n): \sigma \textrm{\textbf{ where }} eqns " \in prg \} \\
& \cup && \{ def ~ | ~ def \in prg, def \textrm{ is (co)data def. } \} \span\span\span\span
\end{alignat*}

Aligning patterns of a program before applying destructor extraction guarantees that the destructor extraction doesn't introduce overlaps. As shown in subsection 4.2.2, this follows from the following lemma.

\begin{lemma}
For any program $prg$ without overlapping lhss, whenever two prefixes of lhss of $\langle prg \rangle^{align\_patterns}$ overlap, one of them is a prefix of the other.

\begin{proof}
Consider any function definition of $prg$. After transformation of $prg$ to $\langle prg \rangle^{align\_patterns}$, the function definition for the same function in the transformed program contains only lhss that align with all other lhss in the function definition. This follows directly from Lemma 4.1.1.

Now, suppose there are two lhss $q_1, q_2$ with prefixes $q^{\mathit{pref}}_1$ and $q^{\mathit{pref}}_2$, respectively, such that these prefixes overlap. Especially, this means that one of them -- w.l.o.g., $q^{\mathit{pref}}_1$ -- has an initial destructor chain $C$ that is identical to the entire destructor chain of the other, $q^{\mathit{pref}}_2$. Thus the entire destructor chain of $q^{\mathit{pref}}_2$ is a part of the largest common destructor chain of $q_1$ and $q_2$. It follows that all of the pattern arguments of the initial destructor chain $C$ of $q^{\mathit{pref}}_1$ align with the respective arguments of the destructor chain of $q^{\mathit{pref}}_2$. For the associated instances of these two destructor chains to overlap, it also needs to be the case that their respective pattern arguments overlap. Whenever two patterns overlap and align they are the same, thus $q^{\mathit{pref}}_2$ is a prefix of $q^{\mathit{pref}}_1$.
\end{proof}
\end{lemma}

\section{Transformation steps}

Both de- and refunctionalization are made up of a couple of preprocessing steps, followed by the core de-/refunctionalization, which is essentially the two-way transformation from the paper of Rendel et al.

Automatic defunctionalization consists of the following steps:
\begin{enumerate}
\item Eliminate multiple destructors. ($elim\_multi\_des_d$)

\item Unmix function definitions. ($unmix$)

\item Eliminate constructors from destructor copatterns. ($elim\_cons\_from\_des$)

\item Core defunctionalization. ($d_{core}$)

\end{enumerate}

Automatic refunctionalization consists of the following steps:
\begin{enumerate}
\item Eliminate multiple destructors from copatterns containing constructors. ($elim\_multi\_des_r$)

\item Unmix function definitions. ($unmix$)

\item Eliminate constructors from destructor copatterns. ($elim\_cons\_from\_des$)

\item Eliminate multiple constructors. ($elim\_multi\_con$)

\item Core refunctionalization. ($r_{core}$)

\end{enumerate}

Each preprocessing step is applied to each function definition individually. It is thus assumed that the function definition to be transformed is available to the transformation. For the complete step, simply transform all function definitions, in arbitrary order.

Each preprocessing steps can be defined as a recursive composition of one of the basic building blocks described in the next section. One such building block is an extraction function lifted to programs with $extract\_helpers$. The definitions for each step follow below; for them, the following conditional recursion combinator will be used:
\[
    condrec(f, cond) :=
\begin{cases}
    condrec(f, cond) \circ f,& \text{if $cond$ holds} \\
   id,& \text{otherwise}
\end{cases}
\]

Let $q^{work}$ denote any copattern which isn't yet in the desired form. For $unmix$, this means that it has a destructor, for all other steps this means that the condition passed to $condrec$ holds for the individual copattern. Let $n$ be the number of constructors in the targeted pattern (i.e., the extracted constructor is the right-most one).

\begin{framed}

\begin{alignat*}{1}
&elim\_multi\_des_i = condrec(liftp(extract\_des(q^{work})), cond_i), \textrm{ for } i \in \{d, r\}, \textrm{ with } \\
&\qquad cond_d = \textrm{``the function def. contains multiple destructors''} \\
&\qquad cond_r = \textrm{``the function def. contains a multiple destructor copattern with constructors''} \\
&unmix = condrec(liftp(extract\_des(q^{work})), cond), \textrm{ with}\\
&\qquad cond = \textrm{``the function def. is mixed''} \\
&elim\_cons\_from\_des = condrec(liftp(extract\_patterns(q^{work})), cond), \textrm{ with}\\
&\qquad cond = \textrm{``the function def. has destructor copatterns with constructors''} \\
&elim\_multi\_con = condrec(liftp(extract\_con_n(q^{work})), cond), \textrm{ with}\\
&\qquad cond = \textrm{``the function def. has copatterns with multiple constructors''}
\end{alignat*}

\end{framed}

As stated above, $r_{core}$ and $d_{core}$ are essentially two ways of the transformation of Rendel et al. Their precise definitions are given in section 3.6.

\subsection{Bisimulation}

All of the preprocessing steps are recursive compositions of lifted extraction functions. Each of those extraction functions preserve the semantics of the program in the kind of weak bisimulation described in the previous chapter, when the program doesn't have overlapping lhss.

For the bisimulation to hold, it is therefore necessary that each application of an extraction function in the preprocessing steps doesn't introduce overlapping lhss. This will be shown in the next subsection.

For core de-/refunctionalization, strong bisimulation holds. This is essentially a corollary of that property for the two-way transformation of Rendel et al., as will be shown in the section on core de- and refunctionalization.

\subsection{Absence of overlaps}

The three extraction functions used, $extract\_des$, $extract\_patterns$, and $extract\_con_n$ are shown to preserve the absence of overlapping lhss wherever they are applied in the preprocessing steps. By Proposition 2.4.1, it suffices to show that $q_\epsilon$ doesn't overlap with any lhs of an equation taken over unchanged from the original program.

\subsubsection{Destructor extraction}

By Lemma 4.1.1, the pre-transformation $align\_patterns$ ensures that, whenever two prefixes of lhss of the original program overlap, one of them is a prefix of the other. $q_\epsilon$ is a prefix of an lhs $q_r$ of the original program; if it overlapped with a $q_{r'}$ taken over unchanged then $q_{r'}$ (a prefix of itself) and the prefix $q_\epsilon$ of $q_r$ overlapped. But then, by Lemma 4.1.1, either (a) $q_\epsilon$ is a prefix of $q_{r'}$ or (b) $q_{r'}$ is a prefix of $q_\epsilon$. In case (a) it is either $q_\epsilon = q_{r'}$, but then $q_{r'}$ and $q_r$ overlapped contrary to assumption, or, since $q_{r'}$ doesn't have more destructors than $q_r$, $q_{r'}$ is identical to $q_r$ except for the last destructor call, but then $q_{r'}$ would also be targeted by the destructor extraction, contrary to assumption. In case (b) $q_{r'}$ is a prefix of $q_r$ and thus they overlap, again contrary to assumption.

\subsubsection{Pattern extraction}

The extraction $extract\_patterns$ is only ever applied to single-destructor copatterns and transforms them to single-destructor copatterns without any constructors. It is also only ever used in unmixed function definitions where all lhss have exactly one destructor. If the destructor of such an lhs $q$ is different from that of $q_\epsilon$, they cannot overlap. But if the destructor of $q$ is the same as that of $q_\epsilon$, and thus the same as that of $q_r$, the extraction result is the same for both $q_r$ and $q$. Thus, by the definition of extraction functions, the equations of $q_r$ and $q$ are the same.

\subsubsection{Constructor extraction}

The only transformed equations have no destructors on their lhs. Constructor extraction is only ever used with unmixed function definitions. It follows that, if $q_\epsilon$ overlapped with an unchanged lhs, the latter has no destructors, too. Since the extracted constructor is further right than the right-most constructor in the unchanged lhs, the unchanged lhs and $q_\epsilon$ would overlap, contrary to assumption.

\section{Core de- and refunctionalization}

\subsection{Core defunctionalization}

After the preprocessing steps, the only thing that remains is to apply the defunctionalization for the Codata Fragment of Uroboro, as developed by Rendel et al., to the not yet defunctionalized parts of the program. It can be applied to these parts because the preprocessing steps guarantee that they are in the Codata Fragment. Call the defunctionalization for the Data Fragment of Uroboro $d^{codata}$; the core defunctionalization for programs is defined as follows below.

\begin{alignat*}{3}
\langle prg \rangle^{d_{core}} & = ~&& \langle && \{ def \in prg ~ | ~ def \textrm{ is codata def. or} \\ & && &&\quad \textrm{ function def. with equations } eqns \neq \emptyset: \forall e \in eqns: e \textrm{ has destr. pattern } \} \rangle^{d^{codata}} \\
& \cup && \{ && \textrm{\textbf{data }} ... ~ | ~ `` \textrm{\textbf{data }} ... " \in prg \} \\
& \cup && \{ && \textrm{\textbf{function }} fun(\sigma, \tau_1, ..., \tau_k): \tau \textrm{\textbf{ where }} \{ p = \langle t \rangle^d ~ | ~ "p = t" \in eqns \} \\
& && | && `` \textrm{\textbf{function }} fun(\sigma, \tau_1, ..., \tau_k): \tau \textrm{\textbf{ where }} eqns " \in prg \textrm{ with } \forall e \in eqns: e \textrm{ has hole pattern}\} 
\end{alignat*}

Technical note on constructor subsumption:

The input of $d^{codata}$ in the definition above is actually not in its domain. This is because it can contain constructor calls. The following technical trick allows to transform such inputs as well: For the sake of $d^{codata}$, subsume constructor names under function names (as if they were from the same syntactic domain). After the transformation, since names aren't changed (or when name changes are desired, the original name can still be retrieved), the subsumed constructor names (or their equivalents after a name change) are once again considered constructor names (from the original syntactic domain).

Defunctionalizing terms: \\
$\langle x \rangle^d = x$ \\
$\langle s.des(t_1, ..., t_n) \rangle^d = \langle des \rangle^d (\langle s \rangle^d, \langle t_1 \rangle^d, ..., \langle t_n \rangle^d)$ \\
$\langle fun(t_1, ..., t_n) \rangle^d = \langle fun \rangle^d (\langle t_1 \rangle^d, ..., \langle t_n \rangle^d)$ \\
$\langle con(t_1, ..., t_n) \rangle^d = con(\langle t_1 \rangle^d, ..., \langle t_n \rangle^d)$ \\

\subsubsection{Proof of strong bisimulation}

For $d_{core}$, strong bisimulation holds. The proof relies on properties of $d^{codata}$. As stated in section 2.3.1, the authors' notion of reducibility is the same than that of this work when restricted to the domain of $d^{codata}$, the Codata Fragment.

In section 3, Rendel et al. prove Lemma 5, which in terms of this work can be stated as follows (possible since the reducibility notions are identical):

$s \longrightarrow_{prg} t \iff \langle s \rangle \longrightarrow_{\langle prg \rangle} \langle t \rangle$ for all input terms $s,t$ of $\langle \cdot \rangle$ (*)

Here, the angular brackets can stand for either of their transformations, the refunctionalization $r^{data}$ and the defunctionalization $d^{codata}$. Statement (*) means that strong bisimulation holds for $d^{codata}$.

Using (*), it will now be shown that strong bisimulation holds for $d_{core}$.

\begin{proof}[Proof of strong bisimulation for $d_{core}$] ~

$`` \Rightarrow "$: By induction on the structure of $\mathcal{D}$.

\begin{enumerate}
\item \textbf{``Subst" case}:

\begin{prooftree}
\AxiomC{$\mathcal{D}_{\textrm{PM}}$}
\UnaryInfC{$s =^? q \searrow \sigma$ with $(q, s') \in \textrm{Rules}(prg)$}
\UnaryInfC{$s \longrightarrow s'[\sigma]$}
\end{prooftree}

with $s'[\sigma] = t$; the immediate subterms of $s$ are values; $\mathcal{D}_{\textrm{PM}}$ is a derivation of the pattern matching. This transformation changes input terms, thus $\langle s \rangle = \langle s \rangle^d$, $\langle t \rangle = \langle t \rangle^d$. $d$ is the defunctionalization of terms as defined above. This defunctionalization of terms is also, for all input terms from the fragment, identical to that of the Codata Fragment.

\begin{itemize}

\item \underline{Case 1}: $q$ is hole pattern:

Then the function definition that contains $`` q = s' "$ contains only equations where the left-hand side is a hole pattern (other cases are excluded by the relevant input fragment for $d_{core}$). Such equations (and indeed the function definitions) are left unchanged by $d_{core}$ except for defunctionalizing the right-hand term, as can be seen directly in the definition of $d_{core}$ (last set in the highest-level union). Thus Rules($\langle prg \rangle$) contains $(q, \langle s' \rangle)$.

By inversion, we have from $s =^? q \searrow \sigma$ that $s$ has the form $fun(v_1, ..., v_n)$ for some values $v_1, ..., v_n$, thus $\langle s \rangle = fun(\langle v_1 \rangle, ..., \langle v_n \rangle)$. By inversion for values, we have that each $v_i$ is either a constructor application or a value of codata type. If it is a value of codata type, by inversion on pattern matching, the relevant subpattern of $q$ can only be a variable, thus it is also matched by $\langle v_i \rangle$. If it is a constructor application, the relevant subpattern of $q$ is either a variable, and the same holds, or it is a constructor pattern, and by recursively descending into its subpatterns we still get that $\langle v_i \rangle = con(\langle v^1_i \rangle, ..., \langle v^m_n \rangle)$ matches against the subpattern of $q$.

By carrying the substitutions returned from the matchings along in the above recursive argument, we get a substitution $\sigma'$ such that $\langle s \rangle =^? q \searrow \sigma'$ and, by distributing over $\langle s' \rangle$, $\langle s' \rangle [\sigma'] = \langle s'[\sigma] \rangle = \langle t \rangle$. It follows that $\langle s \rangle \longrightarrow_{\langle prg \rangle} \langle t \rangle$.

\item \underline{Case 2}: $q = fun(p_1, ..., p_n).des(p'_1, ..., p'_k)$:

Then the function definition that contains $`` q = s' "$ contains only equations where the left-hand side is a destructor pattern (other cases are excluded by the relevant input fragment for $d_{core}$). Thus $s$ reduces to $t$ already with respect to the part of the program that is passed to $d^{codata}$, as specified in the definition of $d_{core}$. Let this part, amended by the ``constructor subsumption" noted for the definition of $d_{core}$, be $prg'$; it is: $s \longrightarrow_{prg'} t$

By (*) we would have

\begin{equation*}
s \longrightarrow_{prg'} t \iff \langle s \rangle \longrightarrow_{\langle prg' \rangle^{d^{codata}}} \langle t \rangle,
\end{equation*}

were $prg'$ a well-typed program with copattern coverage for all subterms of $s$. 

For the coverage, bear in mind that the equation $`` q = s' "$ enabling the reduction of $s$ by the ``Subst" rule is part of $prg'$ by the precondition of Case 2. As $s$ matches against $q$, copattern coverage for $s$ is trivially fulfilled in $prg'$. The immediate subterms of $s$ are values with respect to $prg$ and, by inversion, their immediate subterms and so forth, which especially means that there is no rule in $prg$ against which they match. But $prg$ has copattern coverage for such a subterm (TODO: make this a general precondition) and there is already no rule for it in $prg$. It follows that $prg'$ still has copattern coverage for the subterm even though there is no rule for it in $prg'$. This is because, either (1) the subterm is a destructor call, then it can only be covered by destructor copatterns (as it matches against a destructor copattern) and those only occur within $prg'$, or (2) it is a constructor call, which doesn't need to be matched for coverage. It can't be a function call, since these can only be covered by directly matching the call, which isn't the case even in $prg$, for which coverage is assumed. Thus coverage holds for $prg'$.

For well-typedness, simply treat the missing types temporarily, that is, for the sake of (*), as codata types. This is no problem for the restriction to the domain of $d^{codata}$, since such types could be introduced inside the Codata Fragment with codata definitions. To be more precise, empty function definitions can be added for missing ones and empty codata definitions for missing types, and removed again after using (*), without adding or removing possible reductions, respectively. All in all, we have by (*):
\begin{equation*}
s \longrightarrow_{prg'} t \iff \langle s \rangle \longrightarrow_{\langle prg' \rangle^{d^{codata}}} \langle t \rangle
\end{equation*}

But this program $\langle prg' \rangle^{d^{codata}}$ is a subset of $\langle prg \rangle$, as can be seen in the definition of $d_{core}$. This implies the desired $\langle s \rangle \longrightarrow_{\langle prg \rangle} \langle t \rangle$.

\end{itemize}

Other cases are excluded by the relevant input fragment.

\item \textbf{``Cong" case}:

\begin{prooftree}
\AxiomC{$s' \longrightarrow t'$}
\RightLabel{Cong}
\UnaryInfC{$\mathcal{E}[s'] \longrightarrow \mathcal{E}[t']$}
\end{prooftree}

with $\mathcal{E}[s'] = s$ and $\mathcal{E}[t'] = t$.

By the induction hypothesis we have $\langle s' \rangle \longrightarrow_{\langle prg \rangle} \langle t' \rangle$. Let $\langle \mathcal{E} \rangle$ denote the transformation of $\mathcal{E}$, defined analogously to the transformation of terms by transforming the terms in $\mathcal{E}$ and by setting $\langle [] \rangle = []$. By applying the congruence rule we get $\langle \mathcal{E} \rangle[\langle s' \rangle] \longrightarrow_{\langle prg \rangle} \langle \mathcal{E} \rangle[\langle t' \rangle]$. It is clear that $\langle \mathcal{E} \rangle[\langle s' \rangle] = \langle \mathcal{E}[s'] \rangle = \langle s \rangle$ and $\langle \mathcal{E} \rangle[\langle t' \rangle] = \langle \mathcal{E}[t'] \rangle = \langle t \rangle$.

\end{enumerate}

$`` \Leftarrow "$: By induction on the structure of $\mathcal{D}$.

\begin{enumerate}
\item \textbf{``Subst" case}:

\begin{prooftree}
\AxiomC{$\mathcal{D}_{\textrm{PM}}$}
\UnaryInfC{$\langle s \rangle =^? q \searrow \sigma$ with $(q, s') \in \textrm{Rules}(\langle prg \rangle)$}
\UnaryInfC{$\langle s \rangle \longrightarrow_{\langle prg \rangle} s'[\sigma]$}
\end{prooftree}

with $s'[\sigma] = \langle t \rangle$; the immediate subterms of $\langle s \rangle$ are values; $\mathcal{D}_{\textrm{PM}}$ is a derivation of the pattern matching. This transformation changes input terms, thus $\langle s \rangle = \langle s \rangle^d$, $\langle t \rangle = \langle t \rangle^d$. $d$ is the defunctionalization of terms as defined above. This defunctionalization of terms is also, for all input terms from the fragment, identical to that of the Codata Fragment.

The equation $`` q = s' "$ can either be contained in that part of $\langle prg \rangle$ that results from the application of $d^{codata}$ to the relevant part of $prg$, as specified in the definition of $d''$, or it can be in the other part of $\langle prg \rangle$. As can be seen in the definition of $d''$, this other part is taken over unchanged from $prg$ except for defunctionalizing the right-hand terms. Thus for an equation $`` q = s' "$ from this part, the equation $`` q = s'' "$ with $s' = \langle s'' \rangle$ is present in $prg$. For such an equation, $q$ has hole pattern. It can then be easily seen that $s =^? q \searrow \sigma'$ for a $\sigma'$ with $s''[\sigma'] = t$ by an argument analogous to that of $`` \Rightarrow "$, ``Subst" case, Case 1.

Now, suppose that $`` q = s' "$ is contained in the part of $\langle prg \rangle$ that results from the application of $d^{codata}$ to the relevant part $prg' \subseteq prg$. Thus $\langle s \rangle \longrightarrow_{\langle prg' \rangle^{d^{codata}}} \langle t \rangle$.

By (*) we would have

\begin{equation*}
\langle s \rangle \longrightarrow_{\langle prg' \rangle^{d^{codata}}} \langle t \rangle \iff s \longrightarrow_{prg'} t,
\end{equation*}

were $prg'$ a well-typed program with copattern coverage for all subterms of $s$. Both of those properties can be shown or simulated similarly to the way they are in the $`` \Rightarrow "$ part.

But it is $prg' \subseteq prg$, as can be seen in the definition of $d''$. This implies the desired $s \longrightarrow_{prg} t$.

\item \textbf{``Cong" case}:

\begin{prooftree}
\AxiomC{$s' \longrightarrow_{\langle prg \rangle} t'$}
\RightLabel{Cong}
\UnaryInfC{$\mathcal{E}[s'] \longrightarrow \mathcal{E}[t']$}
\end{prooftree}

with $\mathcal{E}[s'] = \langle s \rangle$ and $\mathcal{E}[t'] = \langle t \rangle$.

By the induction hypothesis we have $s'' \longrightarrow_{prg} t''$ with $s' = \langle s'' \rangle$, $t' = \langle t'' \rangle$. Let $\langle \mathcal{E} \rangle$ denote the transformation of $\mathcal{E}$ (defined as in the $`` \Rightarrow "$ part). Apply the congruence rule to get $\mathcal{E}'[s''] \longrightarrow_{prg} \mathcal{E}'[t'']$ with $\mathcal{E} = \langle \mathcal{E}' \rangle$. That is, $\mathcal{E}'$ is the result of applying the inverse of $\langle \cdot \rangle$ to $\mathcal{E}$, which is possible, since, for instance, $\mathcal{E}[s'] = \langle s \rangle$. It is $\langle \mathcal{E}'[s''] \rangle = \langle \mathcal{E}' \rangle[\langle s'' \rangle] = \mathcal{E}[s'] = \langle s \rangle$ and $\langle \mathcal{E}'[t''] \rangle = \langle \mathcal{E}' \rangle[\langle t'' \rangle] = \mathcal{E}[t'] = \langle t \rangle$ and thus we have the desired $s \longrightarrow_{prg} t$.
\end{enumerate}

\end{proof}

\subsection{Core refunctionalization}

This is defined analogously to core defunctionalization, by applying the refunctionalization for the Data Fragment of Uroboro to the not yet refunctionalized parts of the program. It can be applied to these parts because the preprocessing steps guarantee that they are in the Data Fragment. Call the refunctionalization for the Data Fragment of Uroboro $r^{data}$; the core refunctionalization for programs is defined as follows below.

First, a technical note: As $r_{core}$ doesn't allow destructor terms in its inputs, they have to be converted beforehand. This conversion is the same as that of $r$ for terms below, restricted to destructor terms. Call this conversion lifted to programs (in the way that all destructor terms on right-hand sides or as subterms of them are converted) $des\_conv$.

\begin{alignat*}{3}
\langle prg \rangle^{r_{core}} & = ~&& \langle \langle && \{ def \in prg ~ | ~ def \textrm{ is data def. or} \\ & && &&\quad \textrm{ function def. with equations } eqns \neq \emptyset: \forall e \in eqns: e \textrm{ has no destr. pattern}, \\
& && &&\qquad \textrm{the first argument of the lhs isn't a variable } \} \rangle^{des\_conv} \rangle^{r_{core}} \\
& \cup && \{ && \textrm{\textbf{codata }} ... ~ | ~ `` \textrm{\textbf{codata }} ... " \in prg \} \\
& \cup && \{ && \textrm{\textbf{function }} fun(\tau_1, ..., \tau_n): \sigma \textrm{\textbf{ where }} \{ p = \langle t, prg \rangle^r ~ | ~ "p = t" \in eqns \} \\
& && | && `` \textrm{\textbf{function }} fun(\tau_1, ..., \tau_n): \sigma \textrm{\textbf{ where }} eqns " \in prg \textrm{ with } \forall e \in eqns: e \textrm{ has destr. pattern} \\
& && &&\quad \textrm{or where } n = 0 \textrm{ or where the first argument of the lhs is a variable} \} 
\end{alignat*}

Along with the transformation for programs, a transformation of terms is necessary, which is a conservative extension of $r^{data}$ for programs. For this, write $r$ short for $r_{core}$ \\
$\langle x, prg \rangle^r = x$ \\
$\langle s.des(t_1, ..., t_n), prg \rangle^r = \langle s, prg \rangle^r .des(\langle t_1, prg \rangle^r, ..., \langle t_n, prg \rangle^r)$ \\
$\langle fun(t_1, ..., t_n), prg \rangle^r = fun(\langle t_1, prg \rangle^r, ..., \langle t_n, prg \rangle^r)$, \\
if ``\textbf{function} $fun(\tau_n, ..., \tau_n): \sigma$ \textbf{where} $eqns$" $\in prg$  with $\forall e \in eqns: e$ has destructor pattern or where $n = 0$ or where the first argument of the lhs is a variable \\
$\langle fun(t_1, ..., t_n), prg \rangle^r = \langle t_1, prg \rangle^r .\langle fun, prg \rangle^r (\langle t_2, prg \rangle^r, ..., \langle t_n, prg \rangle^r)$, \\
otherwise \\
$\langle con(t_1, ..., t_n), prg \rangle^r = \langle con, prg \rangle^r (\langle t_1, prg \rangle^r, ..., \langle t_n, prg \rangle^r)$ \\

Note that the case distinction above is only necessary because of the special syntax for destructors ($q(...).des(...)$ instead of $des(..., ...)$).

\subsubsection{Proof of strong bisimulation}

For $r_{core}$, strong bisimulation holds. The proof relies on properties of $r^{data}$. As stated in section 2.3.1, the authors' notion of reducibility is the same than that of this work when restricted to the domain of $r^{data}$, the Data Fragment.

In section 3, Rendel et al. prove Lemma 5, which in terms of this work can be stated as follows (possible since the reducibility notions are identical):

$s \longrightarrow_{prg} t \iff \langle s \rangle \longrightarrow_{\langle prg \rangle} \langle t \rangle$ for all input terms $s,t$ of $\langle \cdot \rangle$ (*)

Here, the angular brackets can stand for either of their transformations, the refunctionalization $r^{data}$ and the defunctionalization $d^{codata}$. Statement (*) means that strong bisimulation holds for $r^{data}$.

Using (*), it will now be shown that strong bisimulation holds for $r_{core}$.

\begin{proof}[Proof of strong bisimulation for $r_{core}$] ~

$`` \Rightarrow "$: By induction on the structure of $\mathcal{D}$.

\begin{enumerate}
\item \textbf{``Subst" case}:

\begin{prooftree}
\AxiomC{$\mathcal{D}_{\textrm{PM}}$}
\UnaryInfC{$s =^? q \searrow \sigma$ with $(q, s') \in \textrm{Rules}(prg)$}
\UnaryInfC{$s \longrightarrow s'[\sigma]$}
\end{prooftree}

with $s'[\sigma] = t$; the immediate subterms of $s$ are values; $\mathcal{D}_{\textrm{PM}}$ is a derivation of the pattern matching. This transformation changes input terms, thus $\langle s \rangle = \langle s \rangle^r$, $\langle t \rangle = \langle t \rangle^r$. $r$ is the refunctionalization of terms as defined above (it is omitted that $prg$ is passed to $r$ as well). This refunctionalization of terms is also, for all input terms from the fragment, identical to that of the Data Fragment.

\begin{itemize}

\item \underline{Case 1}: $q$ is destructor pattern:

Then the function definition that contains $`` q = s' "$ contains only equations where the left-hand side is a destructor pattern (other cases are excluded by the relevant input fragment for $r_{core}$). Such equations (and indeed the function definitions) are left unchanged by $r_{core}$ except for refunctionalizing the right-hand term, as can be seen directly in the definition of $r_{core}$ (last set in the highest-level union). Thus Rules($\langle prg \rangle$) contains $(q, \langle s' \rangle)$.

From here, the argument proceeds analogously to that of $`` \Rightarrow "$, ``Subst" case, Case 1, in the proof for $d_{core}$.

\item \underline{Case 2}: $q$ is hole pattern without arguments or where the first argument is a variable:

Then the equation is left unchanged by $r_{core}$ except for refunctionalizing the right-hand term, as can be seen directly in the definition of $r_{core}$ (last set in the highest-level union). Proceed as in Case 1.

\item \underline{Case 3}: $q$ is hole pattern and has a first argument which is a constructor pattern:

Then the function definition that contains $`` q = s' "$ contains only equations where the left-hand side is a hole pattern (other cases are excluded by the relevant input fragment for $r_{core}$), and it has a first argument with data type. Thus $s$ reduces to $t$ already with respect to the part of the program that is passed to $des\_conv$, and then the result of this to $r^{data}$, as specified in the definition of $r_{core}$. Let the part passed to $des\_conv$ be $prg'$; it is: $s \longrightarrow_{prg'} t$.

By (*) we have

\begin{equation*}
s \longrightarrow_{prg'} t \iff \langle s \rangle \longrightarrow_{\langle prg' \rangle^{r^{data}}} \langle t \rangle,
\end{equation*}

But this program $\langle prg' \rangle^{r^{data}}$ is a subset of $\langle prg \rangle$, as can be seen in the definition of $r_{core}$. Thus we have the desired $\langle s \rangle \longrightarrow_{\langle prg \rangle} \langle t \rangle$.

\end{itemize}

\item \textbf{``Cong" case}:

The argument here is identical to that of this case of this direction in the proof for $d_{core}$.

\end{enumerate}

$`` \Leftarrow "$: By induction on the structure of $\mathcal{D}$.

\begin{enumerate}
\item \textbf{``Subst" case}:

\begin{prooftree}
\AxiomC{$\mathcal{D}_{\textrm{PM}}$}
\UnaryInfC{$\langle s \rangle =^? q \searrow \sigma$ with $(q, s') \in \textrm{Rules}(\langle prg \rangle)$}
\UnaryInfC{$\langle s \rangle \longrightarrow_{\langle prg \rangle} s'[\sigma]$}
\end{prooftree}

with $s'[\sigma] = \langle t \rangle$; the immediate subterms of $\langle s \rangle$ are values; $\mathcal{D}_{\textrm{PM}}$ is a derivation of the pattern matching. This transformation changes input terms, thus $\langle s \rangle = \langle s \rangle^r$, $\langle t \rangle = \langle t \rangle^r$. $r$ is the refunctionalization of terms defined above. This refunctionalization of terms is also, for all input terms from the fragment, identical to that of the Data Fragment.

The equation $`` q = s' "$ can either be contained in that part of $\langle prg \rangle$ that results from the application of $des\_conv$ and then $r^{data}$ to the relevant part of $prg$, as specified in the definition of $r_{core}$, or it can be in the other part of $\langle prg \rangle$. As can be seen in the definition of $r_{core}$, this other part is taken over unchanged from $prg$ except for refunctionalizing the right-hand terms. Thus for an equation $`` q = s' "$ from this part, the equation $`` q = s'' "$ with $s' = \langle s'' \rangle$ is present in $prg$. For such an equation, $q$ has hole pattern. It can then be easily seen that $s =^? q \searrow \sigma'$ for a $\sigma'$ with $s''[\sigma'] = t$ by an argument analogous to that of $`` \Rightarrow "$, ``Subst" case, Case 1, in the proof for $d_{core}$.

Now, suppose that $`` q = s' "$ is contained in the part of $\langle prg \rangle$ that results from the application of $des\_conv$ and then $r^{data}$ to the relevant part $prg' \subseteq prg$. Thus $\langle s \rangle \longrightarrow_{\langle \langle prg' \rangle^{des\_conv} \rangle^{r^{data}}} \langle t \rangle$.

By (*) we have

\begin{equation*}
\langle s \rangle \longrightarrow_{\langle \langle prg' \rangle^{des\_conv} \rangle^{r^{data}}} \langle t \rangle \iff s \longrightarrow_{\langle prg' \rangle^{des\_conv}} t.
\end{equation*}

In the result of $des\_conv$, no new matching left-hand sides are added. That is, $prg'$ contains at least all the matching left-hand sides that $\langle prg' \rangle^{des\_conv}$ has. Thus any reduction that is possible with respect to $\langle prg' \rangle^{des\_conv}$ is already possible with respect to $prg'$.

But it is $prg' \subseteq prg$, as can be seen in the definition of $r_{core}$. This implies the desired $s \longrightarrow_{prg} t$.
\end{enumerate}

\item \textbf{``Cong" case}:

The argument here is identical to that of this case of this direction in the proof for $d_{core}$.

\end{proof}

%\section{Simplifying copatterns (Alternative approach to de- and refunc.}
%
%Both de- and refunctionalization are made up of two major parts:
%\begin{enumerate}
%\item First, destructor and constructor extractions alternate to transform the program into a form which can be used by the second part.
%
%\item This second part is the core de-/refunctionalization, which is essentially the two-way transformation from the paper of Rendel et al.
%\end{enumerate}
%
%The \textit{simplifying} part of the defunctionalization transformation is made up of destructor and constructor extraction steps and stops when the program is in the input fragment of core defunctionalization, described below. The simplifying part of the refunctionalization transformation is defined like that, with the only difference being that it stops when the program is in the input fragment of core refunctionalization, also described below.
%
%This simplification is done individually for each function definition, the order in which the function definitions are transformed is unimportant. For one function definition $def$, the algorithm is defined below.
%
%\[
%  \langle prg \rangle^{simplify(def)}=\begin{cases}
%               prg, &\text{ if $def$ is in the desired fragment}\\
%               \langle prg \rangle^{simplify\_step(def)} \rangle^{simplify(def)}, &\text{ otherwise}
%            \end{cases}
%\]
%
%\[
%  \langle prg \rangle^{simplify\_step(def)}=\begin{cases}
%               \langle prg \rangle^{liftp(des\_extract(q^{max}_{def}))}, \\
%               \qquad\text{ if } \langle prg \rangle^{liftp(con_{n_{def}}\_extract(q^{max}_{def}))} \text{ has overlaps}\\
%               \langle prg \rangle^{liftp(con_{n_{def}}\_extract(q^{max}_{def}))},\\
%               \qquad\text{ otherwise}
%            \end{cases}
%\]
%
%with
%
%\[
%q^{max}_{def} = \textrm{max}_{\# con.} \textrm{max}_{\# des.} \{q ~ | ~ q \text{ is lhs in $def$ } \}
%\]
%
%and $n_{def}$ the number of the inner-most constructor in $q^{max}_{def}$ which has a variable ``in its place'' in another lhs of $def$. For a pattern $p$ of a copattern $q$ to be ``in the place'' of another pattern $p'$ in another copattern $q'$ means:
%\begin{itemize}
%\item When $p$ is a subterm of the $n$-th pattern immediately under $q$, then $p'$ is a subterm of the $n$-th pattern immediately under $q$,
%
%\item the same for the $m$-th pattern immediately under the $n$-th patterns if $p$ and/or $p'$ aren't the $n$-th patterns themselves,
%
%\item and so on recursively.
%\end{itemize}
%
%As can be seen in the definition of $simplify\_step$, the algorithm switches from destructor to constructor extraction whenever constructor extraction would produce overlapping lhss. This actually prevents overlaps, because, whenever constructor extraction would produce overlaps, destructor extraction doesn't, as will be shown below.
%
%Clearly, this also means that the algorithm eventually arrives at a function definition without any destructors and without any constructors, unless it stops before that, thus ensuring that the desired fragment will be reached in any case.
%
%\subsection{Bisimulation}
%
%Two properties are desired for this algorithm: some kind of bisimulation, and that no overlapping lhss are generated. Since the algorithm is made up only of $des\_extract$ and $con\_extract$ steps, these properties follow from their respective properties.
%
%When overlapping lhss are absent in the transformed program, we have the kind of weak bisimulation as established by Proposition 2.3.1 for both $des\_extract$ and $con\_extract$, irrespective of which lhs is targeted. That overlaps aren't generated by one such step will be shown in the following subsection.
%
%\subsection{Absence of overlaps}
%
%By Proposition 2.4.1, for both $des\_extract$ and $con\_extract$, it suffices to show that $q_\epsilon$ doesn't overlap with unchanged lhss. This doesn't hold for arbitrary targeted lhss; in order to avoid generating overlaps, $simplify\_step$ was defined as above. It is now shown that this definition really prevents overlaps in the resulting program of one such step. As pointed out above, because of the definition of $simplify\_step$, it suffices to show that destructor extraction doesn't produce overlaps whenever constructor extraction does.
%
%When constructor extraction leads to overlaps, ...

\chapter{Related and future work}

In this chapter, we consider related and future work. We begin with several use cases for our transformations (\autoref{sec:usecases}). Then we review work that our thesis builds upon: the (principal idea for the) language Uroboro (\autoref{sec:reluro}), and the unnesting algorithm of Setzer et al. (\autoref{sec:relunn}). Finally, we talk about avenues for future research (\autoref{sec:futr}).

\section{Use cases}
\label{sec:usecases}

As already hinted at in the introduction, authors like Reynolds and Danvy have shown how de- and refunctionalization, along with other transformations like CPS transformation, can be used to automatically transform programs of a simpler form into semantically equivalent programs with certain desirable properties, and the other way around. In this section, we flesh this out somewhat and connect it to our automatic transformations. First, we talk about how our transformations apply to Reynolds' meta-circular interpreter example (\autoref{ssec:mci}). 

\subsection{The meta-circular interpreter}
\label{ssec:mci}

Reynolds' classical example is the ``meta-circular'' interpreter for the lambda calculus: Instead of writing an interpreter in a first-order language from scratch, he first straightforwardly translates the intuitive understanding of the semantics of the lambda calculus into a higher-order program, then mechanically defunctionalizes this program. The other direction, refunctionalization into a program that can be better understood than its first-order counterpart, has first been considered by Danvy et al.

Rendel et al. have already shown how Uroboro facilitates both the defunctionalization direction and the refunctionalization direction; they have also done so specifically for the meta-circular interpreter example. But their approach still required manual, even though mechanical, work to bring programs into either the form required by their defunctionalization or their refunctionalization. Our work fully automatizes these pretransformations; all in all, we give an algorithm to defunctionalize any Uroboro program with copattern coverage, and one that refunctionalizes any such program. Applied to the meta-circular interpreter example, a programmer can now start with an easier to understand higher-order program, like Reynolds did, and then literally ``press a button'' to get the corresponding first-order program.

...

\section{Uroboro}
\label{sec:reluro}

In the introduction, we have already considered the Data Fragment and Codata Fragment of Uroboro and how they relate to re- and defunctionalization, respectively. We also have already talked about why Uroboro is interesting for the purpose of automatic program transformations \autoref{ssec:urofull}. In this section, we examine why Uroboro is interesting beyond that: As already outlined in \autoref{ssec:urofull}, we think that Uroboro is a potential replacement for certain higher-order languages; in the following we illustrate this point.

For instance, consider the \texttt{filterNats} example, repeated from \autoref{ssec:defunc}. We will desugar it to a form which only uses codata type definitions, but no first-class functions.

\begin{lstlisting}

filterNats :: ((Nat -> Bool), [Nat]) -> [Nat]
filterNats (f, x:xs)
  | f x = x:(filterNats (f, xs))
  | otherwise = filterNats (f, xs)
filterNats (_, []) = []

even :: Nat -> Bool
even Zero = True
even Succ(n) = not (even n)

main :: [Nat]
main = filterNats (even, [1, 2, 3, 4, 5])

\end{lstlisting}

We first turn the Haskell-like syntax into a syntax closer to Uroboro.

\begin{lstlisting}

function filterNats((Nat -> Bool), [Nat]): [Nat] where
  filterNats (f, x:xs)
    | f x = x:(filterNats (f, xs))
    | otherwise = filterNats (f, xs)
  filterNats (_, []) = []

function even(): Nat -> Bool where
  even Zero = True
  even Succ(n) = not (even n)

function main(): [Nat]
  main = filterNats (even, [1, 2, 3, 4, 5])

\end{lstlisting}

Then, we do the following:
\begin{itemize}
\item Add a codata type definition \texttt{NatBoolFun} with one destructor \texttt{apply}.

\item Replace function type \texttt{Nat -> Bool} with codata type \texttt{NatBoolFun}.

\item Desugar the calls to a function of type \texttt{NatBoolFun} to calls to destructor \texttt{apply}.
\end{itemize}
The result of this is shown below.

\begin{lstlisting}

codata NatBoolFun where
  NatBoolFun.apply(Nat): Bool

function filterNats(NatBoolFun, [Nat]): [Nat] where
  filterNats (f, x:xs)
    | f.apply(x) = x:(filterNats (f, xs))
    | otherwise = filterNats (f, xs)
  filterNats (_, []) = []

function even(): NatBoolFun where
  even.apply(Zero) = True
  even.apply(Succ(n)) = not (even.apply(n))

function main(): [Nat]
  main = filterNats (even(), [1, 2, 3, 4, 5])

\end{lstlisting}

For this simple example, it is no problem that we need to introduce the codata type \texttt{NatBoolFun}. However, since we don't have parametric polymorphism, we would have to introduce a codata type for \textit{every} concrete function type. The lack of parametric polymorphism therefore is a rather severe limitation. Rendel et al. are currently working on bringing parametric polymorphism to Uroboro.

\section{Unnesting}
\label{sec:relunn}

TODO: summarize important points of \autoref{ssec:unntransl}

\section{Future research}
\label{sec:futr}

Finally, we consider possible future research avenues. Uroboro is intended to not just be yet another language, but rather a programme with the goal of bringing useful automatic refactorings to higher-order programs. This programme has both theoretical and practical aspects to it. For both, we give some ideas where future research, starting from our work, could lead to.

On the practical side, our transformations can be implemented in an IDE. In such an implementation, on the one hand, the user might be able to interactively carry out individual extraction refactorings. Speaking in the terminology we developed in chapter 3, the user can select a target set of equations, then, just like in other IDEs, he can be informed whether the extraction can be carried out without problems. Such problems can be the introduction of overlaps, or that the selected equations aren't actually a target set in the sense of chapter 3. On the other hand, the user might also request to de- or refunctionalize the program or some of its function definitions. In this case, as described in chapter 4, the extractions necessary for preprocessing are automatically chosen by the IDE. It might also be useful to display to the user whether the program is in a certain desired fragment of Uroboro, like the defunctionalized or refunctionalized fragment, or whether it is coverage complete.

On the theoretical side, we have shone a light on the asymmetry between constructors and destructors in Uroboro in \autoref{sssec:asym}. The underlying problem directly relates to the difference between natural deduction and arbitrary sequent calculus, and it is not exclusive to Uroboro, but rather affects all current languages with copattern matching. Thus it might be beyond the principal scope of the Uroboro programme, but nonetheless an interesting research problem. In the area of object-oriented programming there already exists quite some work on the somewhat analogous topic of \textit{multimethods}. ...
% !TEX root = main.tex
\chapter{Related and future work}
\label{ch:rel}

In this chapter, we consider related and future work. We begin with several example uses for de- and refunctionalization, as they have been considered by previous authors (\autoref{sec:derefuncex}), somewhat continuing our discussion from \autoref{sec:appl}. Then we review work that our thesis builds upon: the unnesting algorithm of Setzer et al. (\autoref{sec:relunn}), and the (principal idea for the) language Uroboro (\autoref{sec:reluro}). Finally, we talk about avenues for future research (\autoref{sec:futr}).

\section{De- and refunctionalization examples}
\label{sec:derefuncex}

As already hinted at in the introduction, authors like Reynolds and Danvy have shown how de- and refunctionalization, along with other transformations like CPS transformation, can be used to automatically transform programs of a more (human-) understandable form into semantically equivalent programs with certain desirable properties with regards to computation, and the other way around. In this section, we flesh this out somewhat and connect it to our automatic transformations; we first consider defunctionalization examples in \autoref{ssec:defuncex}, then refunctionalization examples in \autoref{ssec:refuncex}.

\subsection{Defunctionalization examples}
\label{ssec:defuncex}

We have already discussed Reynolds' classical example, the meta-circular interpreter for the lambda calculus, when we presented applications for our transformations. Danvy and Nielsen\cite{danvy01defunctionalization} explore other applications of defunctionalization, among them defunctionalization of programs processing lists, and of CPS transformed programs. Here, we would just like to pick out a result of their work that concerns a relation between CPS transformations and defunctionalization, which motivates future work on automatic transformations for Uroboro. As Danvy and Nielsen succinctly put it:

\blockcquote[20]{danvy01defunctionalization}{Defunctionalizing a CPS-transformed first-order program provides a systematic way to construct an iterative version of this program that uses a push-down accumulator. One can then freely change the representation of this accumulator.}

In a toolbox for transforming Uroboro programs, one might therefore want to have automatic CPS transformations to compliment the automatic de- and refunctionalization of this work.

\subsection{Refunctionalization examples}
\label{ssec:refuncex}

Danvy and Millikin\cite{danvy09refunctionalization} present two worked-out applications for refunctionalization, a recognizer for Dyck words and Dijktra's shunting yard algorithm; in addition, they point to three more applications for it without giving details. First, we exemplarily consider the recognizer for Dyck words, i.e., well-matched words over the alphabet of two parentheses. We have already discussed how our automatic refunctionalization applies to this example in \autoref{ssec:dyck}. Here, we concentrate on how refunctionalization brings the recognizer into a higher-level form that is easier understood by humans, which is a purpose dual to that of defunctionalization. Afterwards, we relate one of the preprocessings used by Danvy and Millikin to our transformations.

The recognizer for Dyck words, as presented by Danvy and Millikin, is an implementation of a tail-recursive push-down automaton. The stack of this push-down automaton, where ``open brace'' symbols increase the stack and ``close brace'' symbols decrease the stack, is not explicitly present anymore after refunctionalization. This is because the language processor of the higher-order language (in Danvy and Millikin's example, ML) deals with this stack in the background. This demonstrates that the purpose of refunctionalization is dual to that of defunctionalization: Refunctionalization turns programs which are closer to the representation used by the machine, e.g., by explicitly using a stack, into a form which abstracts away such low-level concepts. Note also that after the refunctionalization, the recognizer for Dyck words of Danvy and Millikin is in continuation-passing style. To make it even more human-readable, a transformation to direct style can be helpful. One logical next step in the development of the Uroboro toolbox therefore is, along with the addition of CPS transformations, the addition of direct style transformations.

All of the applications have in common that some preprocessing is necessary to bring them into the form required by their actual refunctionalization.\footnote{As already outlined in the introduction, this refunctionalization eliminates the data types for what are intended to be first-class functions and the apply function by replacing these data type definitions with abstractions that take their content from the respective equations of the apply function, and by replacing calls to the apply function with calls to the first-class function.} One of these preprocessings is called ``disentangling'', and it is used to split up a function that dispatches over more than one of its parameters. In our work, the transformation that corresponds to this is constructor extraction. The following \autoref{sec:relunn} goes into more detail how ``disentangling'' is dealt with in our work. At the end of \autoref{sec:relunn} we give some more consideration to what we contribute with the general notion of extractions.

\section{Unnesting}
\label{sec:relunn}

Our de- and refunctionalization is basically just the de- and refunctionalization of Rendel et al.\cite{rendel15automatic} together with a preprocessing. This preprocessing is the ``unnesting'' algorithm of Setzer et al.\cite{setzer14unnesting} for their copattern language, adapted for Uroboro. We first consider general properties of the unnesting algorithm, and then consider its adaptation to Uroboro.

Setzer et al.'s\cite{setzer14unnesting} unnesting operates on the copattern coverage trees of the function definitions. It is the natural counterpart to the interactive construction of a coverage-complete function definition by refinement, as can be done, e.g., in Agda. ``Unnesting'' essentially means undoing the steps of the derivation of the copattern coverage; the algorithm proceeds step-wise and distinguishes a case for each rule the final derivation step could have used. In this way, it can be guaranteed that an unnesting step always leads to a program with copattern coverage and without overlaps.

The drawback of the unnesting algorithm is that it requires a derivation for the copattern coverage. This has to be established in some way; as already mentioned in \autoref{sec:cc}, we aren't aware of any algorithm for copattern coverage that has better than exponential worst-case time complexity. We have also given some consideration to how to unnest function definitions without knowing the derivation of their copattern coverage. But this always lead us to the problem of how to avoid introducing overlaps; we were only able to avoid these at the cost of exponential worst-case time complexity. Thus, in the end, we decided to adapt the algorithm of Setzer et al.; it is natural in the sense described in the previous paragraph and therefore, in our view, the best algorithm for the task when one presupposes that the derivation of the copattern coverage is known. Our tentative suggestion is that, at least in more cases than is the case today, programmers could construct their programs interactively; this way, the programmer itself would provide the unnesting algorithm with the coverage derivation it needs. As interactive program construction can help avoid simple mistakes, this approach also need not necessarily be to the disadvantage of the programmer.

When we say that we adapted the unnesting algorithm to Uroboro, we mean that we simplified the algorithm from a language with first-class functions to a language without first-class functions. Not counting the initial C\textsubscript{Head} rule, there are only two coverage derivation rules for Uroboro, compared to the five rules of Abel et al.; consequently, the cases to be considered in the unnesting are also only two. Because Uroboro doesn't have function types, the coverage derivation rules needn't deal with application. The ``Head'' and ``App'' rules of Abel et al. are thus combined into one rule C\textsubscript{Head} for Uroboro, and only one rule for result splitting remains, namely C\textsubscript{Des}. Uroboro has fixed size argument lists, therefore all of Abel et al.'s rules for variable splitting are combined into one rule for Uroboro; the separate rules for pairs and units aren't necessary anymore. All in all, two coverage derivation rules remain to be considered in the unnesting algorithm: one for result splitting (C\textsubscript{Des}) and one for variable splitting (C\textsubscript{Var}). For the first, the corresponding ``undo'' transformation is destructor extraction, and for the second it is constructor extraction.

Destructor and constructor extraction are special cases of the general concept which we presented as ``extractions'' in \autoref{ch:extr}. With the notion of extraction, we have identified and generalized a concept which was previously known, but had appeared in various places and under various names, e.g. as two of the cases of the unnesting algorithm of Setzer et al.\cite{setzer14unnesting} or as ``disentangling'' in Danvy and Millikin\cite{danvy09refunctionalization}. The form of the definition of extractions we presented is specific to Uroboro, but the underlying idea is not. We believe that consolidating the knowledge of a topic, as opposed to having it pop up in several contexts where it isn't clear that the underlying concept is the same, is useful for its further exploration. Therefore, concerning the topic of what we decided to call extractions, we think that our work contributes at least the impulse towards such a consolidation.

\section{Uroboro}
\label{sec:reluro}

\begin{figure}
\begin{lstlisting}
  
data Nat where
  Zero: Nat
  Succ: Nat -> Nat

codata Stream where
  Stream.head: Nat
  Stream.tail: Stream

def zipWith: ((Nat, Nat) -> Nat, Stream, Stream) -> Stream where
  (zipWith (f, s1, s2)).head = f (s1.head, s2.head)
  (zipWith (f, s1, s2)).tail = zipWith (f, s1.tail, s2.tail)

def add: (Nat, Nat) -> Nat where
  add (a, Zero) = a
  add (a, Succ b) = Succ (add (a, b))

def fib: Stream where
  fib.head = Zero
  fib.tail.head = Succ Zero
  fib.tail.tail = zipWith (add, fib, fib.tail)

\end{lstlisting}
\caption{Uroboro-like higher-order program}
\label{fig:ch5uro1}
\end{figure}

\begin{figure}
\begin{lstlisting}

data Nat where
  Zero(): Nat
  Succ(Nat): Nat

codata Stream where
  Stream.head(): Nat
  Stream.tail(): Stream

codata Fun1 where
  Fun1.apply(Nat, Nat): Nat

codata Fun2 where
  Fun2.apply(Fun1, Stream, Stream): Stream

function zipWith(): Fun2 where
  zipWith.apply(f, s1, s2).head() = f.apply(s1.head(), s2.head())
  zipWith.apply(f, s1, s2).tail() =
    zipWith.apply(f, s1.tail(), s2.tail())

function add(): Fun1 where
  add().apply(a, Zero()) = a
  add().apply(a, Succ(b)) = Succ(add().apply(a, b))

function fib(): Stream where
  fib().head() = Zero()
  fib().tail().head() = Succ(Zero())
  fib().tail().tail() = zipWith(add(), fib(), fib().tail())

\end{lstlisting}
\caption{Codata conversion of \autoref{fig:ch5uro1}}
\label{fig:ch5uro2}
\end{figure}

\begin{figure}
\begin{lstlisting}

codata Fun<A, B> where
  Fun<A, B>.apply(A): B

data Nat where
  Zero(): Nat
  Succ(Nat): Nat

codata Stream where
  Stream.head(): Nat
  Stream.tail(): Stream

function zipWith():
    Fun<(Fun<(Nat, Nat), Nat>, Stream, Stream), Stream> where
  zipWith().apply((f, s1, s2)).head() =
    f.apply((s1.head(), s2.head()))
  zipWith().apply((f, s1, s2)).tail() =
    zipWith().apply((f, s1.tail(), s2.tail()))

function add(): Fun1 where
  add().apply((a, Zero())) = a
  add().apply((a, Succ(b))) = Succ(add().apply((a, b)))

function fib(): Stream where
  fib().head() = Zero()
  fib().tail().head() = Succ(Zero())
  fib().tail().tail() = zipWith().apply((add(), fib(), fib().tail()))

\end{lstlisting}
\caption{Parametric polymorphism for \autoref{fig:ch5uro2}}
\label{fig:ch5uro3}
\end{figure}

In the introduction, we have already considered the Data Fragment and Codata Fragment of Uroboro and how they relate to re- and defunctionalization, respectively. We also have already talked about why Uroboro is interesting for the purpose of automatic program transformations. In this section, we examine why Uroboro is interesting beyond that: As already outlined in the introduction, we think that Uroboro is a potential replacement for certain higher-order languages; in the following we illustrate this point.

As an example, consider the definition of Fibonacci streams, borrowed from Abel et al.\cite{abel13copatterns}. In \autoref{fig:ch5uro1}, we present this in a hypothetical higher-order language meant to resemble Uroboro.

We will desugar this to a form which only uses codata type definitions, but no first-class functions. To achieve this, we do the following:
\begin{itemize}
\item Add codata type definitions \texttt{Fun1} and \texttt{Fun2} with one destructor \texttt{apply} for each. The argument and result types of \texttt{apply} correspond to the respective function type that is replaced by the codata type in the next step.

\item Replace function types \texttt{(Nat, Nat) -> Nat} and \texttt{((Nat, Nat) -> Nat, Stream, Stream) -> Stream} with codata types \texttt{Fun1} and \texttt{Fun2}, respectively.

\item Desugar the calls to a function of type \texttt{Fun1} or \texttt{Fun2} to calls to the respective destructor \texttt{apply}.
\end{itemize}
We also change the concrete syntax of the original program, removing higher-order constructs such that the resulting program, shown in figure \autoref{fig:ch5uro2}, is valid Uroboro. This is possible because there are no more function types and first-class functions.

For this simple example, it is not much of a problem that we need to introduce two codata types for two different function types. However, since we don't have parametric polymorphism, we would have to introduce a codata type for \textit{every} concrete function type. The lack of parametric polymorphism therefore is a rather severe limitation. Rendel et al. are currently working on bringing parametric polymorphism to Uroboro. We imagine a version of the above program using parametric polymorphism to look similar what is shown in figure \autoref{fig:ch5uro2}; we use arrow brackets to indicate type parameters. We also implicitly use a tuple type, but this is just for simplicity, as we could also just curry all uses of the codata type for functions.

The definition of the parametric codata type \texttt{Fun<A, B>} for functions from \texttt{A} to \texttt{B} could then be moved to a standard library and reused. Consider the usual higher-order syntax for function calls and the arrow syntax for function types as syntactic sugar for a call to the \texttt{apply} destructors and the respective instance of the parametric codata type \texttt{Fun<A, B>} (recursively), respectively. When desugaring the program in this way, we arrive back at the original program from the start (in the hypothetical higher-order language with copatterns).

Our example is straightforwardly generalized. Thus, when parametric polymorphism can be successfully brought to Uroboro, this variant of Uroboro would have the same expressiveness as many of the usual higher-order programming languages with parametric polymorphism.

\section{Future research}
\label{sec:futr}

Finally, we consider possible future research avenues. Uroboro is intended to not just be yet another language, but rather a programme with the goal of bringing useful automatic refactorings to higher-order programs. This programme has both theoretical and practical aspects to it. For both, we give some ideas where future research, starting from our work, could lead to.

On the practical side, our transformations can be implemented in an IDE; our Haskell library \texttt{uroboro-transformations} can be a basis for this. We believe that many of the concepts underlying integrated refactoring, as already successfully applied for many years, e.g., in the Eclipse IDE\footnote{\url{http://eclipse.org}}, can be carried over to Uroboro. Bäumer et al.\cite{baumer2001integrating} outline the three key steps when implementing a refactoring for Eclipse: detecting affected code, structural analysis of the program, and the actual changes to the code.

In an Uroboro IDE, on the one hand, the user might be able to interactively carry out individual extraction refactorings. Speaking in the terminology we developed in \autoref{ch:extr}, the user can select a target set of equations, then, just like in other IDEs, he can be informed whether the extraction can be carried out without problems. Such problems can be the introduction of overlaps, or that the selected equations aren't actually a target set in the sense of \autoref{ch:extr}. The user might also wish to determine the name of the auxiliary function created by the extraction.\footnote{Going one step further, as suggested by Klaus Ostermann, one might want to annotate those auxiliary functions, e.g., with the circumstances of their creation.} Compared with the steps described by Bäumer et al., the detection of affected code falls away since this is done manually by the user, and some structural analysis is needed to detect introduced overlaps or to verify the validity of the target.

On the other hand, the user might also request to de- or refunctionalize the program or some of its function definitions. In this case, as described in \autoref{ch:derefunc}, the extractions necessary for preprocessing are automatically chosen by the IDE. It might also be useful to display to the user whether the program is in a certain desired fragment of Uroboro, like the defunctionalized or refunctionalized fragment, or whether it is coverage complete. Again comparing this with the implementation steps of Bäumer et al., the detection of affected code needs to be done for each unnesting step, using the coverage derivation, while some structural analysis is necessary before the unnesting starts, to check whether unnesting and/or core de- or refunctionalization is even necessary, to derive the coverage\footnote{As suggested in \autoref{sec:relunn}, instead of this one might want to provide interactive program construction (as a feature of the IDE).} (if possible), and to determine whether a valid transformation is possible, which may not be the case when the program isn't coverage complete.

On the theoretical side, it might be desirable to automatize further already known transformations, to add to the toolbox of Uroboro transformations we started with this work. Examples of such transformations are CPS transformation and its converse, direct style transformation, as motivated in \autoref{ssec:defuncex} and \autoref{ssec:refuncex}, respectively.

We also have shone a light on the asymmetry between constructors and destructors in Uroboro in \autoref{sssec:asym}. The underlying problem directly relates to the difference between natural deduction and arbitrary sequent calculus, and it is not exclusive to Uroboro, but rather affects all current languages with copattern matching. Thus it might be beyond the principal scope of the Uroboro programme, at least in the short term, but nonetheless an interesting research problem.

\chapter{Conclusion}
\label{ch:concl}

To conclude our work, we summarize our contributions, highlighting their importance.

Building upon the work of \citet{rendel15automatic}, we have formally defined the language Uroboro (\autoref{ch:uro}). This language is interesting for the purpose of automatic program transformation since it isn't truly higher-order, but rather generalizes first-class functions to codata.

For Uroboro we have developed automatic program transformations. We have identified a subset of those transformations, which we call extractions (\autoref{ch:extr}); we have developed a formal description of them and we have shown how and under which circumstances they preserve the semantics of a program.

Using the concept of extractions, and building upon the work of \citet{rendel15automatic} and \citet{setzer14unnesting}, we have presented algorithms for automatic de- and refunctionalization (\autoref{ch:derefunc}). Such transformations find applications, e.g., with the meta-circular interpreter of \citet{reynolds72definitional}, and in general they can bring programs from a more (human-)understandable form to one with certain properties desirable with regards to computation. In the course of developing these transformations, we have encountered a specific problem concerning the asymmetry between constructors and destructors in Uroboro and other current languages with copatterns (\autoref{sssec:asym}).

Overall, we mostly regard our work as one brick in the building that the Uroboro programme is to become. But we have also shone a light on one problem which goes beyond the principal goals of that programme, namely the above mentioned asymmetry. Therefore, we hope that our work can be both followed up by other work in the Uroboro programme, as well as give an incentive to explore this, more general, asymmetry problem.
\appendix
\chapter{Proofs}

\correcta*
\begin{proof}
By induction on the length of the sequence.
TODO
\end{proof}

\correctb*
\begin{proof}
By induction on the length of the sequence.
TODO
\end{proof}

\clearpage
\newpage

\chapter{German translation of the abstract}

Rendel et al. präsentieren zwei funktionale Programmiersprachen auf denen Reynolds Defunktionalisierung beziehungsweise Danvys Refunktionalisierung total definiert sind. Diese Sprachen, genannt Codata Fragment und Data Fragment, unterstützen Abels Copattern Matching beziehungsweise das übliche Pattern Matching funktionaler Sprachen. Rendel et al. beabsichtigen, dass diese Sprachen Fragmente einer gemeinsamen Sprache mit dem Namen Uroboro sind, welche sowohl über Pattern als auch über Copattern Matching verfügt. In dieser Thesis definieren wir Uroboro formal und entwickeln automatische Programmtransformationen für diese Sprache. Wir erweitern Rendel et al.s automatische De- und Refunktionalisierung auf die gesamte Sprache Uroboro. Außerdem identifizieren wir eine Verallgemeinerung einiger der Schritte aus denen diese Transformationen aufgebaut sind, welche wir als ``Extraction'' bezeichnen. Als ein Nebenprodukt unserer Arbeit beleuchten wir eine Asymmetrie zwischen Pattern und Copattern Matching.


\bibliographystyle{plain}
\bibliography{bibliography}
\end{document}

